In questa tesi sono stati estratti numerosi dati da immagini aeree e satellitari riguardo il fiume Tagliamento e le isole che lo vegetano; si sono osservati numerosi trend spaziali e temporali riguardo la larghezza, la quantità di isole in alveo e le loro dinamiche.
Sono state analizzate più fasi della successione biogeomorfica: quali e quanti sono gli elementi legnosi che si trovano nelle condizioni di poter rigettare rami e radici, come si accrescono ed espandono le isole, complessivamente qual è la loro struttura d'età, quanto vengono erose da eventi di piena di diversa intensità.

Sono stati incontrati alcuni limiti durante lo studio di questi fenomeni naturali, di seguito elencati:
%
\begin{itemize}
	\item il legno si individua abbastanza facilmente, anche se non è chiara la distinzione del tipo di elementi legnosi, se questi stanno rivegetando o meno, se tra un'immagine e la successiva il legno è il medesimo o è stato rimpiazzato da elementi simili;
	inoltre le immagini satellitari ad alta risoluzione e le ortofoto sono risorse preziose, difficili da ottenere e costose, sebbene possano fornire una grande quantità di informazioni;
	%
	\item la formalizzazione di modelli concettuali che includono la crescita vegetativa a partire da elementi legnosi necessita l'approfondimento delle numerose relazioni che intercorrono tra quest'ultimi, la profondità della falda e il suo movimento, \emph{upwelling} e \emph{downwelling}, la disponibilità e il tipo di superfici da colonizzare;
	tutti questi fattori rendono il fenomeno della crescita molto più complesso da studiare di quello dell'erosione;
	%
	\item si è visto che anche le parti più vecchie delle isole, stabilitesi in alveo da diversi anni, possono essere erose da piene di piccola entità;
	questo avviene quando le isole si trovano all'estradosso di canali in curva;
	questo fattore non è stato considerato poiché, data la frequenza delle immagini, prima e durante ogni piena non è possibile conoscere la posizione di ogni canale attivo, se questo è in curva e se un'isola si trova vicino ad esso;
	in più, è stato suggerito che variazioni longitudinali della \emph{stream power} e della percentuale di vegetazione possono influenzare il movimento dei canali \squarecite{Arscott:2002-habitat-dynamics}.
\end{itemize}
%

Il confronto dei risultati ottenuti con dati in letteratura ha chiaramente mostrato come sia necessario definire e condividere procedure uniche e ripetibili per ogni mappa durante l'elaborazione e l'analisi della classificazione, del cambiamento e dell'età delle isole al fine di mantenere regole consistenti, come evidenziato da \squarecite{Zanoni:2008}.
Infatti, si è visto come il confronto delle medesime grandezze in diversi lavori possa portare a risultati leggermente diversi, se non a conclusioni contrastanti.
Nonostante ciò, è stato possibile osservare interessanti trend temporali per la larghezza e la quantità di isole in alveo.

Certamente, per poter esplorare la crescita e lo sviluppo delle isole, occorre sviluppare una metodologia per individuare e riconoscere in maniera automatica gli elementi legnosi; così facendo, sarebbe possibile osservare grandi porzioni di alveo in tempi brevi.
Inoltre, bisogna approfondire la conoscenza e la modellazione riguardo la crescita delle piante riparie, la formazione di isole, la loro coalescenza ed espansione.

È di sicuro interessante verificare se i risultati ottenuti siano trasferibili a fiumi con caratteristiche simili al Tagliamento, come il fiume Piave, i torrenti Meduna e Cellina; contemporaneamente e successivamente, si possono applicare le relazioni tra erosione della vegetazione e regime delle piene in contesti fortemente gestiti, come il fiume Brenta, dove l'eccessiva crescita della vegetazione sta alterando le dinamiche fluviali e sta diminuendo i servizi che l'uomo può sfruttare dagli ecosistemi ripari.

Con questa tesi è stato possibile analizzare estesamente ed efficacemente il Tagliamento, soprattutto grazie alle numerose immagini satellitari che enti governativi, quali la NASA e l'ESA, hanno messo liberamente a disposizione di ogni cittadino.
Negli anni futuri, nella speranza che queste fonti di dati inestimabili non vengano meno, è possibile proseguire e raffinare l'analisi svolta al fine di estendere e validare i risultati ottenuti e le conclusioni che si sono tratte da essi.
