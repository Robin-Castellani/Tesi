\section{Età della vegetazione nelle isole}
Utilizzando le mappe del cambiamento esperito dalle isole si possono ottenere mappe che indicano un'età approssimativa della vegetazione che compone queste isole. 
\\
L'ipotesi che si vuole verificare è che sia la vegetazione più giovane quella ad essere maggiormente erosa in quanto può essere divelta più facilmente.
Non si esclude tuttavia di osservare erosione anche di vegetazione matura poiché le isole poste in corrispondenza dell'estradosso di un canale curvo sono soggette a scavo laterale.

\subsection{Metodi: estendere i confronti e ottenere un'età}
\paragraph{Limiti}
Con le immagini satellitari utilizzate non è possibile distinguere i singoli alberi che compongono le isole; se con le immagini a miglior risoluzione si distinguono gli arbusti isolati, non si può certo ottenere una stima dell'età tanto precisa quanto quella fornita da un carotaggio del tronco.
In più, una cella è classificata come vegetazione solo quando le piante al suo interno presentano una chioma sufficientemente grande da occupare gran parte della cella; pertanto le immagini con celle di \SI{10}{\m} o addirittura \SI{15}{\m} non possono mostrare che grandi macchie di vegetazione.
Prima dei \SI{3}{\anni} una pianta di salice generalmente non ha fronde molto estese.
\\
Occorre poi tenere in conto che le piante crescono differentemente in base alle condizioni ambientali: periodi di stress idrico o termico rallentano la crescita, così come il parziale seppellimento con ghiaia dovuto ad eventi di piena; una falda non troppo profonda la cui altezza varia lentamente, terreno di granulometria sottile con sabbia e buone temperature sono invece fattori favorevoli alla crescita \squarecite{Gurnell:2001-island-formation}.

Si crede che ciò che possa accadere è che fino ad una certa età, circa \SI{3}{\anni}, la nuova vegetazione non sia visibile dal satellite. 

Le mappe del cambiamento sono state ottenute dalle mappe di classificazione; non essendo le seconde correttamente georeferenziate, neanche le prime lo sono. L'errore residuo nel processo di correzione è di 1~cella.

Infine, per poter confrontare nel corso degli anni ogni cella, le mappe del cambiamento sono state ricampionate alla risoluzione più bassa, corrispondente a celle di~\SI{15}{\m} di lato. Si è proceduto come mostrato nel paragrafo \ref{par:camb-limiti} ottenendo sempre valori $RSQR < \SI{1.5}{\percent}$; i confronti a~\SI{10}{\m} sono stati ricampionati a~\SI{15}{\m} applicando il valore del $75_\mathrm{mo}$ percentile (terzo quartile).
 

%TODO le piante crescono diversamente in base alle condizioni ambientali (cita articolo Gurnell?) → definire un'età in base al tempo di osservazione è un approccio limitato (qualche altro lavoro simile)

\paragraph{Obiettivo ed approccio} 
Date le precedenti premesse, bisogna riflettere su ciò che si vuole ottenere: dividere la vegetazione in classi di età.
Si vuole sapere quanta vegetazione giovane è stata erosa; non importa se questa ha esattamente \SI{3}{\anni} o \SI{4}{\anni}, l'importante è che sia grossomodo classificata come giovane.

Ogni mappa del cambiamento è formata dalle informazioni contenute in due mappe: quella più vecchia e quella più recente. La distanza temporale tra queste due mappe definisce il periodo di osservazione.
\\
Per la mappa del cambiamento più vecchia, l'età della vegetazione delle isole è stata arbitrariamente essere pari al periodo di osservazione, cui si sono aggiunti \SI{3}{\anni}.
\\
Per le mappe via via più recenti, l'età nelle celle delle isole che non sono cambiate è pari al periodo di osservazione sommato all'età precedente; per le celle si sono vegetate a partire dall'alveo l'età è pari al periodo di osservazione con l'aggiunta di \SI{3}{\anni}, ritenuto il periodo minimo affinché un insieme di piante diventi visibile per il satellite.

\medskip
L'approccio di definire un'età in base al periodo di osservazione è limitato, in particolare per le mappe più vecchie poiché non è possibile tener conto della vegetazione che era presente antecedentemente alla prima immagine valida;
si ritiene che a partire dalla $4^a$ o $5^a$ mappa di età il metodo inizi ad essere affidabile (generalmente quindi dalla mappa di età del 2007-09-21).
Per gli scopi della ricerca questo approccio risulta essere sufficiente.

\paragraph{Validazione}
Il CSM (\emph{Canopy Surface Model}) è il modello digitale della copertura arborea; similmente al DEM, mostra delle quote; queste sono tuttavia riferite all'altezza della vegetazione sopra il terreno.
\\
È generalmente sensato affermare che piante più mature abbiano un'altezza maggiore di piante giovani.
Quindi si sono utilizzate le informazioni del CSM relativo alla ortofoto del 2013-10-22 per verificare che nelle celle della mappa di età della vegetazione del 2013-09-05 l'altezza fosse all'incirca proporzionale all'età.
%TODO grafico con i quantili

\subsection{Risultati: un trend per la vegetazione erosa}
I grafici in \vref{graph:distr-eta} mostrano la distribuzione di età della vegetazione nelle isole in termini di areale occupato; per gli scopi del lavoro ci si limita a definire 3~classi di vegetazione:
%
\begin{itemize}
	\item giovane, con meno di \SI{5}{\anni};
	\item intermedia, con età compresa tra \SIrange[range-phrase={ e }]{5}{10}{\anni};
	\item matura, con più di \SI{10}{\anni}.
\end{itemize}

%TODO da migliorare (togliere la descrizione all'asse y?)
%TODO come rappresentare i dati? Una colonna per la veg presente e una per quella erosa? Come differenza con stack negative = separate?
\begin{figure}
	\centering
	\tikzsetnextfilename{eta_tratti_8_11_17}
\begin{tikzpicture}
	\begin{groupplot}[
		group style = {
			group size = 3 by 1,
			x descriptions at = edge bottom,
			y descriptions at = edge left,
			xticklabels at = all,
			horizontal sep = 0.2cm,
			vertical sep = 0.25cm,
		},
		width = 0.37\textwidth,
		height = 0.8\textwidth,
	    xbar stacked,
		enlarge x limits = 0.02,
		enlarge y limits = 0.10,
		symbolic y coords = {
			2008-07-05, 2009-07-08, 
			2012-08-01, 2013-09-05,  
			2016-09-13, 2017-06-13
		},
		ytick distance = 1,
		%scaled x ticks = false,
		xlabel = {Areale \si{[\m\tothe{2}]}},
		xmajorgrids = true,
		xlabel style = {xshift = -0.2cm},
		]
		\nextgroupplot % tratto_8
			\addplot[bar shift = .4cm, pattern = north east lines]
		       	table [y=data, x=giovane-e] {graphics/data/tr_8_eta.txt};
			\addplot[bar shift = .4cm, fill = green]
		       	table [
		       		y=data, 
		       		x expr=\thisrow{giovane} - \thisrow{giovane-e}
		       		] {graphics/data/tr_8_eta.txt};
		       		
			\resetstackedplots
			\addplot[bar shift = 0cm, pattern = north east lines, forget plot]
		       	table [y=data, x=interm-e] {graphics/data/tr_8_eta.txt};
			\addplot[bar shift = 0cm, fill = green!75!black]
		       	table [
		       		y=data, 
		       		x expr=\thisrow{interm} - \thisrow{interm-e}
		       		] {graphics/data/tr_8_eta.txt};
		       		
			\resetstackedplots
			\addplot[bar shift = -.4cm, pattern = north east lines, forget plot]
		       	table [y=data, x=matura-e] {graphics/data/tr_8_eta.txt};
			\addplot[bar shift = -.4cm, fill = green!40!black]
		       	table [
		       		y=data, 
		       		x expr=\thisrow{matura} - \thisrow{matura-e}
		       		] {graphics/data/tr_8_eta.txt};
		    
        	\node [fill = white, draw = black, anchor = north east] 
        		at (axis description cs: 1,1) {Tr. 8};
        %
		\nextgroupplot [% tratto_11
			legend style = {
				at = {(0.5,1.02)},
				legend columns = 4,
				anchor = south
			}, 
			]
			\addplot[bar shift = .4cm, pattern = north east lines]
		       	table [y=data, x=giovane-e] {graphics/data/tr_11_eta.txt};
		    \addlegendentry{Erosione}
			\addplot[bar shift = .4cm, fill = green]
		       	table [
		       		y=data, 
		       		x expr=\thisrow{giovane} - \thisrow{giovane-e}
		       		] {graphics/data/tr_11_eta.txt};
		    \addlegendentry{Giovane}
		       		
			\resetstackedplots
			\addplot[bar shift = 0cm, pattern = north east lines, forget plot]
		       	table [y=data, x=interm-e] {graphics/data/tr_11_eta.txt};
			\addplot[bar shift = 0cm, fill = green!75!black]
		       	table [
		       		y=data, 
		       		x expr=\thisrow{interm} - \thisrow{interm-e}
		       		] {graphics/data/tr_11_eta.txt};
		    \addlegendentry{Intermedia}
		       		
			\resetstackedplots
			\addplot[bar shift = -.4cm, pattern = north east lines, forget plot]
		       	table [y=data, x=matura-e] {graphics/data/tr_11_eta.txt};
			\addplot[bar shift = -.4cm, fill = green!40!black]
		       	table [
		       		y=data, 
		       		x expr=\thisrow{matura} - \thisrow{matura-e}
		       		] {graphics/data/tr_11_eta.txt};
		    \addlegendentry{Matura}
		    
        	\node [fill = white, draw = black, anchor = north east] 
        		at (axis description cs: 1,1) {Tr. 11};
        	%
        	
		\nextgroupplot % tratto_17
			\addplot[bar shift = .4cm, pattern = north east lines]
		       	table [y=data, x=giovane-e] {graphics/data/tr_17_eta.txt};
			\addplot[bar shift = .4cm, fill = green]
		       	table [
		       		y=data, 
		       		x expr=\thisrow{giovane} - \thisrow{giovane-e}
		       		] {graphics/data/tr_17_eta.txt};
		       		
			\resetstackedplots
			\addplot[bar shift = 0cm, pattern = north east lines, forget plot]
		       	table [y=data, x=interm-e] {graphics/data/tr_17_eta.txt};
			\addplot[bar shift = 0cm, fill = green!75!black]
		       	table [
		       		y=data, 
		       		x expr=\thisrow{interm} - \thisrow{interm-e}
		       		] {graphics/data/tr_17_eta.txt};
		       		
			\resetstackedplots
			\addplot[bar shift = -.4cm, pattern = north east lines, forget plot]
		       	table [y=data, x=matura-e] {graphics/data/tr_17_eta.txt};
			\addplot[bar shift = -.4cm, fill = green!40!black]
		       	table [
		       		y=data, 
		       		x expr=\thisrow{matura} - \thisrow{matura-e}
		       		] {graphics/data/tr_17_eta.txt};
		    
        	\node [fill = white, draw = black, anchor = north east] 
        		at (axis description cs: 1,1) {Tr. 17};
	\end{groupplot}
\end{tikzpicture}




	\caption[areale delle classi d'età per i tratti~8, 11 e~17]{areale delle tre classi di età per alcune immagini dei tratti~8, 11 e~17.}% \numlist[list-final-separator={ e }]{8;11;17}.}
	\label{graph:distr-eta}
\end{figure}
%

Focalizzandosi sugli eventi di piena della fine del~2008 e del~2012, si vede chiaramente come l'areale della vegetazione giovane sia fortemente diminuito dalle immagini antecedenti le piene alle immagini successive (tratti \numrange[range-phrase={ e }]{11}{17});
sia in termini relativi che in termini assoluti, la vegetazione giovane è quella che è prevalentemente portata via dalle piene, quando è presente;
in seguito alle due piene considerate si osserva inoltre una diminuzione dell'area occupata dalle isole.
Si tenga comunque in conto che parte della vegetazione non erosa invecchia e può passare da una classe a quella successiva; questo fenomeno non è tuttavia quello predominante durante le piene del~2008 e del~2012 per evidenza.
\\
Con il confronto tra anni in cui non c'è stata una forte espansione delle isole ma in cui ci sono stati particolari eventi, come tra il \numrange[range-phrase={ e il }]{2014}{2015} o tra il \numrange[range-phrase={ e il }]{2017}{2018}, non si nota un cambiamento particolare nella distribuzione d'età.
%Anzi, nell'undicesimo tratto, dove sono presenti grandi isole, la piena ne ha probabilmente asportato alcune con età intermedia

% magari secondo grafico con stesso tratto, stesse immagini ma solo vegetazione erosa

















