Con i dati a disposizione è possibile ricercare delle relazioni che leghino il regime delle piene con la dinamica delle isole.
In quanto l'erosione delle isole è solamente legata agli effetti delle piene, mentre l'accrescimento è influenzato da più fattori (come è stato spiegato nella sezione~\ref{sec:descr-area-studio}), l'analisi esclude quest'ultimo.

\section{Metodi: parametri considerati}

\paragraph{Isole erose}
Dallo studio dei cambiamenti esperiti dalle isole tra un'immagine e quella successiva si sono ottenuti i dati di areale di isole erose (sezione~\ref{sec:cambiamento});
questi dati sono stati suddivisi secondo tre classi di età della vegetazione: giovane, intermedia e matura (sezione~\ref{sec:eta}).
Quindi è nota la quantità di isole erose a cavallo di ogni immagine, in ogni tratto e per ogni classe di età.

\paragraph{Integrale dei livelli}
Dai dati di livello dell'acqua presso la stazione idrometrica di Villuzza, degli autori hanno ottenuto una statistica dei tempi di ritorno dei picchi delle piene superiori ad \SI{1}{\m} \squarecite{Bertoldi:2009-2m}; questa è stata estesa utilizzando i dati di livello idrometrico dal~2000-01-01 al~2018-12-21 ed è riportata nel grafico in \cref{graph:tr-picchi}; in totale sono stati individuati poco più di~200 picchi.
%
\begin{figure}
	\centering
	\tikzsetnextfilename{tr_picchi}
\begin{tikzpicture}
	\begin{semilogxaxis}[
		width = \textwidth,
		height = 0.5\textwidth,
%		enlarge x limits = 0.05,
%		enlarge y limits = 0.01,
%		ytick distance = 0.5,
		ylabel = {Livello idrometrico \si{[\m]}},
		xlabel = {Tempo di ritorno \si{[\anni]}},
		y tick label style = {
			/pgf/number format/.cd,
			fixed,
			fixed zerofill,
			precision = 1,
			/tikz/.cd,
		},
		grid = major,
        log ticks with fixed point,		
		]
		\addplot
        	[blue, no markers]
        	table [x=tr_anni, y=picchi] {graphics/data/tr_picchi.txt};
        
        \draw[<->] (0.1,3) -- (0.5,3);
        \node [at = {(0.23,3)}, anchor = north] {\emph{Flow pulses}};
        
        \draw[<->] (0.5,1) -- (3,1);
        \node [at = {(1.3,1)}, anchor = south] {\emph{Flood pulses}};
        
        \draw[<->] (3,1) -- (20,1);
        \node [at = {(7,1)}, anchor = south] {\emph{Bankfull}};
	\end{semilogxaxis}
\end{tikzpicture}
	\caption[tempi di ritorno dei picchi superiori ad \SI{1}{\m}]{tempi di ritorno dei picchi superiori ad \SI{1}{\m} ottenuti dall'individuazione di più di 200 picchi dai dati idrometrici.}
	\label{graph:tr-picchi}
\end{figure}
%
\\
Non sono stati utilizzati i massimi annuali poiché piene con tempo di ritorno inferiore ad \SI{1}{\anno} hanno anch'esse importi effetti sulla morfologia del fondo e sulle isole.
%Inoltre, la statistica effettuata da \squarecite{Bertoldi:2009-2m} utilizzava i dati dal 1981 al 2007.
% Non sono stati utilizzati dati idrometrici anteriori al 1981 poiché precedentemente i livelli erano manualmente letti su un'asta graduata in un singolo momento della giornata; si capisce che questi dati sono molto meno affidabili di quelli rilevati con intervallo orario o semi orario, che sono in grado di descrivere dettagliatamente il passaggio di ogni piena.

Con questa statistica è possibile tenere conto di diversi tipi di eventi (in accordo con quanto riportato in letteratura \squarecites{Bertoldi:2009-2m}{Bertoldi:2010-d50}{Surian:2015}):
%
\begin{itemize}
	\item \emph{flow pulses}, cioè piene di modesta entità, sono quelle con un livello inferiore ai \SI{2}{\m}, e quindi con un tempo di ritorno di circa \SI{6}{\mesi}; questi periodi di morbida possono rimodellare il fondo, anche se generalmente non riescono a sommergere le isole insediatesi in alveo da anni;
	\item \emph{flood pulses}, piene intense, superano il livello di \SI{2}{\m} e arrivano fino a \SI{3}{\m}, presentando quindi un tempo di ritorno dell'ordine di \SI{1}{\anno}; queste piene hanno effetti sulla vegetazione in quanto riescono ad erodere lateralmente le isole e a sommergere quelle a quote relative più basse;
	\item eventi \emph{bankfull}, cioè piene di grande magnitudine, oltrepassano i \SI{3}{\m} di livello idrometrico e hanno un tempo di ritorno superiore a \SIrange[range-phrase={-}]{3}{4}{\anni}; l'alveo viene completamente sommerso dall'acqua, così come la stragrande maggioranza delle isole.
\end{itemize}
%

Si ritiene che gli effetti di una piena sulle isole non dipendano soltanto dall'intensità (cioè dal livello raggiunti durante il picco) ma anche dalla durata: a parità di picco, un evento più lungo probabilmente eroderà un maggiore areale di isole di un evento molto breve.
\\
Per tenere contemporaneamente conto dell'intensità, della durata e del tempo di ritorno delle piene, si è calcolato per ogni intervallo di tempo tra due immagini successive l'integrale temporale dei livelli sopra un livello soglia definito dal grafico in \cref{graph:tr-picchi} scegliendo un tempo di ritorno.
Scalando la coordinata temporale per mostrare i giorni, l'integrale si misura in \si{[\m\giorno]} ed è equivalente ad un evento di piena di durata pari ad~\SI{1}{\giorno} e livello sopra soglia pari all'integrale.
Poiché non è possibile ottenere un'affidabile scala di deflusso per ottenere i valori di portata dai livelli, l'integrale costituisce un sostituto alla somma delle portate sopra una soglia che fluiscono durante una piena.
\\
Un esempio di integrale per due livelli soglia è mostrato nei grafici in \cref{graph:esempio-integrale-livelli}.
\\
Se tra due immagini successive l'integrale è nullo, allora non ha avuto luogo alcun evento con picco superiore al livello soglia corrispondente al tempo di ritorno scelto;
se l'integrale non è nullo, allora ci sono state piene sopra il livello soglia;
se un'integrale è maggiore di altri, nell'intervallo tra le immagini sono avvenute una o più piene particolarmente intense e/o durature.
%
\begin{figure}
	\centering
	\tikzsetnextfilename{esempio_integrale_livelli}
\begin{tikzpicture}
	\begin{groupplot}[
		group style = {
			group size = 2 by 2,
			y descriptions at = edge left,
			xlabels at = edge top,
			horizontal sep = 0.5cm,
			vertical sep = 0.5cm,
		},
		width = 0.5\textwidth,
		height = 0.5\textwidth,
%		ymin = 0.5,
%		ymax = 3.5,
		enlarge y limits = 0.05,
		enlarge x limits = 0.05,
		ylabel = {Livello idrometrico \si{[\m]}},
		xlabel = {Tempo di ritorno \si{[\anni]}},
		]
	
	\nextgroupplot[
			xmode = log,
        	log ticks with fixed point,
			grid = major,
        ]
		\addplot[
			blue,
			no markers,
			]
        	table [x = tr_anni, y = picchi] {graphics/data/tr_picchi.txt};
        	
        \draw[->, orange, very thick] (0.2,1) -- (0.2,1.5) -- (0.1,1.5);
		
		\node at (axis description cs: 1,0) [draw = black, fill = white, anchor = south east, align = left] {TR \SIrange[range-phrase={-}, range-units = single]{2}{3}{\mesi} \\ Livello \SI{1.5}{\m}};
	
	\nextgroupplot[
			xmode = log,
        	log ticks with fixed point,
			grid = major,
        ]
		\addplot[
			blue,
			no markers,
			]
        	table [x = tr_anni, y = picchi] {graphics/data/tr_picchi.txt};
        	
        \draw[->, green!70!black, very thick] (0.45,1) -- (0.45,2) -- (0.1,2);
		
		\node at (axis description cs: 1,0) [draw = black, fill = white, anchor = south east, align = left] {TR \SIrange[range-phrase={-}, range-units = single]{4}{5}{\mesi} \\ Livello \SI{2}{\m}};
    
	\nextgroupplot[
			date coordinates in = x,
			xticklabel = {$\year-\month-\day$},
			xticklabel style = {
				rotate = 80,
				anchor = near xticklabel,
				font = \footnotesize,
			},
			xmin = 2002-05-18 00:00,
			xmax = 2002-06-12 23:30,
		]
		\addplot[
			blue,
			no markers,
			name path = livelli,
			]
        	table [x = data, y = livello, col sep = comma] {graphics/data/Idro_primo_intervallo.txt};
        	
        \addplot[
        	dashed,
        	very thick,
        	orange,
        	name path = soglia,
        	]
        	coordinates {(2002-05-18 00:00, 1.5) (2002-06-12 23:30, 1.5)};
        	
		\addplot fill between [
			of = soglia and livelli,
			split,
			every segment no 0/.style = {white},
			every segment no 1/.style = {orange},
			every segment no 2/.style = {white},
			every segment no 3/.style = {orange},
			every segment no 4/.style = {white},
			every segment no 5/.style = {white},
		];
		
		\node at (axis description cs: 1,0) [draw = black, fill = white, anchor = south east,] {$Int = \SI{0.90}{\m\giorno}$};
    
	\nextgroupplot[
			date coordinates in = x,
			xticklabel = {$\year-\month-\day$},
			xticklabel style = {
				rotate = 80,
				anchor = near xticklabel,
				font = \footnotesize,
			},
			xmin = 2002-05-18 00:00,
			xmax = 2002-06-12 23:30,
		]
		\addplot[
			blue,
			no markers,
			name path = livelli,
			]
        	table [x = data, y = livello, col sep = comma] {graphics/data/Idro_primo_intervallo.txt};
        	
        \addplot[
        	dashed,
        	very thick,
        	green!70!black,
        	name path = soglia,
        	]
        	coordinates {(2002-05-18 00:00, 2) (2002-06-12 23:30, 2)};
        	
		\addplot fill between [
			of = soglia and livelli,
			split,
			every segment no 0/.style = {white},
			every segment no 1/.style = {green!70!black},
			every segment no 2/.style = {green!70!black},
			every segment no 3/.style = {white},
			every segment no 4/.style = {white},
		];
		
		\node at (axis description cs: 1,0) [draw = black, fill = white, anchor = south east,] {$Int = \SI{0.12}{\m\giorno}$};
	\end{groupplot}
\end{tikzpicture}	
	\caption[esempio di integrale dei livelli]{esempio di integrale dei livelli per due tempi di ritorno considerando l'intervallo tra le immagini \AST{} del 2002-05-18 e del 2002-06-12; l'integrale è pari all'area colorata ed è riportato nei grafici in basso.}
	\label{graph:esempio-integrale-livelli}
\end{figure}
%

