\section{Potenza della corrente}
La potenza della corrente per unità di larghezza e di profondità $\Omega$ in \si{[\watt\per\meter\tothe{2}]} è definita come:
%
\begin{equation}
	\label{eq:omega}
	\Omega = \frac{\rho \, g \, Q \, i_f}{B}
\end{equation}
%
dove:
%
\begin{itemize}
	\item $\rho$ è la densità dell'acqua, pari a \SI{1000}{\kilo\g\per\meter\tothe{3}};
	\item $g$ è l'accelerazione gravitazionale, pari a \SI{9.81}{\m\per\s\tothe{2}};
	\item $Q$ è la portata fluente nel canale misurata in \si{[\m\tothe{3}\per\s]};
	\item $i_f$ è la pendenza in \si{[\m\per\m]};
	\item $B$ è la larghezza del canale in \si{[\m]}.
\end{itemize}
%
In letteratura sono stati proposti modelli concettuali che relazionano la potenza della corrente con la quantità di vegetazione e il tipo di forma vegetata presente:
a bassi livelli di $\Omega$, il disturbo indotto dalla corrente è modesto e si possono formare numerose isole sulle barre nude in alveo;
se la potenza della corrente, cioè il disturbo, è maggiore, si potranno formare meno isole e le forme fluviali rimarranno prevalentemente nude\squarecite{Gurnell:2014-plants-eng}.
\\
Difatti l'espansione maggiore della vegetazione e l'erosione minore hanno luogo dove il tasso di crescita della vegetazione è elevato e dove l'energia della corrente è ridotta \squarecite{Gurnell:2006-omega}.
\\
Inoltre le piante che crescono più rapidamente sembrano essere maggiormente flessibili \squarecite{Bertoldi:2011-ASTER}: la ricrescita vegetativa da tronchi vivi, la quale permette un rapido sviluppo, è proprio la caratteristica fondamentale delle piante che abitano questo ambiente, come già esposto nella sezione~\ref{sec:descr-area-studio}.
È dunque lecito supporre che proprio nei tratti dove si osserva una ridotta potenza della corrente si possa trovare un'elevata quantità di isole nella maggior parte del periodo di studio.


