\tikzsetnextfilename{erosione_matrix}
\begin{tikzpicture}
	\begin{axis}[
		width = \textwidth,
		height = \textwidth,
		symbolic x coords = {2000-09-17, 2001-06-07, 2002-05-18, 2002-06-12, 2003-06-22, 2004-10-14, 2005-08-30, 2006-07-16, 2007-09-21, 2008-07-05, 2009-07-08, 2010-09-29, 2011-10-02, 2012-08-01, 2013-09-05, 2014-09-08, 2014-10-31, 2015-08-13, 2015-09-12, 2015-10-22, 2016-09-13, 2017-04-21, 2017-06-13, 2018-06-15, 2018-09-16},
		xticklabel style = {
			rotate = 90,
		},
		xtick distance = 1,
		ymin = 1,
		ymax = 23,
		ytick = data,
		ylabel = {Tratto},
		y dir = reverse,
		enlargelimits = 0.02,
		colorbar horizontal,
		colorbar style = {xlabel = {Erosione rispetto alle isole inizialmente presenti},},
		]
		\addplot[
			matrix plot*,
			mesh/cols = 23,	% per fargli leggere colonne formate da 23 righe dal file di testo
			shader = flat corner,	% per interpolare i colori
			point meta max = 1,
		]
        	table [y = tratto, x = data, point meta = \thisrow{cambiamento}] {graphics/data/erosione_matrix.txt};
    \end{axis}
\end{tikzpicture}