\section{Metodi: mappare gli elementi legnosi in alveo}
Le piante che caratterizzano l'ambiente ripario nell'area di studio (\emph{Salix spp.} e \emph{Populus nigra}) sono in grado di riprodursi tramite la dispersione di semi;
tuttavia, la riproduzione vegetativa a partire da elementi legnosi, quali pezzi di tronchi, piante e arbusti e accumuli di legname, ottiene un successo maggiore e costituisce il fondamentale meccanismo di propagazione, formazione ed espansione delle isole \squarecite{Gurnell:2001-island-formation}.
\\
La dimensione degli elementi legnosi è molto variabile: da frammenti vegetativi a piccole piante deposte sulle barre, da alberi alti diversi metri a consistenti accumuli di legname.
Con le immagini a disposizione, solo le ortofoto, le immagini \Pl{} e le immagini \WV{} hanno una risoluzione sufficientemente elevata da permettere il riconoscimento del legno.
\\
Si è proceduto quindi digitalizzando manualmente le posizioni di tutti gli elementi legnosi in una porzione di alveo compresa tra il tratto~7 e l'8 per le immagini ad alta risoluzione del~2010-08, 2013-10-22, 2014-10-31, 2015-08-13 e~2017-06-26/08-02; queste hanno un errore di georeferenziazione nell'ordine di una decina di centimetri.
\\
Non si sono digitalizzate altre aree oltre a quella scelta poiché sembrano essere meno ricche di legno; non è stato possibile implementare un sistema di riconoscimento automatico o semi-automatico neanche per le immagini satellitari multibande poiché il colore degli elementi legnosi è molto variabile e non è stato possibile legarlo con una combinazione di diverse bande come nel caso delle isole con l'NDVI.
Non sono state considerate le immagini del~2005-05 e~2011-06-26/07-02 poiché la prima non ha una qualità tale da poter distinguere bene gli elementi più piccoli e la seconda non è perfettamente georeferenziata e tentativi di rigeoreferenziazione come quelli eseguiti precedentemente sulle immagini \AST{} hanno dato risultati poco soddisfacenti.
\\
Sono stati digitalizzati più di un migliaio di elementi legnosi per ogni immagine considerata.
La \cref{fig:digitalizzazione-legno} mostra l'area dove si è effettuata la digitalizzazione e un esempio di individuazione degli elementi legnosi.
%
\begin{figure}
	\centering
	\includegraphics[width = \textwidth]{files/digitalizzazione_legno.jpeg}
	\caption[area di digitalizzazione degli elementi legnosi]{a sinistra l'area di digitalizzazione degli elementi legnosi (in marrone) e la maschera computazionale divisa in tratti (in blu); a destra un esempio di digitalizzazione in cui sono stati distinti i tronchi (in arancione) dagli accumuli (in viola); sullo sfondo è presente l'ortofoto del~2013-10-22.}
	\label{fig:digitalizzazione-legno}
\end{figure}
%
\\
In seguito per ogni digitalizzazione si è calcolata la distanza tra ogni punto e quello più vicino nella digitalizzazione successiva. Ad esempio, per ogni elemento legnoso nella mappa del~2014-10-31 è stato ottenuta la distanza dall'elemento legnoso più prossimo nella mappa del~2015-08-13.
\\
Infine per le ortofoto associate ad un rilievo LiDAR (il~2010-08 e il~2013-10-22) si è ottenuta la quota di ogni punto rispetto al DEM privo dell'effetto della pendenza (questo viene calcolato sottraendo al DEM la quota media in quella zona, cioè eliminando l'effetto della pendenza; in questo modo le quote sono riferite ad uno zero locale e mostrano chiaramente zone incise, come i canali, e zone elevate, come le barre).
