\section{Potenza della corrente}
La potenza della corrente per unità di larghezza $\Omega$ in \si{[\watt\per\meter\tothe{2}]} è definita come:
%
\begin{equation}
	\label{eq:omega}
	\Omega = \frac{\rho \, g \, Q \, i_f}{B}
	\quad
	\si{[\watt\per\meter\tothe{2}]}
\end{equation}
%
dove:
%
\begin{itemize}
	\item $\rho$ è la densità dell'acqua, pari a \SI{1000}{\kilo\g\per\meter\tothe{3}};
	\item $g$ è l'accelerazione gravitazionale, pari a \SI{9.81}{\m\per\s\tothe{2}};
	\item $Q$ è la portata fluente nel canale misurata in \si{[\m\tothe{3}\per\s]};
	\item $i_f$ è la pendenza in \si{[\m\per\m]};
	\item $B$ è la larghezza del canale in \si{[\m]}.
\end{itemize}
%
In letteratura sono stati proposti modelli concettuali che relazionano la potenza della corrente con la quantità di vegetazione e il tipo di forma vegetata presente:
a bassi livelli di $\Omega$, il disturbo indotto dalla corrente è modesto e si possono formare numerose isole sulle barre nude in alveo;
se la potenza della corrente, cioè il disturbo, è maggiore, si potranno formare meno isole e le forme fluviali rimarranno prevalentemente nude \squarecite{Gurnell:2014-plants-eng}.
\\
Difatti l'espansione maggiore della vegetazione e l'erosione minore hanno luogo dove il tasso di crescita della vegetazione è elevato e dove l'energia della corrente è ridotta \squarecite{Gurnell:2006-omega}.
\\
Inoltre le piante che crescono più rapidamente sembrano essere maggiormente flessibili \squarecite{Bertoldi:2011-ASTER}: la ricrescita vegetativa da tronchi vivi, la quale permette un rapido sviluppo, è proprio la caratteristica fondamentale delle piante che abitano questo ambiente, come già esposto nella sezione~\ref{sec:descr-area-studio}.
È dunque lecito supporre che proprio nei tratti dove si osserva una ridotta potenza della corrente si possa trovare un'elevata quantità di isole nella maggior parte del periodo di studio.


\subsection{Metodi: calcolare la potenza della corrente}
Per ottenere la \emph{stream power} si è prima ottenuta la pendenza media di ogni tratto grazie al DEM del 2009 nella seguente maniera:
%
\begin{aenumerate}
	\item si è definita una coordinata curvilinea che segue il corso principale del fiume;
	\item sono state considerate le quote in tutti i punti sopra i quali scorre la coordinata curvilinea;
	\item è stata effettuata una regressione lineare tra la coordinata curvilinea e le quote raggruppando i tratti quattro a quattro (tranne per l'ultimo gruppo, formato dai tratti 21, 22 e 23);
	\item il coefficiente angolare della retta rappresenta la pendenza del gruppo di 4 tratti.
\end{aenumerate}
%
Il risultato è riportato nella \cref{tab:pendenza}.
%
\begin{table}
	\centering
	\begin{tabular}{
		S[list-separator={, }, list-final-separator={ e }]
		S[table-format=1.2]
	}
		\toprule
		\multicolumn{1}{c}{Tratti}	&	\multicolumn{1}{c}{Pendenza \si{[\percent]}}	\\
		\midrule
		\numlist{1;2;3;4}	&	0.52	\\
		\numlist{5;6;7;8}	&	0.48	\\
		\numlist{9;10;11;12}	&	0.23	\\
		\numlist{13;14;15;16}	&	0.40	\\
		\numlist{17;18;19;20}	&	0.31	\\
		\numlist{21;22;23}	&	0.25	\\
		\bottomrule
	\end{tabular}
	\caption[pendenze dei tratti]{pendenze dei tratti.}
	\label{tab:pendenza}
\end{table}
%
\\
In quanto modificazioni della pendenza avvengono in periodi di tempo di anni soprattutto in seguito ad eventi naturali e non di apporto o asporto di sedimenti, che non sono avvenuti in maniera diffusa ed intensa durante gli ultimi due decenni, si considera la pendenza ottenuta come rappresentativa e costante nel periodo di studio.
In più, questi valori sono in accordo con quanto presente in letteratura \squarecites{Arscott:2002-habitat-dynamics}{Gurnell:2006-omega}{Bertoldi:2010-d50}{Sitzia:2016-d50}.

Mentre la larghezza $B$ è nota per quasi ogni tratto in ogni immagine, la portata $Q$ non è nota.
Secondo quanto descritto nella sezione \ref{sec:descr-area-studio},
%TODO verifica questo riferimento
la percentuale di area di bacino drenante in ogni tratto può fornire un'informazione sulla portata fluente.
\\
Poiché i dati di livello alla stazione idrometrica di Villuzza rappresentano in maniera affidabile l'entità delle piene, si è deciso di riferire l'area drenante in ogni tratto a quella del tratto~12, posto immediatamente a monte del sensore idrometrico (si veda la \cref{fig:overview-sat} e la \cref{fig:23-tratti}).
In questo modo si suppone che, durante le piene, in ogni tratto scorra una portata che è proporzionale alla percentuale di bacino drenante rispetto al bacino sotteso alla stazione idrometrica; tale percentuale è minore dell'unità a monte della stazione, mentre è maggiore all'unità a valle della stazione.
Per definire l'area drenante in ogni tratto sono stati utilizzati i dati riguardo gli affluenti mostrati nell'introduzione.
\\
L'equazione~\eqref{eq:omega-finta} definisce formalmente la potenza della corrente fittizia utilizzata nel presente lavoro.
Per comodità e semplicità, la definizione e nomenclatura della potenza della corrente originale viene sostituita dalla potenza della corrente fittizia.
La nuova $\Omega$ si misura in \si{[\newton\per\metre\tothe{4}]}.
%
\begin{equation}
	\label{eq:omega-finta}
	\Omega = \frac{A_\mathrm{tr}}{A_\mathrm{rif}} \frac{\rho \, g \, i_f}{B}
	\quad
	\si{[\newton\per\metre\tothe{4}]}
\end{equation}
%
dove:
%
\begin{itemize}
	\item $A_\mathrm{tr}$ è l'area drenante di ogni tratto in \si{[\m\tothe{2}]};
	\item $A_\mathrm{rif}$ è l'area drenante del tratto~12, esattamente a monte del sensore idrometrico di Villuzza, pari a \SI{2204}{\m\tothe{2}}.
\end{itemize}
%
Si noti che se durante una piena si avesse una misura affidabile di portata, moltiplicandola per questa $\Omega$ si otterrebbe la potenza della corrente precedentemente definita.

\subsection{Risultati: come la potenza della corrente limita la crescita delle isole}
Il grafico in \cref{graph:omega-perc-50} mostra la mediana temporale della potenza della corrente per ogni tratto; si rappresenta la mediana poiché la variazione di $\Omega$ nel tempo è trascurabile (pochi punti percentuali).
Questo grafico riflette sia le caratteristiche morfologiche (pendenza e larghezza) che idrologiche (percentuale di area drenante).
%
\begin{figure}
	\centering
	\tikzsetnextfilename{omega_perc_50}
\begin{tikzpicture}
	\begin{axis}[
		width = 0.95\textwidth,
		height = .5\textwidth,
		enlarge x limits = 0.01,
		%enlarge y limits = 0.01,
		xlabel = {Tratto},
		ylabel = {Potenza della corrente \si{[\newton\per\metre\tothe{4}]}},
		xtick = data,
		grid = major,
		yticklabel style = {
			/pgf/number format/fixed
		},
		]
		\addplot[only marks, mark = *]
        	table [x = tratto, y = omega,] {graphics/data/omega_perc_50.txt};
	\end{axis}
\end{tikzpicture}
	\caption[potenza della corrente in ogni tratto]{mediana temporale della potenza della corrente per ogni tratto pesata con la relativa area drenante.}
	\label{graph:omega-perc-50}
\end{figure}
%
\\
È possibile osservare come i tratti più a monte abbiano una \emph{stream power} elevata grazie alla forte pendenza e alla ridotta larghezza; a monte del tratto~3 confluisce il Fella, il maggiore affluente del Tagliamento, e il suo contributo in termini di area drenante è evidente.
Più a valle, dove l'alveo si allarga a la pendenza diminuisce, $\Omega$ cala; tuttavia a monte e a valle della stretta di Pinzano (tratti~12 e~13, rispettivamente) il brusco restringimento incrementa la potenza della corrente.
Nei tratti planiziali $\Omega$ non varia particolarmente, mentre nell'ultimo tratto, dove la morfologia diventa di tipo transizionale e l'alveo si restringe sensibilmente, il valore di $\Omega$ quasi triplica.

Rappresentando la potenza della corrente rispetto alla proporzione di isole sull'alveo attivo, si vede un andamento iperbolico: per elevate; utilizzando per entrambi gli assi una scala logaritmica (\cref{graph:omega-area-percentuale}), è possibile osservare l'effetto di controllo sulla quantità massima di vegetazione esercitato da $\Omega$.
\\
Da queste analisi è stato escluso il tratto~9, dove è presente l'isola di Cornino, poiché quest'isola è fondata su roccia e non è soggetta alle stesse dinamiche delle altre isole.
%
\begin{figure}
	\centering
	\tikzsetnextfilename{omega_area_percentuale_linear}
\begin{tikzpicture}
	\begin{axis}[
		width = \textwidth,
		height = .7\textwidth,
		enlarge x limits = 0.01,
		%enlarge y limits = 0.01,
		ylabel = {$\Omega$ \si{[\newton\per\metre\tothe{4}]}},
		xlabel = {Isole rispetto all'alveo attivo},
		xtick distance = 0.04,
		grid = major,
		legend columns = -1,
		legend style = {
			anchor = south,
			at = {(0.5, 1.01)},
		},
		ticklabel style = {
			/pgf/number format/.cd,fixed,
		},
		]
		\foreach \tratto in {1,2,...,23}
			{
			\addplot[
				only marks,
				forget plot,
			]
				table [y = om_tr_\tratto, x = area_tr_\tratto]
				{graphics/data/omega_area_percentuale.txt};
			}
		
%		\addplot [color = green, % T1
%			 line width = 2 pt,
%			 domain = 1e-3:4e-1,
%			 samples = 10,
%			 ] 
%			 {10^(-1.5184 - 0.1838 * log10(x))};
%		\addlegendentry{Fit 1}
%%		\node [fill = white, draw = green, anchor = east] % T1 
%%        	at (axis description cs: 1,0.6) {$y = 10^{-1.5184} \, x^{- 0.1838}$};
%        	
%		\addplot [color = orange, % T2
%			 line width = 2 pt,
%			 domain = 1e-3:4e-1,
%			 samples = 10,
%			 ]
%			 {10^(-1.5056 - 0.2516 * log10(x))};
%		\addlegendentry{Fit 2}
%%		\node [fill = white, draw = orange, anchor = east] % T2 
%%        	at (axis description cs: 1,0.75) {$y = 10^{-1.5056} \, x^{- 0.2516}$};
%        	
%		\addplot [color = cyan, % T3
%			 line width = 2 pt,
%			 domain = 1e-3:4e-1,
%			 samples = 10,
%			 ]
%			 {10^(-1.4085 - 0.2371 * log10(x))};
%		\addlegendentry{Fit 3}
%		\node [fill = white, draw = cyan, anchor = east] % T3 
%        	at (axis description cs: 1,0.9) {$y = 10^{-1.4085} \, x^{- 0.2371}$};
%        
%		\node [fill = white, draw = black, anchor = south west] % T3 
%        	at (axis description cs: 0,0) {$R^2 \in [0.3, 0.6]$ $P_\mathrm{val} < 0.0001$};
	\end{axis}
\end{tikzpicture}
	\caption[potenza della corrente rispetto alla proporzione di isole sull'alveo attivo, grafico lineare]{potenza della corrente rispetto alla proporzione di isole sull'alveo attivo.}
	\label{graph:omega-area-percentuale-linear}
\end{figure}
%
\begin{figure}
	\centering
	\tikzsetnextfilename{omega_area_percentuale}
\begin{tikzpicture}
	\begin{axis}[
		width = \textwidth,
		height = .5\textwidth,
		%enlarge x limits = 0.01,
		%enlarge y limits = 0.01,
		ylabel = {$\Omega$ \si{[\newton\per\metre\tothe{4}]}},
		xlabel = {Isole rispetto all'alveo attivo},
		grid = major,
		legend columns = -1,
		legend style = {
			anchor = south,
			at = {(0.5, 1.01)},
		},
		colormap = {fitting point colormap}{
				color = (black)
				color = (white!80!black)
%				color = (cyan!75!black)
%				color = (orange!75!black)
%				color = (green!75!black)
            },
        log ticks with fixed point,
        xmode = log,
        ymode = log,
        log basis x = 10,
        log basis y = 10,
		]
		\foreach \tratto in {1,2,...,23}
			{
			\addplot[
				scatter,
				only marks,
				point meta = {ifthenelse(y < -1.5184-0.1838*x, 1, ifthenelse(y < -1.5056 - 0.2516 * x, 0.7, 0))},
				point meta max = 1,
				point meta min = 0,
				forget plot,
			]
				table [y = om_tr_\tratto, x = area_tr_\tratto]
				{graphics/data/omega_area_percentuale.txt};
			}
		
		\addplot [color = green, % T1
			 line width = 2 pt,
			 domain = 1e-3:4e-1,
			 samples = 10,
			 ] 
			 {10^(-1.5184 - 0.1838 * log10(x))};
		\addlegendentry{Fit 1}
%		\node [fill = white, draw = green, anchor = east] % T1 
%        	at (axis description cs: 1,0.6) {$y = 10^{-1.5184} \, x^{- 0.1838}$};
        	
		\addplot [color = orange, % T2
			 line width = 2 pt,
			 domain = 1e-3:4e-1,
			 samples = 10,
			 ]
			 {10^(-1.5056 - 0.2516 * log10(x))};
		\addlegendentry{Fit 2}
%		\node [fill = white, draw = orange, anchor = east] % T2 
%        	at (axis description cs: 1,0.75) {$y = 10^{-1.5056} \, x^{- 0.2516}$};
        	
		\addplot [color = cyan, % T3
			 line width = 2 pt,
			 domain = 1e-3:4e-1,
			 samples = 10,
			 ]
			 {10^(-1.4085 - 0.2371 * log10(x))};
		\addlegendentry{Fit 3}
		\node [fill = white, draw = cyan, anchor = east] % T3 
        	at (axis description cs: 1,0.9) {$y = 10^{-1.4085} \, x^{- 0.2371}$};
        
		\node [fill = white, draw = black, anchor = south west] % T3 
        	at (axis description cs: 0,0) {$R^2 \in [0.3, 0.6]$ $P_\mathrm{val} < 0.0001$};
	\end{axis}
\end{tikzpicture}
	\caption[potenza della corrente rispetto alla proporzione di isole sull'alveo attivo, grafico bilogaritmico]{potenza della corrente rispetto alla proporzione di isole sull'alveo attivo con rette di regressione; ogni fit successivo considera solo i punti posti al disopra del fit precedente.}
	\label{graph:omega-area-percentuale}
\end{figure}
%
%\begin{figure}
%	\centering
%	\tikzsetnextfilename{omega_area_pura}
\begin{tikzpicture}
	\begin{axis}[
		width = \textwidth,
		height = .5\textwidth,
		%enlarge x limits = 0.01,
		%enlarge y limits = 0.01,
		ylabel = {$\Omega$ \si{[\newton\per\metre\tothe{4}]}},
		xlabel = {Isole rispetto all'alveo attivo \si{[\percent]}},
		xmode = log,
		ymode = log,
		grid = major,
		legend entries = {1,2,3,4,5,6,7,8,10,11,12,13,14,15,16,17,18,19,20,21,22,23},
		legend columns = 15,
		legend style = {
			anchor = south,
			at = {(0.5, 1.01)},
		},
		]
		\foreach \tratto in {1,2,...,23}
			{
			\addplot+[only marks]
				table [y = om_tr_\tratto, x = area_tr_\tratto]
				{graphics/data/omega_area_pura.txt};
			}
		
		\addplot [color = green, % T1
			 line width = 2 pt,
			 domain = 1e3:1e6,
			 samples = 10,
			 ] 
			 {10^(-0.3741 - 0.1820 * log10(x))};
		\node [fill = white, draw = cyan, anchor = east] % T1 
        	at (axis description cs: 1,0.9) {$10^{-0.3741} \, x^{- 0.1820}$};
        	
		\addplot [color = orange, % T2
			 line width = 2 pt,
			 domain = 1e3:1e6,
			 samples = 10,
			 ]
			 {10^(-0.0157 - 0.2366 * log10(x))};
		\node [fill = white, draw = orange, anchor = east] % T2 
        	at (axis description cs: 1,0.75) {$10^{-0.0157} \, x^{- 0.2366}$};
        	
		\addplot [color = cyan, % T3
			 line width = 2 pt,
			 domain = 1e3:1e6,
			 samples = 10,
			 ]
			 {10^(0.1130 - 0.2507 * log10(x))};
		\node [fill = white, draw = green, anchor = east] % T3 
        	at (axis description cs: 1,0.6) {$10^{0.1130} \, x^{- 0.2507}$};
        	
        \node [fill = white, draw = black, anchor = south west] % T3 
        	at (axis description cs: 0,0) {$R^2 \in [0.4, 0.8]$ $P_\mathrm{val} < 0.0001$};
	\end{axis}
\end{tikzpicture}
%	\caption[potenza della corrente rispetto all'areale delle isole]{potenza della corrente rispetto all'areale delle isole.}
%	\label{graph:omega-area-pura}
%\end{figure}
%
\\
Per ottenere una relazione che indichi quale sia il limite massimo di isole sono state fatte regressioni lineari in successione: da una prima regressione su tutti i punti si sono selezionati solo i punti al disopra della retta; si è eseguita una nuova regressione; con i punti posti superiormente alla seconda retta, si è ottenuta la retta finale di regressione.
Questa terza retta mostra un $R^2 \simeq 0.6$ ed un $P_\mathrm{value}$ ottenuto tramite il test statistico di Pearson minore di \num{0.0001}.
Data la discreta bontà di questa regressione, la si accetta come valida.


In un precedente lavoro è stato definito un unico valore limite di \emph{stream power} oltre il quale le isole non riescono più ad insediarsi a causa dell'intenso disturbo \squarecite{Gurnell:2006-omega}.
Tuttavia, questo valore è stato calcolato utilizzando l'equazione \eqref{eq:area-portata-mosetti}; per quanto già esposto si è preferito non utilizzare tale relazione.
\\
Nei grafici si vede che c'è un limite superiore di $\Omega$, circa \SIrange[range-phrase={-}]{0.12}{0.13}{\newton\per\metre\tothe{4}}, oltre il quale non è più presente vegetazione; al disotto di questo limite, la retta superiore di regressione mostra come la proporzione massima di isole che riesce a stabilirsi con una data potenza della corrente aumenti.
Si vede inoltre che l'alveo non ospita percentuali di vegetazione superiori al \SIrange[range-phrase={-}]{30}{35}{\percent}.

La relazione ottenuta presenta un limite implicito: $\Omega$ non tiene esattamente conto della connessione con la falda, dell'\emph{upwelling} e del \emph{downwelling}, che influenzano notevolmente la crescita delle piante.
Anzi, è stato mostrato che le isole complesse sono confinate nei tratti dove la crescita delle piante può essere sufficientemente rapida (\SIrange[range-phrase={-}]{1}{3}{\m} in \SI{10}{\anni}); questi tratti sono quelli relativamente più stretti, dove c'è disponibilità d'acqua durante i periodi di magra grazie alla falda non troppo profonda, come i tratti pochi chilometri a monte della stretta di Pinzano o quelli a monte della zona di cambiamento di morfologia fluviale \squarecite{Gurnell:2006-omega}.
Dall'altra parte, come gli autori osservano e come è verificato dai risultati appena mostrati, dove i tratti si restringono maggiormente la potenza della corrente è tanto grande che nemmeno l'incrementato tasso di crescita delle piante è sufficiente da permettere alle isole di insediarsi prima di essere portate via.
Gli autori suggeriscono quindi l'esistenza di un equilibrio tra processi idrologici, piante riparie e sviluppo delle isole, che si concretizza nei seguenti aspetti:
%
\begin{itemize}
	\item un valore massimo di $\Omega$, \SIrange[range-phrase={-}]{0.12}{0.13}{\newton\per\metre\tothe{4}}, oltre il quale non sono presenti isole;
	\item un range di $\Omega$ in cui è possibile lo stabilirsi di isole, ma nel quale c'è comunque un limite in cui il disturbo delle piene è predominante anche nelle zone relativamente più elevate dell'alveo;
	\item un limite massimo di isole, \numrange[range-phrase={-}]{0.30}{0.35}, che possono essere presenti anche con $\Omega$ molto bassi, in quanto l'insediamento di isole oltre queste limite avrebbe luogo su zone dell'alveo a quote relativamente minori, che sono le più disturbate dalle piene, mentre la zone e quote relativamente maggiori (creste delle barre, altre isole) sono tutte già vegetate.
\end{itemize}