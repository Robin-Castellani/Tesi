\section{Risultati}

Per ricercare delle relazioni si è effettuata più volte una regressione lineare dei punti nei grafici utilizzando due numeri come indicatori della bontà della relazione: l'$R^2$ e il $P_\mathbb{value}$ ottenuto dal test di Pearson.
Questi valori indicano una buona e forte relazione lineare quando $R^2 = 1$ e $P_\mathbb{value} < 0.05$, mentre se $R^2 \ll 1$ e/o $P_\mathbb{value} > 0.05$ allora i punti graficati non hanno una tendenza lineare.
\\
Al fine di migliorare questi valori si è cercato il miglior modo di rappresentare i dati con i seguenti modi:
%
\begin{itemize}
	\item scalando un asse con il logaritmo in base 10;
	\item dividendo la quantità rappresentata su un asse per un'altra quantità (ad esempio dividendo le isole erose per le isole presenti prima dell'erosione);
	\item accorpando i dati di più tratti adiacenti.
\end{itemize}