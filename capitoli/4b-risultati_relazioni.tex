\section{Risultati: (im)prevedibilità dell'erosione delle isole}
Per ricercare delle relazioni si è effettuata più volte una regressione lineare dei punti nei grafici utilizzando due numeri come indicatori della bontà della relazione: l'$R^2$ e il $P_\mathrm{value}$ ottenuto dal test di Pearson.
Questi valori indicano una buona e forte relazione lineare quando $R^2 = 1$ e $P_\mathrm{value} < 0.05$, mentre se $R^2 \ll 1$ e/o $P_\mathrm{value} > 0.05$ allora i punti graficati non hanno una tendenza lineare.
\\
Al fine di migliorare questi due valori si sono usate (non sempre contemporaneamente) queste metodologie:
%
\begin{itemize}
	\item scalare un asse con il logaritmo in base 10;
	\item dividere la quantità rappresentata su un asse per un'altra quantità (ad esempio dividere le isole erose per le isole presenti prima dell'erosione);
	\item accorpare i dati di più tratti adiacenti (ad esempio dei primi 4 tratti);
	\item escludere dall'analisi alcuni tratti con caratteristiche peculari;
	\item ignorare i punti che mostrano erosione anche per integrali nulli, in quanto gli areali di erosioni sono molto modesti e sono con tutta probabilità dovuti agli errori di correzione della georeferenziazione attuata nella fase di elaborazione del cambiamento
	\item considerare un numero minimo di punti per effettuare le regressioni e ritenerle valide (per due punti passa una ed una sola retta, e in tal caso una regressione lineare darebbe ottimi risultati, ma poco validi; con un maggior numero di punti l'affidabilità dei risultati migliora);
	\item limitare agli anni più recenti la suddivisioni in classi d'età, che è molto debolmente valida nei primi anni di osservazione (si veda la sezione~\ref{sec:eta}).
\end{itemize}

\subsection{Condizioni in cui si può formulare buone previsioni}
Per la classe d'età giovane utilizzando un tempo di ritorno di \SIrange[range-phrase = {-}, range-units = single]{3}{4}{\mesi} (livello soglia pari a \SI{2}{\m} sono stati ottenuti risultati molto positivi applicando ai dati le seguenti trasformazioni:
%
\begin{itemize}
	\item aggregando i dati in 5~gruppi di 4~tratti adiacenti a partire dal primo (1, \ldots, 4) e in 1~gruppo di 3~tratti (21, 22 e 23);
		come è stato considerato nel calcolo della pendenza, i tratti che compongono ogni gruppo presentano caratteristiche comuni
		(rispettivamente per i 6~gruppi: confinamento da parte delle montagne, allargamento con \emph{downwelling}, pronunciato \emph{upwelling} e restringimento, riallargamento con \emph{downwelling} nella zona valliva, scorrimento nella parte planiziale, restringimento con cambiamento di morfologia)
	\item normalizzando gli areali delle isole erose con gli areali delle isole presenti nella prima immagine del cambiamento, ottenendo quindi la percentuale di isole erose;
	\item escludendo il tratto~9 poiché vi è l'isola di Cornino, che si fonda su roccia, e il tratto~23, in quanto presenta una morfologia nettamente diversa rispetto agli altri tratti;
	\item imponendo di avere un minimo di~5 punti per effettuare le regressioni
	\item graficando solo i dati di erosione posteriori all'immagine del 2007-09-21 incluso;
	\item mantenendo una scala lineare.
\end{itemize}
%
I risultati con le regressioni lineari sono riportati nei grafici in \cref{graph:giov-iote-4tr-buono} e in \cref{tab:giov-iote-4tr-buono}.
%
\begin{figure}
	\centering
	\tikzsetnextfilename{giov_iote_4tr_buono}
\begin{tikzpicture}
	\begin{groupplot}[
		group style = {
			group size = 3 by 2,
			y descriptions at = edge left,
			x descriptions at = edge bottom,
			horizontal sep = 0.1cm,
			vertical sep = 0.1cm,
		},
		width = 0.4\textwidth,
		height = 0.4\textwidth,
		ymin = 0,
		ymax = 0.65,
		enlarge y limits = 0.05,
		enlarge x limits = 0.05,
		ylabel = {Erosione / isole},
		xlabel = {Int livelli \si{[\m\giorno]}},
		grid = major,		
		]
		
	\nextgroupplot[] % gruppo 1
			\addplot[only marks]
	        	table [x = integrale_piene, y = gr_1] {graphics/data/giov_iote_4tr_buono.txt};
	        
	        \addplot[
	        	line width = 2pt,
	        	domain = 0:2,
	        	samples = 3,
				green,
	        	]
	        	{0.1852 * x + 0.1032};
			
			\node at (axis description cs: 0,1) [draw = black, fill = white, anchor = north west, align = left] {1$\div$4};
	
	\nextgroupplot[] % gruppo 2
			\addplot[only marks]
	        	table [x = integrale_piene, y = gr_2] {graphics/data/giov_iote_4tr_buono.txt};
	        
	        \addplot[
	        	line width = 2pt,
	        	domain = 0:2,
	        	samples = 3,
				green,
	        	]
	        	{0.1700 * x + 0.0669};
			
			\node at (axis description cs: 0,1) [draw = black, fill = white, anchor = north west, align = left] {5$\div$8};
	
	\nextgroupplot[] % gruppo 3
			\addplot[only marks]
	        	table [x = integrale_piene, y = gr_3] {graphics/data/giov_iote_4tr_buono.txt};
	        
	        \addplot[
	        	line width = 2pt,
	        	domain = 0:2,
	        	samples = 3,
				green,
	        	]
	        	{0.1392 * x + 0.0775};
			
			\node at (axis description cs: 0,1) [draw = black, fill = white, anchor = north west, align = left] {10$\div$12};
	
	\nextgroupplot[] % gruppo 4
			\addplot[only marks]
	        	table [x = integrale_piene, y = gr_4] {graphics/data/giov_iote_4tr_buono.txt};
	        
	        \addplot[
	        	line width = 2pt,
	        	domain = 0:2,
	        	samples = 3,
				green,
	        	]
	        	{0.2126 * x + 0.1189};
			
			\node at (axis description cs: 0,1) [draw = black, fill = white, anchor = north west, align = left] {13$\div$16};
	
	\nextgroupplot[] % gruppo 5
			\addplot[only marks]
	        	table [x = integrale_piene, y = gr_5] {graphics/data/giov_iote_4tr_buono.txt};
	        
	        \addplot[
	        	line width = 2pt,
	        	domain = 0:2,
	        	samples = 3,
				green,
	        	]
	        	{0.2553 * x + 0.1076};
			
			\node at (axis description cs: 0,1) [draw = black, fill = white, anchor = north west, align = left] {17$\div$20};
	
	\end{groupplot}
\end{tikzpicture}
	\caption[dati e regressioni lineari per la vegetazione giovane]{dati e regressioni lineari per la vegetazione giovane con l'integrale calcolato sopra la soglia di \SI{2}{\m} e i tratti raggruppati quattro a quattro; le equazioni e i valori degli $R^2$ e $P_\mathrm{value}$ sono riportati nella \cref{tab:giov-iote-4tr-buono}.}
	\label{graph:giov-iote-4tr-buono}
\end{figure}
%
\begin{table}
	\centering
	\begin{tabular}{c c S[table-format = 0.2] S[table-format = 0.3, table-comparator = true]}
		\toprule
		\textbf{Tratti}			&	\textbf{Equazione}		&	\multicolumn{1}{c}{$\mathbf{R^2}$}	&	\multicolumn{1}{c}{$	\mathbf{P_\mathrm{value}}$}	\\
		\textbf{di validità}	&	\textbf{della retta}	&	&	\\
		\midrule
		1$\div$4	&	$y = 0.1852 \, x + 0.1032$	&	0.49	&	<0.02	\\
		5$\div$8	&	$y = 0.1700 \, x + 0.0669$	&	0.46	&	<0.006	\\
		10$\div$12	&	$y = 0.1392 \, x + 0.0775$	&	0.48	&	<0.02	\\
		13$\div$16	&	$y = 0.2126 \, x + 0.1189$	&	0.40	&	<0.009	\\
		17$\div$20	&	$y = 0.2553 \, x + 0.1076$	&	0.59	&	<0.004	\\
		\bottomrule
	\end{tabular}
	\caption[equazioni, $R^2$ e $P_\mathrm{value}$ delle regressioni per la vegetazione giovane]{equazioni, $R^2$ e $P_\mathrm{value}$ delle regressioni per la vegetazione giovane, mostrate nel grafico in \cref{graph:giov-iote-4tr-buono}.}
	\label{tab:giov-iote-4tr-buono}
\end{table}
%
\\
Le regressioni vengono considerate valide grazie all'$R^2$ relativamente elevato, il $P_\mathrm{value}$ generalmente piccolo, l'elevato numero di punti (13 in media) e la validità spazialmente estesa di queste relazioni.

Dato che si sono trovate relazioni con differenze spaziali non elevatissime, è naturale ricercare anche una relazione che valga graficando tutti i punti assieme.
Il grafico in \cref{fig:giov-iote-4tr-buono-accorpato}
mostra la regressione e i numeri $R^2$ e $P_\mathrm{value}$: sembra che esista una chiara tendenza lineare tra la percentuale di isole erose e l'integrale dei livelli sopra la soglia di \SI{2}{\m}.
%
\begin{figure}
	\centering
	\tikzsetnextfilename{giov_iote_4tr_buono_accorpato}
\begin{tikzpicture}
	\begin{axis}[
		width = \textwidth,
		height = 0.5\textwidth,
%		ymin = 0,
%		ymax = 0.65,
%		enlarge y limits = 0.05,
%		enlarge x limits = 0.05,
		ylabel = {Erosione / Isole},
		xlabel = {Integrale dei livelli sopra soglia \si{[\m\giorno]}},
		grid = major,
		title = {TR \SIrange[range-phrase={-}, range-units = single]{4}{5}{\mesi}},		
		]
		
		\pgfplotsforeachungrouped \gr in {1,2,...,5} {
		\addplot[only marks]
			table [x = integrale_piene_tr_0.5, y = gr_\gr_tr_0.5] {graphics/data/giov_iote_4tr_buono.txt};
		}
	
		\addplot[
			line width = 2pt,
			domain = 0:2,
			samples = 3,
			green,
			]
			{0.1859 * x + 0.0987};
			
		\node at (axis description cs: 0,1) [draw = black, fill = white, anchor = north west, align = left, name = range] {1$\div$22};
		\node at (range.south west) [draw = black, fill = white, anchor = north west, align = left] {$y = 0.1859 \, x + 0.0987$};
		
		\node at (axis description cs: 1,0) [draw = black, fill = white, anchor = south east, align = left] {$R^2 = 0.42 \quad P_\mathrm{value} < 0.00001$};
	
	\end{axis}
\end{tikzpicture}
	\caption[regressione per la vegetazione giovane considerando tutti i gruppi]{dati, equazione, $R^2$ e $P_\mathrm{value}$ della regressione per la vegetazione giovane considerando tutti i gruppi.}
	\label{fig:giov-iote-4tr-buono-accorpato}
\end{figure}


