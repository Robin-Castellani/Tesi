\section{Risultati e discussione: relazioni tra piene ed isole erose}
\subsection{Un confronto con dati da letteratura}
In letteratura è presente un lavoro che ha già ottenuto delle relazioni tra il regime delle piene e la proporzione di isole erose per tre tempi di ritorno (TR~\SI{<1}{\anno}, TR~\SI{1.2}{\anni}, TR~\SI{2.5}{\anni}) \squarecite{Surian:2015}:
gli autori hanno confrontato~8 ortofoto (1986, 1993, 1997, 1999, 2003, 2005, 2009, 2011) che comprendono i tratti 6$\div$12;
hanno utilizzato una scala di deflusso sviluppata su una sezione presso Venzone, dove è presente una stazione idrometrica, e grazie a questa hanno ottenuto le portate cumulate annualmente tra ogni coppia di immagini successive sopra una portata soglia definita secondo un tempo di ritorno;
infine, hanno eseguito delle regressioni utilizzando leggi di potenza.
\\
I grafici in \cref{graph:relazioni-piene-erosione-vs-surian} mostrano le relazioni ottenute da \squarecite{Surian:2015} (grafici nella prima riga), confrontandole con le percentuali di isole erose rispetto a quelle presenti in funzione dell'integrale dei livelli sopra soglia per gli stessi tempi di ritorno di cui sopra; sono stati inizialmente considerati i tratti dal~6 al~12 (tranne il~9) e leggi esponenziali per le regressioni (grafici nella seconda riga); in seguito sono stati utilizzati tutti i tratti (tranne il~9 e il~23) e sono state applicate regressioni lineari.
%
\begin{figure}
	\centering
	\tikzsetnextfilename{relazioni_piene_erosione_vs_surian}
\begin{tikzpicture}
	\begin{groupplot}[
		group style = {
			group size = 3 by 3,
			y descriptions at = edge left,
			horizontal sep = 0.1cm,
			vertical sep = 1.9cm,
		},
		width = 0.4\textwidth,
		height = 0.4\textwidth,
		grid = major,
		title style = {yshift = -0.25cm},
		every tick label/.append style  = {
			/pgf/number format/.cd,
			fixed,
			fixed zerofill,
			precision = 1,
			/tikz/.cd,
		},
	]
	\nextgroupplot[ % TR < 1 anno
		ylabel = {Erosione annuale \si{[\percent\per\anno]}},
		title = {TR \SI{<1}{\anno}},
		xlabel = {},
		ymax = 16,
		]
		\addplot [
			only marks,
			blue,			
			]
			table[x = {RI<1yr_x}, y = RI<1yr_y] {graphics/data/relazioni_piene_erosione_surian_2014.txt};
			
		\addplot [
			no markers,
			red,
			smooth,
			ultra thick,		
			]
			table[x = {Eq_RI<1yr_x}, y = Eq_RI<1yr_y] {graphics/data/relazioni_piene_erosione_surian_2014.txt};
			
		\node at (axis description cs: 1,0) [anchor = south east, draw = black, fill = white] {$R^2 = 0.28$};
		
	%--------------------------------------------------	
		
	\nextgroupplot[ % TR = 1.2 anni
		title = {TR \SI{1.2}{\anni}},
		xlabel = {Portata cumulata annuale \SI[retain-unity-mantissa = false]{1e9}{[\m\tothe{3}\per\anno]}},
		ymax = 16,
		]
		\addplot [
			only marks,
			blue,			
			]
			table[x = {RI=1.2yr_x}, y = {RI=1.2yr_y}] {graphics/data/relazioni_piene_erosione_surian_2014.txt};
			
		\addplot [
			no markers,
			red,
			smooth,
			ultra thick,		
			]
			table[x = {Eq_RI=1.2yr_x}, y = {Eq_RI=1.2yr_y}] {graphics/data/relazioni_piene_erosione_surian_2014.txt};
			
		\node at (axis description cs: 1,0) [anchor = south east, draw = black, fill = white] {$R^2 = 0.37$};
		
	%--------------------------------------------------	
		
	\nextgroupplot[ % TR = 2.5 anni
		title = {TR \SI{2.5}{\anni}},
		xlabel = {},
		ymax = 16,
		]
		\addplot [
			only marks,
			blue,			
			]
			table[x = {RI=2.5yr_x}, y = {RI=2.5yr_y}] {graphics/data/relazioni_piene_erosione_surian_2014.txt};
			
		\addplot [
			no markers,
			red,
			smooth,
			ultra thick,		
			]
			table[x = {Eq_RI=2.5yr_x}, y = {Eq_RI=2.5yr_y}] {graphics/data/relazioni_piene_erosione_surian_2014.txt};
			
		\node at (axis description cs: 1,0) [anchor = south east, draw = black, fill = white] {$R^2 = 0.79$};
		
	%€€€€€€€€ SOLO TRATTI 6-, RELAZIONI ESPONENZIALI €€€€€€€€€
		
	\nextgroupplot[ % mia TR = 0.5 anni (liv = 2m)
%		xlabel = {Int livelli \si{[\m\giorno]}},
		ylabel = {Erosione/Isole \si{[\percent]}},
		grid = major,
		title = {TR \SIrange[range-phrase={-}, range-units = single]{4}{5}{\mesi}},		
		ymax = 100,	
		]
		
		\pgfplotsforeachungrouped \tr in {6,7,...,12} {
		\addplot[
			only marks,
			mark size = 1pt,
			blue,
			]
			table [x = integrale_piene, y expr = \thisrow{tr_\tr} * 100] {graphics/data/iote_tr_0.5_vs_surian.txt};
		};
	
		\addplot[
			ultra thick,
			domain = 0:2,
			samples = 20,
			red,
			]
			{10^(0.3467 * x - 1.1774) * 100};
			
		%\node at (axis description cs: 0,1) [draw = black, fill = white, anchor = north west, align = left, name = range] {1$\div$22};
		
		\node at (axis description cs: 1,0) [draw = black, fill = white, anchor = south east, align = left] {$R^2 = 0.24$};
		
	%--------------------------------------------------	
		
	\nextgroupplot[ % mia TR = 1.2 anni (liv = 2.5m)
		xlabel = {Integrale dei livelli sopra soglia \si{[\m\giorno]}},
		grid = major,
		title = {TR \SI{1.2}{\anni}},
		ymax = 100,	
		xtick distance = 0.25,
		]
		
		\pgfplotsforeachungrouped \tr in {6,7,...,12} {
		\addplot[
			only marks,
			mark size = 1pt,
			blue,
			]
			table [x = integrale_piene, y expr = \thisrow{tr_\tr} * 100] {graphics/data/iote_tr_1.2_vs_surian.txt};
		};
	
		\addplot[
			ultra thick,
			domain = 0:0.75,
			samples = 20,
			red,
			]
			{10^(0.6533 * x - 0.9201) * 100};
		
		\node at (axis description cs: 1,0) [draw = black, fill = white, anchor = south east, align = left] {$R^2 = 0.12$};
		
	%--------------------------------------------------	
		
	\nextgroupplot[ % mia TR = 2.5 anni (liv = 2.9m)
		grid = major,
		title = {TR \SI{2.5}{\anni}},
		ymax = 100,	
		]
		
		\pgfplotsforeachungrouped \tr in {6,7,...,12} {
		\addplot[
			only marks,
			mark size = 1pt,
			blue,
			]
			table [x = integrale_piene, y expr = \thisrow{tr_\tr} * 100] {graphics/data/iote_tr_2.5_vs_surian.txt};
		};
	
		\addplot[
			ultra thick,
			domain = 0:0.27,
			samples = 20,
			red,
			]
			{10^(0.9813 * x -0.7358) * 100};
		
		\node at (axis description cs: 1,0) [draw = black, fill = white, anchor = south east, align = left] {$R^2 = 0.07$};
		
	%€€€€€€€€ TUTTI TRATTI, RELAZIONI LINEARI €€€€€€€€€
		
	\nextgroupplot[ % mia TR = 0.5 anni (liv = 2m)
%		xlabel = {Int livelli \si{[\m\giorno]}},
		ylabel = {Erosione/Isole \si{[\percent]}},
		grid = major,
		title = {TR \SIrange[range-phrase={-}, range-units = single]{4}{5}{\mesi}},		
		ymax = 100,	
		]
		
		\pgfplotsforeachungrouped \tr in {1,2,...,23} {
		\addplot[
			only marks,
			mark size = 1pt,
			blue,
			]
			table [x = integrale_piene, y expr = \thisrow{tr_\tr} * 100] {graphics/data/iote_tr_0.5_vs_surian.txt};
		};
	
		\addplot[
			ultra thick,
			domain = 0:2,
			samples = 3,
			red,
			]
			{(0.1343 * x + 0.0989) * 100};
			
		%\node at (axis description cs: 0,1) [draw = black, fill = white, anchor = north west, align = left, name = range] {1$\div$22};
		
		\node at (axis description cs: 1,0) [draw = black, fill = white, anchor = south east, align = left] {$R^2 = 0.26$};
		
	%--------------------------------------------------	
		
	\nextgroupplot[ % mia TR = 1.2 anni (liv = 2.5m)
		xlabel = {Integrale dei livelli sopra soglia \si{[\m\giorno]}},
		grid = major,
		title = {TR \SI{1.2}{\anni}},
		ymax = 100,	
		xtick distance = 0.25,
		]
		
		\pgfplotsforeachungrouped \tr in {1,2,...,23} {
		\addplot[
			only marks,
			mark size = 1pt,
			blue,
			]
			table [x = integrale_piene, y expr = \thisrow{tr_\tr} * 100] {graphics/data/iote_tr_1.2_vs_surian.txt};
		};
	
		\addplot[
			ultra thick,
			domain = 0:0.75,
			samples = 3,
			red,
			]
			{(0.2867 * x + 0.1724) * 100};
		
		\node at (axis description cs: 1,0) [draw = black, fill = white, anchor = south east, align = left] {$R^2 = 0.10$};
		
	%--------------------------------------------------	
		
	\nextgroupplot[ % mia TR = 2.5 anni (liv = 2.9m)
		grid = major,
		title = {TR \SI{2.5}{\anni}},
		ymax = 100,	
		]
		
		\pgfplotsforeachungrouped \tr in {1,2,...,23} {
		\addplot[
			only marks,
			mark size = 1pt,
			blue,
			]
			table [x = integrale_piene, y expr = \thisrow{tr_\tr} * 100] {graphics/data/iote_tr_2.5_vs_surian.txt};
		};
	
		\addplot[
			ultra thick,
			domain = 0:0.27,
			samples = 3,
			red,
			]
			{(0.7014 * x + 0.2004) * 100};
		
		\node at (axis description cs: 1,0) [draw = black, fill = white, anchor = south east, align = left] {$R^2 = 0.11$};
		
		
	\end{groupplot}
\end{tikzpicture}
	\vspace*{-0.7cm}
	\caption[relazioni tra proporzione di isole erose e livello cumulato sopra soglia confrontato con dati da \squarecite{Surian:2015}]{relazioni tra proporzione di isole erose e livello cumulato sopra soglia per tre diversi tempi di ritorno confrontato con dati da \squarecite{Surian:2015} (in alto).
	Questi hanno usato~8 mappe (dal~1986 al~2011) che si estendono dal tratto~6 al tratto~12 compresi; per ottenere le portate cumulate hanno usato una scala di deflusso applicata ai dati idrometrici del sensore presso Venzone; le regressioni (curve in rosso) sono leggi di potenza.
	Nella seconda riga sono mostrate le proporzioni di isole erose in funzione dell'integrale dei livelli sopra soglia della presente tesi considerando i tratti dal~6 al~12 (escluso il~9, dove è presente l'isola di Cornino) eseguendo regressioni con leggi esponenziali (curve rosse); l'ultima riga mostra tutti i tratti (escluso il~9) e regressioni lineari (rette rosse).}
	\label{graph:relazioni-piene-erosione-vs-surian}
\end{figure}
%
\\
Questo confronto serve a mostrare come i dati della presente tesi siano in numero molto maggiore rispetto a quelli da letteratura e questo possa portare a relazioni più deboli;
come relazioni lineari e a legge di potenza diano in questo caso risultati simili;
come il numero di confronti si riduce in maniera considerevole quando si utilizzano livelli soglia riferiti a tempi di ritorno via via maggiori;
come ci siano percentuali di erosione molto più alte rispetto ai dati di \squarecite{Surian:2015};
come siano presenti delle forti differenze spaziali, evidenziate anche nelle discussioni sui grafici che mostravano i tassi di accrescimento e di erosione (sezione~\ref{sec:camb-ris}), che portano ad avere alte percentuali di isole erose nei tratti dove sono presenti poche isole.
\\
Si ritiene che la fonte principale delle differenze presentate risieda nel metodo con cui sono state digitalizzate le isole, come è stato discusso nella sezione~\ref{sec:perc-isole-in-alveo} e nel maggior numero di immagini nel presente lavoro, che permette di osservare cambiamenti anche importanti ad una scala temporale minore (difatti se si avessero avuto poche immagini, gli integrali dei livelli avrebbero avuto significato a scala di periodi di anni; avendo più di un'immagine all'anno, è possibile osservare gli effetti quasi del singolo evento di piena).




%\FloatBarrier
\subsection{Suddividere la vegetazione in classi d'età}
Si considerano i grafici precedenti con le regressioni lineari, in cui si dividono i tassi di erosione secondo le tre classi d'età della vegetazione (sezione~\ref{sec:eta}); si considerano tutti i tratti tranne il~9 e il~23.
I grafici in \cref{graph:tr-05-iote-classi-eta-lin}, in \cref{graph:tr-12-iote-classi-eta-lin} e in \cref{graph:tr-25-iote-classi-eta-lin} mostrano i dati e i risultati delle regressioni lineari.
\\
Si vede come la suddivisione in classi d'età mostra come queste abbiano comportamenti differenti, e come alcune classi mostrino buone relazioni.
I risultati possono essere molto positivi procedendo nella ricerca delle relazioni con il nuovo approccio di considerare le isole divise in fasce d'età.
%
\begin{figure}
	\centering
	\tikzsetnextfilename{tr_0.5_iote_classi_eta_lin}
\begin{tikzpicture}	
	\begin{groupplot}[
		group style = {
			group size = 1 by 3,
			y descriptions at = edge left,
			x descriptions at = edge bottom,
			horizontal sep = 0.1cm,
			vertical sep = 0.3cm,
		},
		width = 0.95\textwidth,
		height = 0.5\textwidth,
		ymin = 0,
		ymax = 1,
		enlarge y limits = 0.05,
		ylabel = {Erosione / Isole},
		xlabel = {Integrale dei livelli \si{[\m\giorno]}},
		grid = major,	
		]
		
	\nextgroupplot[
		title = {TR \SIrange[range-phrase = {-}, range-units = single]{4}{5}{\mesi}},		
		] % giovane
		\pgfplotsforeachungrouped \tr in {1,2,...,23} {
		\addplot[only marks]
			table [x = integrale_piene, y = giov_tr_\tr] {graphics/data/tr_0.5_iote_classi_eta_lin.txt};
		}
			
	        \addplot[
	        	line width = 2pt,
	        	domain = 0:2,
	        	samples = 3,
				green,
	        	]
	        	{0.1979 * x + 0.1009};
			
			\node at (axis description cs: 0,1) [draw = black, fill = white, anchor = north west] {Giovane};
			
			\node at (axis description cs: 1,1) [draw = black, fill = white, anchor = north east] {$R^2 = 0.40$ $P_\mathrm{value} < 0.00001$};
		
	\nextgroupplot[] % intermedia
		\pgfplotsforeachungrouped \tr in {1,2,...,23} {
		\addplot[only marks]
			table [x = integrale_piene, y = int_tr_\tr] {graphics/data/tr_0.5_iote_classi_eta_lin.txt};
		}
			
	        \addplot[
	        	line width = 2pt,
	        	domain = 0:2,
	        	samples = 3,
				green,
	        	]
	        	{0.0667 * x + 0.1022};
			
			\node at (axis description cs: 0,1) [draw = black, fill = white, anchor = north west] {Intermedia};
			
			\node at (axis description cs: 1,1) [draw = black, fill = white, anchor = north east] {$R^2 = 0.07$ $P_\mathrm{value} < 0.00002$};
		
	\nextgroupplot[] % matura
		\pgfplotsforeachungrouped \tr in {1,2,...,23} {
		\addplot[only marks]
			table [x = integrale_piene, y = mat_tr_\tr] {graphics/data/tr_0.5_iote_classi_eta_lin.txt};
		}
			
	        \addplot[
	        	line width = 2pt,
	        	domain = 0:2,
	        	samples = 3,
				green,
	        	]
	        	{0.0521 * x + 0.0599};
			
			\node at (axis description cs: 0,1) [draw = black, fill = white, anchor = north west] {Matura};
			
			\node at (axis description cs: 1,1) [draw = black, fill = white, anchor = north east] {$R^2 = 0.06$ $P_\mathrm{value} < 0.0003$};
	
	\end{groupplot}
\end{tikzpicture}

	\caption[suddivisione in classi d'età ricercando relazioni tra erosione e piene con livello maggiore di \SI{2}{\m}]{esempio di suddivisione della vegetazione in classi d'età nella ricerca di relazioni lineari tra i tassi di erosione e l'integrale dei livelli sopra la soglia di \SI{2}{\m} (TR~\SIrange[range-phrase = {-}, range-units = single]{4}{5}{\mesi}); le rette in verde rappresentano le regressioni lineari.}
	\label{graph:tr-05-iote-classi-eta-lin}
\end{figure}
%
%
\begin{figure}
	\centering
	\tikzsetnextfilename{tr_1.2_iote_classi_eta_lin}
\begin{tikzpicture}	
	\begin{groupplot}[
		group style = {
			group size = 1 by 3,
			y descriptions at = edge left,
			x descriptions at = edge bottom,
			horizontal sep = 0.1cm,
			vertical sep = 0.3cm,
		},
		width = \textwidth,
		height = 0.5\textwidth,
		ymin = 0,
		ymax = 1,
		enlarge y limits = 0.05,
		ylabel = {Erosione / Isole},
		xlabel = {Integrale dei livelli \si{[\m\giorno]}},
		grid = major,	
		]
		
	\nextgroupplot[
		title = {TR \SI{1.2}{\anni}},		
		] % giovane
		\pgfplotsforeachungrouped \tr in {1,2,...,23} {
		\addplot[only marks]
			table [x = integrale_piene, y = giov_tr_\tr] {graphics/data/tr_1.2_iote_classi_eta_lin.txt};
		}
			
	        \addplot[
	        	line width = 2pt,
	        	domain = 0:0.8,
	        	samples = 3,
				green,
	        	]
	        	{0.345 * x + 0.2385};
			
			\node at (axis description cs: 0,1) [draw = black, fill = white, anchor = north west] {Giovane};
			
			\node at (axis description cs: 1,1) [draw = black, fill = white, anchor = north east] {$R^2 = 0.16$ $P_\mathrm{value} < 0.00001$};
		
	\nextgroupplot[] % intermedia
		\pgfplotsforeachungrouped \tr in {1,2,...,23} {
		\addplot[only marks]
			table [x = integrale_piene, y = int_tr_\tr] {graphics/data/tr_1.2_iote_classi_eta_lin.txt};
		}
			
	        \addplot[
	        	line width = 2pt,
	        	domain = 0:0.8,
	        	samples = 3,
				green,
	        	]
	        	{0.2167 * x + 0.1074};
			
			\node at (axis description cs: 0,1) [draw = black, fill = white, anchor = north west] {Intermedia};
			
			\node at (axis description cs: 1,1) [draw = black, fill = white, anchor = north east] {$R^2 = 0.08$ $P_\mathrm{value} < 0.0008$};
		
	\nextgroupplot[] % matura
		\pgfplotsforeachungrouped \tr in {1,2,...,23} {
		\addplot[only marks]
			table [x = integrale_piene, y = mat_tr_\tr] {graphics/data/tr_1.2_iote_classi_eta_lin.txt};
		}
			
	        \addplot[
	        	line width = 2pt,
	        	domain = 0:0.8,
	        	samples = 3,
				green,
	        	]
	        	{0.1706 * x + 0.0638};
			
			\node at (axis description cs: 0,1) [draw = black, fill = white, anchor = north west] {Matura};
			
			\node at (axis description cs: 1,1) [draw = black, fill = white, anchor = north east] {$R^2 = 0.06$ $P_\mathrm{value} < 0.004$};
	
	\end{groupplot}
\end{tikzpicture}

	\caption[suddivisione in classi d'età ricercando relazioni tra erosione e piene con con livello maggiore di \SI{2.5}{\m}]{esempio di suddivisione della vegetazione in classi d'età nella ricerca di relazioni lineari tra i tassi di erosione e l'integrale dei livelli sopra la soglia di \SI{1.2}{\anni}); le rette in verde rappresentano le regressioni lineari.}
	\label{graph:tr-12-iote-classi-eta-lin}
\end{figure}
%
%
\begin{figure}
	\centering
	\tikzsetnextfilename{tr_2.5_iote_classi_eta_lin}
\begin{tikzpicture}	
	\begin{groupplot}[
		group style = {
			group size = 1 by 3,
			y descriptions at = edge left,
			x descriptions at = edge bottom,
			horizontal sep = 0.1cm,
			vertical sep = 0.3cm,
		},
		width = \textwidth,
		height = 0.5\textwidth,
		ymin = 0,
		ymax = 1,
		enlarge y limits = 0.05,
		ylabel = {Erosione / Isole},
		xlabel = {Integrale dei livelli \si{[\m\giorno]}},
		x tick label style = {
			/pgf/number format/.cd,
			fixed,
			fixed zerofill,
			precision = 2,
			/tikz/.cd,
		},
		y tick label style = {
			/pgf/number format/.cd,
			fixed,
			fixed zerofill,
			precision = 1,
			/tikz/.cd,
		},
		grid = major,	
		]
		
	\nextgroupplot[
		title = {TR \SI{2.5}{\anni}},		
		] % giovane
		\pgfplotsforeachungrouped \tr in {1,2,...,23} {
		\addplot[only marks]
			table [x = integrale_piene, y = giov_tr_\tr] {graphics/data/tr_2.5_iote_classi_eta_lin.txt};
		}
			
	        \addplot[
	        	line width = 2pt,
	        	domain = 0:0.3,
	        	samples = 3,
				green,
	        	]
	        	{0.9307 * x + 0.2511};
			
			\node at (axis description cs: 0,1) [draw = black, fill = white, anchor = north west] {Giovane};
			
			\node at (axis description cs: 1,1) [draw = black, fill = white, anchor = north east] {$R^2 = 0.26$ $P_\mathrm{value} < 0.00001$};
		
	\nextgroupplot[] % intermedia
		\pgfplotsforeachungrouped \tr in {1,2,...,23} {
		\addplot[only marks]
			table [x = integrale_piene, y = int_tr_\tr] {graphics/data/tr_2.5_iote_classi_eta_lin.txt};
		}
			
	        \addplot[
	        	line width = 2pt,
	        	domain = 0:0.3,
	        	samples = 3,
				green,
	        	]
	        	{0.597 * x + 0.1064};
			
			\node at (axis description cs: 0,1) [draw = black, fill = white, anchor = north west] {Intermedia};
			
			\node at (axis description cs: 1,1) [draw = black, fill = white, anchor = north east] {$R^2 = 0.17$ $P_\mathrm{value} < 0.0002$};
		
	\nextgroupplot[] % matura
		\pgfplotsforeachungrouped \tr in {1,2,...,23} {
		\addplot[only marks]
			table [x = integrale_piene, y = mat_tr_\tr] {graphics/data/tr_2.5_iote_classi_eta_lin.txt};
		}
			
	        \addplot[
	        	line width = 2pt,
	        	domain = 0:0.3,
	        	samples = 3,
				green,
	        	]
	        	{0.4823 * x + 0.0644};
			
			\node at (axis description cs: 0,1) [draw = black, fill = white, anchor = north west] {Matura};
			
			\node at (axis description cs: 1,1) [draw = black, fill = white, anchor = north east] {$R^2 = 0.17$ $P_\mathrm{value} < 0.0004$};
	
	\end{groupplot}
\end{tikzpicture}

	\caption[suddivisione in classi d'età ricercando relazioni tra erosione e piene con con livello maggiore di \SI{2.9}{\m}]{esempio di suddivisione della vegetazione in classi d'età nella ricerca di relazioni lineari tra i tassi di erosione e l'integrale dei livelli sopra la soglia di \SI{2.5}{\anni}); le rette in verde rappresentano le regressioni lineari.}
	\label{graph:tr-25-iote-classi-eta-lin}
\end{figure}
%




\FloatBarrier
\subsection{Esplorare altre metodologie}
\label{sec:metodologie-piene-erosione}
Al fine di filtrare le differenze spaziali più marcate, si sono aggregati i dati dei tratti in~6 gruppi di~4 tratti adiacenti (a meno dei tratti esclusi), come è stato fatto nel calcolo della pendenza e nelle considerazioni sui tassi di erosione e crescita.
Si è mantenuta la rappresentazione dell'erosione come tasso rispetto alle isole della stessa fascia d'età presenti.
Nella \cref{tab:iote-4tr-lin-ntr-r2-pval} sono riportate, per quattro tempi di ritorno, il numero, la mediana dell'$R^2$ e del $P_\mathrm{value}$ dei tratti in cui le relazioni sono valide.
%
\begin{table}
	\centering
	\begin{tabular}{c c *{3}{S[table-format = 2.3, table-comparator = true]}}
		\toprule
		\multicolumn{5}{p{0.7\textwidth}}{Relazioni lineari in gruppi di~4 tratti tra proporzione di isole erose rispetto alle isole presenti della stessa età in funzione degli integrali dei livelli sopra soglia}	\\
		\midrule
			&	&	{\textbf{Giovane}}	&	{\textbf{Intermedia}}	&	{\textbf{Matura}}	\\
		\midrule
		\multirow{3}*{\begin{sideways}\SIrange[range-phrase = {-}, range-units = single]{2}{3}{\mesi}\end{sideways}}	&	\# tratti	&	12	&	4	&	6	\\
			&	Mediana $R^2$	&	0.51	&	0.41	&	0.69	\\
			&	Mediana $P_\mathrm{value}$	&	0.005	&	0.019	&	<0.001	\\
		\midrule
		\multirow{3}*{\begin{sideways}\SIrange[range-phrase = {-}, range-units = single]{4}{5}{\mesi}\end{sideways}}	&	\# tratti	&	19	&	{-}	&	2	\\
			&	Mediana $R^2$	&	0.51	&	{-}	&	0.37	\\
			&	Mediana $P_\mathrm{value}$	&	0.005	&	{-}	&	0.036	\\
		\midrule
		\multirow{3}*{\begin{sideways}\SI{1}{\anno}\end{sideways}}	&	\# tratti	&	4	&	7	&	8	\\
			&	Mediana $R^2$	&	0.48	&	0.65	&	0.54	\\
			&	Mediana $P_\mathrm{value}$	&	0.026	&	0.015	&	0.027	\\
		\midrule
		\multirow{3}*{\begin{sideways}\SI{2}{\anni}\end{sideways}}	&	\# tratti	&	4	&	4	&	{-}	\\
			&	Mediana $R^2$	&	0.57	&	0.46	&	{-}	\\
			&	Mediana $P_\mathrm{value}$	&	0.012	&	0.032	&	{-}	\\
		\bottomrule
	\end{tabular}
	\caption[numero di tratti nei gruppi di~4 tratti con relazioni significative dividendo la vegetazione in classi d'età]{numero di tratti per cui valgono relazioni significative tra tassi di erosione della vegetazione suddivisa in fasce d'età e integrale dei livelli sopra soglia secondo quattro tempi di ritorno; sono riportate le mediane degli $R^2$ e $P_\mathrm{value}$ in questi tratti; “-” indica assenza di relazioni valide; i tratti sono stati uniti 4 a~4.}
	\label{tab:iote-4tr-lin-ntr-r2-pval}
\end{table}
%
\\
Per la classe giovane ci sono buone relzioni su molti tratti per TR~\SI{< 1}{\anno};
ci sono discreti risultati anche per le classi di vegetazione intermedia e matura per TR di \SIrange[range-phrase = {-}, range-units = single]{2}{3}{\mesi} e di \SI{1}{\anno};
per il TR \SI{2}{\anni} le relazioni sono valide su pochi tratti e quindi non vengono considerate.

%Le tabelle seguenti riportano risultati in forma simile a quelli mostrati in \cref{tab:iote-4tr-lin-ntr-r2-pval}, ma dove sono state applicate delle trasformazioni ai parametri considerati oppure sono state applicate regressioni esponenziali.

Nella tabella \cref{tab:iote-lin-ntr-r2-pval} sono presenti i risultati che si ottengono non accorpando i tratti in gruppi di~4.
%
\begin{table}
	\centering
	\begin{tabular}{c c *{3}{S[table-format = 2.3, table-comparator = true]}}
		\toprule
		\multicolumn{5}{p{0.7\textwidth}}{Relazioni lineari per tutti i tratti tra proporzione di isole erose rispetto alle isole presenti della stessa età in funzione degli integrali dei livelli sopra soglia}	\\
		\midrule
			&	&	{\textbf{Giovane}}	&	{\textbf{Intermedia}}	&	{\textbf{Matura}}	\\
		\midrule
		\multirow{3}*{\begin{sideways}\SIrange[range-phrase = {-}, range-units = single]{2}{3}{\mesi}\end{sideways}}	&	\# tratti	&	8	&	5	&	6	\\
			&	Mediana $R^2$	&	0.53	&	0.40	&	0.40	\\
			&	Mediana $P_\mathrm{value}$	&	0.006	&	0.028	&	0.023	\\
		\midrule
		\multirow{3}*{\begin{sideways}\SIrange[range-phrase = {-}, range-units = single]{4}{5}{\mesi}\end{sideways}}	&	\# tratti	&	13	&	{-}	&	2	\\
			&	Mediana $R^2$	&	0.57	&	{-}	&	0.46	\\
			&	Mediana $P_\mathrm{value}$	&	0.007	&	{-}	&	0.039	\\
		\midrule
		\multirow{3}*{\begin{sideways}\SI{1}{\anno}\end{sideways}}	&	\# tratti	&	3	&	6	&	7	\\
			&	Mediana $R^2$	&	0.62	&	0.67	&	0.84	\\
			&	Mediana $P_\mathrm{value}$	&	0.037	&	0.030	&	0.003	\\
		\midrule
		\multirow{3}*{\begin{sideways}\SI{2}{\anni}\end{sideways}}	&	\# tratti	&	3	&	2	&	6	\\
			&	Mediana $R^2$	&	0.65	&	0.74	&	0.85	\\
			&	Mediana $P_\mathrm{value}$	&	0.028	&	0.021	&	0.003	\\
		\bottomrule
	\end{tabular}
	\caption[numero di tratti per cui valgono relazioni significative dividendo la vegetazione in classi d'età]{numero di tratti per cui valgono relazioni significative tra tassi di erosione della vegetazione suddivisa in fasce d'età e integrale dei livelli sopra soglia secondo quattro tempi di ritorno; sono riportate le mediane degli $R^2$ e $P_\mathrm{value}$ in questi tratti; “-” indica assenza di relazioni valide.}
	\label{tab:iote-lin-ntr-r2-pval}
\end{table}
%
\\
Ci sono diversi casi con un cospicuo numero di tratti in cui ci sono relazioni valide, anche se in numero minore rispetto ai gruppi di~4 tratti.
Inoltre, per i tempi di ritorno di \SI{1}{\anno} e \SI{2}{\anni} i tratti con buone relazioni non sono adiacenti o vicini, ma sono distribuiti casualmente nel tratto di studio (non mostrato in tabella); questo può fare supporre che le relazioni siano quasi frutto del caso anziché di una regolarità naturale; ciò non avviene per i tempi di ritorno minori.
\\
Per TR \SI{< 1}{\anno}, le relazioni della classe giovane sono positive su molti tratti e presentano valori di $R^2$ e $P_\mathrm{value}$ soddisfacenti.


Si sono considerati nuovamente i tratti uniti 4 a~4 e i tassi di erosione in funzione degli integrali sopra soglia, ma si sono applicate regressioni a legge di potenza; i risultati sono presentati in \cref{tab:iote-4tr-log-ntr-r2-pval}.
%
\begin{table}
	\centering
	\begin{tabular}{c c *{3}{S[table-format = 2.3, table-comparator = true]}}
		\toprule
		\multicolumn{5}{p{0.7\textwidth}}{Relazioni a legge di potenza per i gruppi di~4 tratti tra proporzione di isole erose rispetto alle isole presenti della stessa età in funzione degli integrali dei livelli sopra soglia}	\\
		\midrule
			&	&	{\textbf{Giovane}}	&	{\textbf{Intermedia}}	&	{\textbf{Matura}}	\\
		\midrule
		\multirow{3}*{\begin{sideways}\SIrange[range-phrase = {-}, range-units = single]{2}{3}{\mesi}\end{sideways}}	&	\# tratti	&	4	&	{-}	&	6	\\
			&	Mediana $R^2$	&	0.40	&	{-}	&	0.58	\\
			&	Mediana $P_\mathrm{value}$	&	0.021	&	{-}	&	0.002	\\
		\midrule
		\multirow{3}*{\begin{sideways}\SIrange[range-phrase = {-}, range-units = single]{4}{5}{\mesi}\end{sideways}}	&	\# tratti	&	19	&	{-}	&	4	\\
			&	Mediana $R^2$	&	0.41	&	{-}	&	0.39	\\
			&	Mediana $P_\mathrm{value}$	&	0.016	&	{-}	&	0.031	\\
		\midrule
		\multirow{3}*{\begin{sideways}\SI{1}{\anno}\end{sideways}}	&	\# tratti	&	{-}	&	4	&	{-}	\\
			&	Mediana $R^2$	&	{-}	&	0.51	&	{-}	\\
			&	Mediana $P_\mathrm{value}$	&	{-}	&	0.030	&	{-}	\\
		\midrule
		\multirow{3}*{\begin{sideways}\SI{2}{\anni}\end{sideways}}	&	\# tratti	&	{-}	&	{-}	&	{-}	\\
			&	Mediana $R^2$	&	{-}	&	{-}	&	{-}	\\
			&	Mediana $P_\mathrm{value}$	&	{-}	&	{-}	&	{-}	\\
		\bottomrule
	\end{tabular}
	\caption[numero di tratti in gruppi di~4 con relazioni esponenziali significative dividendo la vegetazione in classi d'età]{numero di tratti per cui valgono relazioni esponenziali significative tra tassi di erosione della vegetazione suddivisa in fasce d'età e integrale dei livelli sopra soglia secondo quattro tempi di ritorno; sono riportate le mediane degli $R^2$ e $P_\mathrm{value}$ in questi tratti; “-” indica assenza di relazioni valide; i tratti sono stati accorpati in gruppi di~4.}
	\label{tab:iote-4tr-log-ntr-r2-pval}
\end{table}
%
\\
Solamente i dati riguardo la vegetazione giovane, usando un TR \SIrange[range-phrase = {-}, range-units = single]{4}{5}{\mesi}, sono rappresentati da valide relazioni esponenziali, mentre in quasi tutti gli altri casi questa combinazione fornisce poche o nessuna relazione.
Solo la vegetazione matura, per il minimo TR considerato, mostra di avere qualche tratto in cui si presentano buone relazioni.


Ricercando relazioni lineari, al posto dei tassi di erosione sono stati considerati gli areali, in \si{[\m\tothe{2}]}, delle isole erose in funzione dell'integrale die libelli sopra soglia; sono stati mantenuti i gruppi di~4 tratti; la \cref{tab:nc-4tr-lin-ntr-r2-pval} mostra i risultati.
%
\begin{table}
	\centering
	\begin{tabular}{c c *{3}{S[table-format = 2.3, table-comparator = true]}}
		\toprule
		\multicolumn{5}{p{0.7\textwidth}}{Relazioni lineari in gruppi di~4 tratti tra areale delle isole erose in funzione degli integrali dei livelli sopra soglia}	\\
		\midrule
			&	&	{\textbf{Giovane}}	&	{\textbf{Intermedia}}	&	{\textbf{Matura}}	\\
		\midrule
		\multirow{3}*{\begin{sideways}\SIrange[range-phrase = {-}, range-units = single]{2}{3}{\mesi}\end{sideways}}	&	\# tratti	&	{-}	&	4	&	6	\\
			&	Mediana $R^2$	&	{-}	&	0.69	&	0.32	\\
			&	Mediana $P_\mathrm{value}$	&	{-}	&	<0.001	&	0.044	\\
		\midrule
		\multirow{3}*{\begin{sideways}\SIrange[range-phrase = {-}, range-units = single]{4}{5}{\mesi}\end{sideways}}	&	\# tratti	&	4	&	11	&	{-}	\\
			&	Mediana $R^2$	&	0.35	&	0.38	&	{-}	\\
			&	Mediana $P_\mathrm{value}$	&	0.021	&	0.043	&	{-}	\\
		\midrule
		\multirow{3}*{\begin{sideways}\SI{1}{\anno}\end{sideways}}	&	\# tratti	&	4	&	7	&	{-}	\\
			&	Mediana $R^2$	&	0.53	&	0.64	&	{-}	\\
			&	Mediana $P_\mathrm{value}$	&	0.017	&	0.013	&	{-}	\\
		\midrule
		\multirow{3}*{\begin{sideways}\SI{2}{\anni}\end{sideways}}	&	\# tratti	&	4	&	4	&	{-}	\\
			&	Mediana $R^2$	&	0.73	&	0.38	&	{-}	\\
			&	Mediana $P_\mathrm{value}$	&	0.002	&	0.043	&	{-}	\\
		\bottomrule
	\end{tabular}
	\caption[numero di tratti nei gruppi di~4 tratti con relazioni significative dividendo la vegetazione in classi d'età e considerando gli areali anziché i tassi di erosione]{numero di tratti per cui valgono relazioni significative tra areali di erosione della vegetazione suddivisa in fasce d'età e integrale dei livelli sopra soglia secondo quattro tempi di ritorno; sono riportate le mediane degli $R^2$ e $P_\mathrm{value}$ in questi tratti; “-” indica assenza di relazioni valide; i tratti sono stati uniti 4 a~4.}
	\label{tab:nc-4tr-lin-ntr-r2-pval}
\end{table}
%
\\
La classe giovane presenta pochi tratti in cui le relazioni sono valide, mentre la vegetazione intermedia mostra discrete relazioni per i TR intermedi; la fascia matura ha qualche relazione solo per TR \SIrange[range-phrase = {-}, range-units = single]{2}{3}{\mesi}, ma con $R^2$ e $P_\mathrm{value}$ appena compreso entro le soglie che erano state poste ($R^2 \num{> 0.3}$ e $P_\mathrm{value} \num{< 0.05}$).


Mantenendo gli stessi dati considerati precedentemente (areale in \si{[\m\tothe{2}]} di erosione in funzione degli integrali delle piene sopra soglia, gruppi di~4 tratti), si ricercano regressioni a legge di potenza; i risultati sono in \cref{tab:nc-4tr-log-ntr-r2-pval}.
%
\begin{table}
	\centering
	\begin{tabular}{c c *{3}{S[table-format = 2.3, table-comparator = true]}}
		\toprule
		\multicolumn{5}{p{0.7\textwidth}}{Relazioni a legge di potenza in gruppi di~4 tratti tra areale delle isole erose in funzione degli integrali dei livelli sopra soglia}	\\
		\midrule
			&	&	{\textbf{Giovane}}	&	{\textbf{Intermedia}}	&	{\textbf{Matura}}	\\
		\midrule
		\multirow{3}*{\begin{sideways}\SIrange[range-phrase = {-}, range-units = single]{2}{3}{\mesi}\end{sideways}}	&	\# tratti	&	6	&	4	&	6	\\
			&	Mediana $R^2$	&	0.38	&	0.61	&	0.37	\\
			&	Mediana $P_\mathrm{value}$	&	0.024	&	<0.001	&	0.028	\\
		\midrule
		\multirow{3}*{\begin{sideways}\SIrange[range-phrase = {-}, range-units = single]{4}{5}{\mesi}\end{sideways}}	&	\# tratti	&	6	&	4	&	{-}	\\
			&	Mediana $R^2$	&	0.40	&	0.44	&	{-}	\\
			&	Mediana $P_\mathrm{value}$	&	0.027	&	0.005	&	{-}	\\
		\midrule
		\multirow{3}*{\begin{sideways}\SI{1}{\anno}\end{sideways}}	&	\# tratti	&	{-}	&	4	&	{-}	\\
			&	Mediana $R^2$	&	{-}	&	0.45	&	{-}	\\
			&	Mediana $P_\mathrm{value}$	&	{-}	&	0.023	&	{-}	\\
		\midrule
		\multirow{3}*{\begin{sideways}\SI{2}{\anni}\end{sideways}}	&	\# tratti	&	{-}	&	{-}	&	{-}	\\
			&	Mediana $R^2$	&	{-}	&	{-}	&	{-}	\\
			&	Mediana $P_\mathrm{value}$	&	{-}	&	{-}	&	{-}	\\
		\bottomrule
	\end{tabular}
	\caption[numero di tratti nei gruppi di~4 tratti con relazioni esponenziali significative dividendo la vegetazione in classi d'età e considerando gli areali anziché i tassi di erosione]{numero di tratti per cui valgono relazioni esponenziali significative tra areali di erosione della vegetazione suddivisa in fasce d'età e integrale dei livelli sopra soglia secondo quattro tempi di ritorno; sono riportate le mediane degli $R^2$ e $P_\mathrm{value}$ in questi tratti; “-” indica assenza di relazioni valide; i tratti sono stati uniti 4 a~4.}
	\label{tab:nc-4tr-log-ntr-r2-pval}
\end{table}
%
Le regressioni esponenziali, in questo caso, portano a considerare come validi un minor numero di tratti rispetto a quelle lineari; in alcuni casi permettono di avere più tratti validi (fascia giovane e TR \SI{< 1}{\anno}, in altri diminuiscono drasticamente il numero di tratti validi (le altre fasce e TR).


Si è provato a dividere l'integrale delle piene per la somma della durata di tutte le piene sopra soglia che hanno luogo in ciascun intervallo; questo rapporto indica il livello medio costante sopra soglia che avrebbe una piena che durasse tanto quanto sono durate tutte le piene osservate.
Sono stati mantenuti l'unione dei tratti 4 a~4 e la rappresentazione dell'erosione come tasso.
La \cref{tab:iote-4tr-lin-tuttep-ntr-r2-pval} mostra i risultati ottenuti. 
%
\begin{table}
	\centering
	\begin{tabular}{c c *{3}{S[table-format = 2.3, table-comparator = true]}}
		\toprule
		\multicolumn{5}{p{0.7\textwidth}}{Relazioni lineari in gruppi di~4 tratti tra proporzione di isole erose rispetto alle isole presenti della stessa età in funzione del rapporto tra gli integrali dei livelli sopra soglia e la durata complessiva delle piene in ogni confronto}	\\
		\midrule
			&	&	{\textbf{Giovane}}	&	{\textbf{Intermedia}}	&	{\textbf{Matura}}	\\
		\midrule
		\multirow{3}*{\begin{sideways}\SIrange[range-phrase = {-}, range-units = single]{2}{3}{\mesi}\end{sideways}}	&	\# tratti	&	{-}	&	{-}	&	{-}	\\
			&	Mediana $R^2$	&	{-}	&	{-}	&	{-}	\\
			&	Mediana $P_\mathrm{value}$	&	{-}	&	{-}	&	{-}	\\
		\midrule
		\multirow{3}*{\begin{sideways}\SIrange[range-phrase = {-}, range-units = single]{4}{5}{\mesi}\end{sideways}}	&	\# tratti	&	4	&	{-}	&	{-}	\\
			&	Mediana $R^2$	&	0.36	&	{-}	&	{-}	\\
			&	Mediana $P_\mathrm{value}$	&	0.040	&	{-}	&	{-}	\\
		\midrule
		\multirow{3}*{\begin{sideways}\SI{1}{\anno}\end{sideways}}	&	\# tratti	&	{-}	&	{-}	&	{-}	\\
			&	Mediana $R^2$	&	{-}	&	{-}	&	{-}	\\
			&	Mediana $P_\mathrm{value}$	&	{-}	&	{-}	&	{-}	\\
		\midrule
		\multirow{3}*{\begin{sideways}\SI{2}{\anni}\end{sideways}}	&	\# tratti	&	4	&	{-}	&	{-}	\\
			&	Mediana $R^2$	&	0.55	&	{-}	&	{-}	\\
			&	Mediana $P_\mathrm{value}$	&	0.015	&	{-}	&	{-}	\\
		\bottomrule
	\end{tabular}
	\caption[numero di tratti nei gruppi di~4 tratti con relazioni significative considerando i tassi di erosione e i livelli medi sopra soglia durante le piene]{numero di tratti per cui valgono relazioni significative tra tassi di erosione della vegetazione suddivisa in fasce d'età e rapporto tra integrale dei livelli sopra soglia e durata delle piene in ogni confronto secondo quattro tempi di ritorno; sono riportate le mediane degli $R^2$ e $P_\mathrm{value}$ in questi tratti; “-” indica assenza di relazioni valide; i tratti sono stati uniti 4 a~4.}
	\label{tab:iote-4tr-lin-tuttep-ntr-r2-pval}
\end{table}
%
\\
Le poche relazioni valide che si osservano sono probabilmente casuali, mentre la stragrande maggioranza di combinazioni senza regressioni lineari accettabili fa concludere che questi parametri non siano adatti per rappresentare questo fenomeno.


Infine, si è diviso l'integrale dei livelli per la durata dei confronti: si ottiene il livello medio costante sopra soglia che avrebbe una piena di durata pari al confronto considerato.
La \cref{tab:iote-4tr-lin-interv-ntr-r2-pval} mostra i tratti in cui si trovano relazioni valide.
%
\begin{table}
	\centering
	\begin{tabular}{c c *{3}{S[table-format = 2.3, table-comparator = true]}}
		\toprule
		\multicolumn{5}{p{0.7\textwidth}}{Relazioni lineari in gruppi di~4 tratti tra proporzione di isole erose rispetto alle isole presenti della stessa età in funzione del rapporto tra gli integrali dei livelli sopra soglia e la durata di ogni confronto}	\\
		\midrule
			&	&	{\textbf{Giovane}}	&	{\textbf{Intermedia}}	&	{\textbf{Matura}}	\\
		\midrule
		\multirow{3}*{\begin{sideways}\SIrange[range-phrase = {-}, range-units = single]{2}{3}{\mesi}\end{sideways}}	&	\# tratti	&	12	&	6	&	{-}	\\
			&	Mediana $R^2$	&	0.46	&	0.35	&	{-}	\\
			&	Mediana $P_\mathrm{value}$	&	0.010	&	0.033	&	{-}	\\
		\midrule
		\multirow{3}*{\begin{sideways}\SIrange[range-phrase = {-}, range-units = single]{4}{5}{\mesi}\end{sideways}}	&	\# tratti	&	12	&	{-}	&	{-}	\\
			&	Mediana $R^2$	&	0.49	&	{-}	&	{-}	\\
			&	Mediana $P_\mathrm{value}$	&	0.011	&	{-}	&	{-}	\\
		\midrule
		\multirow{3}*{\begin{sideways}\SI{1}{\anno}\end{sideways}}	&	\# tratti	&	8	&	{-}	&	{-}	\\
			&	Mediana $R^2$	&	0.54	&	{-}	&	{-}	\\
			&	Mediana $P_\mathrm{value}$	&	0.023	&	{-}	&	{-}	\\
		\midrule
		\multirow{3}*{\begin{sideways}\SI{2}{\anni}\end{sideways}}	&	\# tratti	&	4	&	4	&	{-}	\\
			&	Mediana $R^2$	&	0.56	&	0.44	&	{-}	\\
			&	Mediana $P_\mathrm{value}$	&	0.013	&	0.035	&	{-}	\\
		\bottomrule
	\end{tabular}
	\caption[numero di tratti nei gruppi di~4 tratti con relazioni significative considerando i tassi di erosione e i livelli medi sopra soglia durante i confronti]{numero di tratti per cui valgono relazioni significative tra tassi di erosione della vegetazione suddivisa in fasce d'età e rapporto tra integrale dei livelli sopra soglia e durata dei confronti secondo quattro tempi di ritorno; sono riportate le mediane degli $R^2$ e $P_\mathrm{value}$ in questi tratti; “-” indica assenza di relazioni valide; i tratti sono stati uniti 4 a~4.}
	\label{tab:iote-4tr-lin-interv-ntr-r2-pval}
\end{table}
%
\\
Ci sono significativamente più tratti validi rispetto al precedente rapporto dell'integrale, soprattutto per la fascia giovane; le altre classe d'età hanno in genere relazioni deboli, tranne la classe intermedia per TR \SIrange[range-phrase = {-}, range-units = single]{2}{3}{\mesi}.



\subsection{Le relazioni valide tra piene ed isole erose}
\label{sec:migliori-tratti-piene-erosione}
L'obiettivo è di trovare un unico set di trasformazioni che permetta di prevedere il comportamento della vegetazione in molti tempi di ritorno diversi; con la suddivisione in classi d'età, si ricerca similmente il set che dia il maggior numero di relazioni valide su più classi.
\\
A fronte dei risultati presentati, si ritiene che per ottenere quante più regressioni valide su più tratti adiacenti e in più fasce d'età occorra:
%
\begin{itemize}
	\item considerare i tassi di erosione in funzione degli integrali dei livelli sopra soglia;
	\item accorpare i tratti in gruppi di~4;
	\item cercare regressioni lineari.
\end{itemize}
%
Sebbene alcune classi d'età non abbiano abbastanza tratti validi per alcuni tempi di ritorno (come la fascia intermedia per il TR minimo), questo set ha un gran numero di tratti validi per molte classi e TR.
Nel caso in cui non si accorpano i tratti 4 a~4 ci sono molte relazioni valide, ma queste si riferiscono a tratti distanti tra loro e con caratteristiche diverse; si sono preferiti i casi in cui i tratti con buone regressioni fossero adiacenti poiché si ritiene che questo rifletta la naturale regolarità presente nei fenomeni che si stanno osservando.
\\
Si ritengono significative le relazioni trovate per ogni classe nel TR~\SIrange[range-phrase = {-}, range-units = single]{2}{3}{\mesi}, per la classe giovane nel TR~\SIrange[range-phrase = {-}, range-units = single]{4}{5}{\mesi} e nel TR~\SI{1}{\anno} e per le classi di vegetazione intermedia e matura nel TR~\SI{1}{\anno}.
Ogni relazione ha un discreto numero di punti (da un massimo di~17 punti ad un minimo di~7 punti, in media più di~12 punti)
\\
Il numero di tratti validi per ogni TR e ogni classe d'età sono stati mostrati in \cref{tab:iote-4tr-lin-ntr-r2-pval}.
I grafici in \cref{graph:giov-iote-4tr-buono}, \cref{graph:interm-iote-4tr-buono} e \cref{graph:mat-iote-4tr-buono} mostrano i dati e le regressioni considerate valide rispettivamente per le fasce d'età giovane, intermedia e matura, con l'unione dei tratti 4 a~4;
i valori delle regressioni lineari e le equazioni delle rette sono riportate per le tre classi d'età nella \cref{tab:giov-iote-4tr-buono}, nella \cref{tab:interm-iote-4tr-buono} e nella \cref{tab:mat-iote-4tr-buono}.

Dati gli ottimi risultati ottenuti per la classe giovane e TR~\SIrange[range-phrase = {-}, range-units = single]{4}{5}{\mesi}, sono stati unitamente considerati tutti i gruppi di~4 tratti dove ci sono relazioni valide: mettendo tutti i dati in un grafico unico (\cref{graph:giov-iote-4tr-buono-accorpato}), si ottiene una regressione lineare discretamente buona ($R^2 \num{> 0.4}$ e $P_\mathrm{value} \num{\ll 0.05}$).

% GIOVANE
\begin{figure}
	\centering
	\tikzsetnextfilename{giov_iote_4tr_buono}
\begin{tikzpicture}
	\begin{groupplot}[
		group style = {
			group size = 3 by 2,
			y descriptions at = edge left,
			x descriptions at = edge bottom,
			horizontal sep = 0.1cm,
			vertical sep = 0.1cm,
		},
		width = 0.4\textwidth,
		height = 0.4\textwidth,
		ymin = 0,
		ymax = 0.65,
		enlarge y limits = 0.05,
		enlarge x limits = 0.05,
		ylabel = {Erosione / Vegetazione},
		xlabel = {Int livelli \si{[\m\giorno]}},
		grid = major,		
		]
		
	\nextgroupplot[] % gruppo 1
			\addplot[only marks]
	        	table [x = integrale_piene, y = gr_1] {graphics/data/giov_iote_4tr_buono.txt};
	        
	        \addplot[
	        	line width = 2pt,
	        	domain = 0:2,
	        	samples = 3,
				green,
	        	]
	        	{0.1852 * x + 0.1032};
			
			\node at (axis description cs: 0,1) [draw = black, fill = white, anchor = north west, align = left] {1$\div$4};
			
			\node at (axis description cs: 1,0) [draw = black, fill = white, anchor = south east] {$R^2 = 0.49$};
	
	\nextgroupplot[
		title = {TR \SIrange[range-phrase={-}, range-units = single]{4}{5}{\mesi}},
		] % gruppo 2
			\addplot[only marks]
	        	table [x = integrale_piene, y = gr_2] {graphics/data/giov_iote_4tr_buono.txt};
	        
	        \addplot[
	        	line width = 2pt,
	        	domain = 0:2,
	        	samples = 3,
				green,
	        	]
	        	{0.1700 * x + 0.0669};
			
			\node at (axis description cs: 0,1) [draw = black, fill = white, anchor = north west, align = left] {5$\div$8};
			
			\node at (axis description cs: 1,0) [draw = black, fill = white, anchor = south east] {$R^2 = 0.46$};
	
	\nextgroupplot[] % gruppo 3
			\addplot[only marks]
	        	table [x = integrale_piene, y = gr_3] {graphics/data/giov_iote_4tr_buono.txt};
	        
	        \addplot[
	        	line width = 2pt,
	        	domain = 0:2,
	        	samples = 3,
				green,
	        	]
	        	{0.1392 * x + 0.0775};
			
			\node at (axis description cs: 0,1) [draw = black, fill = white, anchor = north west, align = left] {10$\div$12};
			
			\node at (axis description cs: 1,0) [draw = black, fill = white, anchor = south east] {$R^2 = 0.48$};
	
	\nextgroupplot[] % gruppo 4
			\addplot[only marks]
	        	table [x = integrale_piene, y = gr_4] {graphics/data/giov_iote_4tr_buono.txt};
	        
	        \addplot[
	        	line width = 2pt,
	        	domain = 0:2,
	        	samples = 3,
				green,
	        	]
	        	{0.2126 * x + 0.1189};
			
			\node at (axis description cs: 0,1) [draw = black, fill = white, anchor = north west, align = left] {13$\div$16};
			
			\node at (axis description cs: 1,0) [draw = black, fill = white, anchor = south east] {$R^2 = 0.40$};
	
	\nextgroupplot[] % gruppo 5
			\addplot[only marks]
	        	table [x = integrale_piene, y = gr_5] {graphics/data/giov_iote_4tr_buono.txt};
	        
	        \addplot[
	        	line width = 2pt,
	        	domain = 0:2,
	        	samples = 3,
				green,
	        	]
	        	{0.2553 * x + 0.1076};
			
			\node at (axis description cs: 0,1) [draw = black, fill = white, anchor = north west, align = left] {17$\div$20};
			
			\node at (axis description cs: 1,0) [draw = black, fill = white, anchor = south east] {$R^2 = 0.59$};
	
	\end{groupplot}
\end{tikzpicture}
	\caption[proporzione di vegetazione giovane erosa in funzione dell'integrale dei livelli sopra la soglia di \SI{2}{\m}; tratti uniti quattro a quattro]{proporzione di vegetazione giovane erosa rispetto alla vegetazione giovane presente in funzione dell'integrale dei livelli (Int livelli) sopra la soglia di \SI{1.5}{\m}, di \SI{2}{\m} e di \SI{2.4}{\m}; i tratti sono raggruppati quattro a quattro; le rette verdi sono le regressioni lineari, le cui equazioni e valori di $P_\mathrm{value}$ sono riportati nella \cref{tab:giov-iote-4tr-buono}.}
	\label{graph:giov-iote-4tr-buono}
\end{figure}
%
\begin{table}
	\centering
	\begin{tabular}{c c c S[table-format = 0.2] S[table-format = 0.3, table-comparator = true]}
		\toprule
		{\textbf{Tempo di}}	&	\textbf{Tratti}			&	\textbf{Equazione}		&	\multicolumn{1}{c}{$\mathbf{R^2}$}	&	\multicolumn{1}{c}{$	\mathbf{P_\mathrm{value}}$}	\\
		{\textbf{ritorno}}	&	\textbf{di validità}	&	\textbf{della retta}	&	\\
		\midrule
		\multirow{3}*{\SIrange[range-phrase = {-}, range-units = single]{2}{3}{\mesi}}	&	5$\div$8	&	$y = 0.0348 \, x + 0.0812$	&	0.32	&	<0.02	\\
			&	13$\div$16	&	$y = 0.0516 \, x + 0.1127$	&	0.41	&	<0.005	\\
			&	17$\div$20	&	$y = 0.0564 \, x + 0.1184$	&	0.39	&	<0.008	\\
		\midrule
		\multirow{5}*{\SIrange[range-phrase = {-}, range-units = single]{4}{5}{\mesi}}	&	1$\div$4	&	$y = 0.1837 \, x + 0.1052$	&	0.51	&	<0.006	\\
			&	5$\div$8	&	$y = 0.1762 \, x + 0.0600$	&	0.49	&	<0.003	\\
			&	10$\div$12	&	$y = 0.1470 \, x + 0.0679$	&	0.53	&	<0.008	\\
			&	13$\div$16	&	$y = 0.2149 \, x + 0.1164$	&	0.43	&	<0.005	\\
			&	17$\div$20	&	$y = 0.2553 \, x + 0.1076$	&	0.59	&	<0.004	\\
		\midrule
		\SI{1}{\anno}	&	5$\div$8	&	$y = 0.4425 \, x + 0.0900$	&	0.48	&	<0.003	\\
		\bottomrule
	\end{tabular}
	\caption[equazioni, $R^2$ e $P_\mathrm{value}$ delle regressioni per la vegetazione giovane]{equazioni, $R^2$ e $P_\mathrm{value}$ delle regressioni per la vegetazione giovane, mostrate nel grafico in \cref{graph:giov-iote-4tr-buono}.}
	\label{tab:giov-iote-4tr-buono}
\end{table}
%
%	INTERMEDIA
\begin{figure}
	\centering
	\tikzsetnextfilename{interm_iote_4tr_buono}
\begin{tikzpicture}
	\begin{groupplot}[
		group style = {
			group size = 3 by 1,
			y descriptions at = edge left,
			x descriptions at = edge bottom,
			horizontal sep = 0.1cm,
			vertical sep = 0.1cm,
		},
		width = 0.4\textwidth,
		height = 0.4\textwidth,
		ymin = 0,
		ymax = 0.4,
		enlarge y limits = 0.05,
		enlarge x limits = 0.05,
		ylabel = {Erosione / Isole},
		xlabel = {Int livelli \si{[\m\giorno]}},
		grid = major,		
		]
		
	\nextgroupplot[
		title = {TR \SIrange[range-phrase={-}, range-units=single]{2}{3}{\mesi}},
		] % gruppo 5 livello 1.5m
			\addplot[only marks]
	        	table [x = integrale_piene_tr_0.2, y = gr_5_tr_0.2] {graphics/data/interm_iote_4tr_buono.txt};
	        
	        \addplot[
	        	line width = 2pt,
	        	domain = 0:6.9,
	        	samples = 3,
				green,
	        	]
	        	{0.0293 * x + 0.0809};
			
			\node at (axis description cs: 0,1) [draw = black, fill = white, anchor = north west, align = left, name = gr_5_tr_02] {17$\div$20};
			
			\node at (axis description cs: 1,0) [draw = black, fill = white, anchor = south east] {$R^2 = 0.41$};
	
	\nextgroupplot[ % gruppo 2 livello 2.4
		xmin = 0,
		xmax = 1,
		title = {TR \SI{\sim 1}{\anno}},
		]
			\addplot[only marks]
	        	table [x = integrale_piene_tr_1, y = gr_2_tr_1] {graphics/data/interm_iote_4tr_buono.txt};
	        
	        \addplot[
	        	line width = 2pt,
	        	domain = 0:1,
	        	samples = 3,
				green,
	        	]
	        	{0.2819 * x + 0.0331};
			
			\node at (axis description cs: 0,1) [draw = black, fill = white, anchor = north west, align = left, name = gr_2_tr_1] {5$\div$8};
			
			\node at (axis description cs: 1,0) [draw = black, fill = white, anchor = south east] {$R^2 = 0.62$};
	
	\nextgroupplot[ % gruppo 3 livello 2.4
		xmin = 0,
		xmax = 1,
		title = {TR \SI{\sim 1}{\anno}},
		]
			\addplot[only marks]
	        	table [x = integrale_piene_tr_1, y = gr_3_tr_1] {graphics/data/interm_iote_4tr_buono.txt};
	        
	        \addplot[
	        	line width = 2pt,
	        	domain = 0:1,
	        	samples = 3,
				green,
	        	]
	        	{0.1335 * x + 0.0109};
			
			\node at (axis description cs: 0,1) [draw = black, fill = white, anchor = north west, align = left, name = gr_3_tr_1] {10$\div$12};
			
			\node at (axis description cs: 1,0) [draw = black, fill = white, anchor = south east] {$R^2 = 0.67$};
	
	\end{groupplot}
\end{tikzpicture}
	\caption[proporzione di vegetazione intermedia erosa in funzione dell'integrale dei livelli sopra due soglie; tratti uniti quattro a quattro]{proporzione di vegetazione intermedia erosa rispetto alla vegetazione intermedia presente in funzione dell'integrale dei livelli (Int livelli) sopra la soglia di \SI{1.5}{\m} per i tratti 17$\div$20, sopra la soglia di \SI{2.4}{\m} per i tratti 5$\div$8 e 10$\div$12; le rette verdi sono le regressioni lineari, le cui equazioni e valori di $P_\mathrm{value}$ sono riportati nella \cref{tab:interm-iote-4tr-buono}.}
	\label{graph:interm-iote-4tr-buono}
\end{figure}
%
\begin{table}
	\centering
	\begin{tabular}{
		c
		S[table-format = 1.1]
		c
		c
		S[table-format = 0.2]
		S[table-format = 0.3, table-comparator = true]
	}
		\toprule
		{\textbf{Tempo di}}	&	{\textbf{Livello soglia}}	&	\textbf{Tratti}			&	\textbf{Equazione}		&	\multicolumn{1}{c}{$\mathbf{R^2}$}	&	{$\mathbf{P_\mathrm{value}}$}	\\
		{\textbf{ritorno}}	&	\multicolumn{1}{s}{[\m]}	&	\textbf{di validità}	&	\textbf{della retta}	&	\\
		\midrule
		\SIrange[range-phrase = {-}, range-units = single]{2}{3}{\mesi}	&	1.5	&	17$\div$20	&	$y = 0.0293 \, x + 0.0809$	&	0.41	&	<0.02	\\
		\SI{1}{\anno}	&	2.4	&	5$\div$8	&	$y = 0.2819 \, x + 0.0331$	&	0.62	&	<0.008	\\
		\SI{1}{\anno}	&	2.4	&	10$\div$12	&	$y = 0.1335 \, x + 0.0109$	&	0.67	&	<0.03	\\
		\bottomrule
	\end{tabular}
	\caption[equazioni, $R^2$ e $P_\mathrm{value}$ delle regressioni per la vegetazione intermedia]{equazioni, $R^2$ e $P_\mathrm{value}$ delle regressioni per la vegetazione intermedia, mostrate nel grafico in \cref{graph:interm-iote-4tr-buono}.}
	\label{tab:interm-iote-4tr-buono}
\end{table}
%
%	MATURA
\begin{figure}
	\centering
	\tikzsetnextfilename{mat_iote_4tr_buono}
\begin{tikzpicture}
	\begin{groupplot}[
		group style = {
			group size = 2 by 2,
			y descriptions at = edge left,
			xlabels at = edge bottom,
			horizontal sep = 0.1cm,
			vertical sep = 0.75cm,
		},
		width = 0.4\textwidth,
		height = 0.4\textwidth,
		ymin = 0,
		ymax = 0.4,
		enlargelimits = 0.05,
		ylabel = {Erosione / isole},
		xlabel = {Int livelli \si{[\m\giorno]}},
		grid = major,		
		]
		
	\nextgroupplot[ % gruppo 5 livello 1.5m
%		xtick pos = right,
%		xlabel = {Int livelli \si{[\m\giorno]}},
		xmin = 0,
		xmax = 7,
		]
			\addplot[only marks]
	        	table [x = integrale_piene_tr_0.2, y = gr_5_tr_0.2] {graphics/data/mat_iote_4tr_buono.txt};
	        
	        \addplot[
	        	line width = 2pt,
	        	domain = 0:7,
	        	samples = 3,
				green,
	        	]
	        	{0.0220 * x + 0.0259};
			
			\node at (axis description cs: 0,1) [draw = black, fill = white, anchor = north west, align = left, name = gr_5_tr_02] {17$\div$20};
			
			\node at (gr_5_tr_02.south west) [draw = black, fill = white, anchor = north west, align = left] {TR \SIrange[range-phrase={-}, range-units=single]{2}{3}{\mesi}};
			
	\nextgroupplot[ % gruppo 6 livello 1.5m
%		xtick pos = right,
%		xlabel = {Int livelli \si{[\m\giorno]}},
		xmin = 0,
		xmax = 7,
		]
			\addplot[only marks]
	        	table [x = integrale_piene_tr_0.2, y = gr_6_tr_0.2] {graphics/data/mat_iote_4tr_buono.txt};
	        
	        \addplot[
	        	line width = 2pt,
	        	domain = 0:7,
	        	samples = 3,
				green,
	        	]
	        	{0.0169 * x + 0.0161};
			
			\node at (axis description cs: 0,1) [draw = black, fill = white, anchor = north west, align = left, name = gr_6_tr_02] {21$\div$22};
			
			\node at (gr_6_tr_02.south west) [draw = black, fill = white, anchor = north west, align = left] {TR \SIrange[range-phrase={-}, range-units=single]{2}{3}{\mesi}};
	
	\nextgroupplot[ % gruppo 1 livello 2.4
		xmin = 0,
		xmax = 1,
		]
			\addplot[only marks]
	        	table [x = integrale_piene_tr_1, y = gr_1_tr_1] {graphics/data/mat_iote_4tr_buono.txt};
	        
	        \addplot[
	        	line width = 2pt,
	        	domain = 0.1:1,
	        	samples = 3,
				green,
	        	]
	        	{0.2504 * x - 0.0326};
			
			\node at (axis description cs: 0,1) [draw = black, fill = white, anchor = north west, align = left, name = gr_1_tr_1] {1$\div$4};
			
			\node at (gr_1_tr_1.south west) [draw = black, fill = white, anchor = north west, align = left] {TR \SI{\sim 1}{\anno}};
	
	\nextgroupplot[ % gruppo 2 livello 2.4
		xmin = 0,
		xmax = 1,
		]
			\addplot[only marks]
	        	table [x = integrale_piene_tr_1, y = gr_2_tr_1] {graphics/data/mat_iote_4tr_buono.txt};
	        
	        \addplot[
	        	line width = 2pt,
	        	domain = 0:1,
	        	samples = 3,
				green,
	        	]
	        	{0.1629 * x + 0.0119};
			
			\node at (axis description cs: 0,1) [draw = black, fill = white, anchor = north west, align = left, name = gr_2_tr_1] {5$\div$8};
			
			\node at (gr_2_tr_1.south west) [draw = black, fill = white, anchor = north west, align = left] {TR \SI{\sim 1}{\anno}};
	
	\end{groupplot}
\end{tikzpicture}
	\caption[proporzione di vegetazione matura erosa in funzione dell'integrale dei livelli sopra due soglie; tratti uniti quattro a quattro]{proporzione di vegetazione matura erosa rispetto alla vegetazione matura presente in funzione dell'integrale dei livelli (Int livelli) sopra la soglia di \SI{1.5}{\m} per i tratti 17$\div$22, sopra la soglia di \SI{2.4}{\m} per i tratti 1$\div$8; le rette verdi sono le regressioni lineari, le cui equazioni e valori di $P_\mathrm{value}$ sono riportati nella \cref{tab:mat-iote-4tr-buono}.}
	\label{graph:mat-iote-4tr-buono}
\end{figure}
%
\begin{table}
	\centering
	\begin{tabular}{
		c		
		S[table-format = 1.1]
		c
		c
		S[table-format = 0.2]
		S[table-format = 0.3, table-comparator = true]
	}
		\toprule
		{\textbf{Tempo di}}	&	{\textbf{Livello soglia}}	&	\textbf{Tratti}			&	\textbf{Equazione}		&	\multicolumn{1}{c}{$\mathbf{R^2}$}	&	\multicolumn{1}{c}{$	\mathbf{P_\mathrm{value}}$}	\\
		{\textbf{ritorno}}	&	\multicolumn{1}{s}{[\m]}	&	\textbf{di validità}	&	\textbf{della retta}	&	\\
		\midrule
		\SIrange[range-phrase = {-}, range-units = single]{2}{3}{\mesi}	&	1.5	&	17$\div$20	&	$y = 0.0220 \, x + 0.0259$	&	0.69	&	<0.0005	\\
		\SIrange[range-phrase = {-}, range-units = single]{2}{3}{\mesi}	&	1.5	&	21$\div$22	&	$y = 0.0169 \, x + 0.0161$	&	0.56	&	<0.004	\\
		\SI{1}{\anno}	&	2.4	&	1$\div$4	&	$y = 0.2504 \, x - 0.0326$	&	0.54	&	<0.04	\\
		\SI{1}{\anno}	&	2.4	&	5$\div$8	&	$y = 0.1629 \, x + 0.0119$	&	0.54	&	<0.02	\\
		\bottomrule
	\end{tabular}
	\caption[equazioni, $R^2$ e $P_\mathrm{value}$ delle regressioni per la vegetazione matura]{equazioni, $R^2$ e $P_\mathrm{value}$ delle regressioni per la vegetazione matura, mostrate nel grafico in \cref{graph:mat-iote-4tr-buono}.}
	\label{tab:mat-iote-4tr-buono}
\end{table}
%
%
\begin{figure}
	\centering
	\tikzsetnextfilename{giov_iote_4tr_buono_accorpato}
\begin{tikzpicture}
	\begin{axis}[
		width = 0.98\textwidth,
		height = 0.5\textwidth,
		enlarge x limits = 0.02,
		ylabel = {Erosione / Isole},
		xlabel = {Integrale dei livelli sopra soglia \si{[\m\giorno]}},
		tick label style = {
			/pgf/number format/.cd,
			fixed,
			fixed zerofill,
			precision = 1,
			/tikz/.cd,
		},
		grid = major,
		title = {TR \SIrange[range-phrase={-}, range-units = single]{4}{5}{\mesi}},		
		]
		
		\pgfplotsforeachungrouped \gr in {1,2,...,5} {
		\addplot[only marks]
			table [x = integrale_piene_tr_0.5, y = gr_\gr_tr_0.5] {graphics/data/giov_iote_4tr_buono.txt};
		}
	
		\addplot[
			line width = 2pt,
			domain = 0:2,
			samples = 3,
			green,
			]
			{0.1859 * x + 0.0987};
			
		\node at (axis description cs: 0,1) [draw = black, fill = white, anchor = north west, align = left, name = range] {1$\div$22};
		\node at (range.south west) [draw = black, fill = white, anchor = north west, align = left] {$y = 0.1859 \, x + 0.0987$};
		
		\node at (axis description cs: 1,0) [draw = black, fill = white, anchor = south east, align = left] {$R^2 = 0.42 \quad P_\mathrm{value} < 0.00001$};
	
	\end{axis}
\end{tikzpicture}
	\caption[proporzione di vegetazione giovane erosa in funzione dell'integrale dei livelli sopra la soglia di \SI{2}{\m}; tutti i tratti]{proporzione di vegetazione giovane erosa rispetto alla vegetazione giovane presente in funzione dell'integrale dei livelli (Int livelli) sopra la soglia di \SI{2}{\m}; vengono considerati tutti i tratti tranne il~9 e il~23; la retta verde rappresenta la regressione lineare dei punti.}	
	\label{graph:giov-iote-4tr-buono-accorpato}
\end{figure}
%

\subsection{Considerazioni sulle relazioni piene - erosione}
Le seguenti considerazioni si basano sull'osservazione dei risultati ottenuti con le diverse trasformazioni dei parametri (sezione~\ref{sec:metodologie-piene-erosione}).
 
Esiste un trend lineare abbastanza forte che lega l'erosione della vegetazione giovane con le piene che avvengono frequentemente (TR \SI{< 1}{\anno}), come è mostrato dal considerevole numero di tratti che presentano relazioni valide.
Questi sono spesso quelli intermedi e vallivi, a volte anche quelli montani.
\\
Per la vegetazione intermedia, si sono trovati trend positivi per TR~\SIrange[range-phrase = {-}, range-units = single]{2}{3}{\mesi} e TR~\SI{1}{\anno}, in un caso anche per TR~\SIrange[range-phrase = {-}, range-units = single]{4}{5}{\mesi}; per tutti i set di trasformazioni, i tratti validi adiacenti sono sempre quelli intermedi e vallivi.
\\
I trend migliori che si sono ottenuti per la fascia matura della vegetazione sono per TR~\SIrange[range-phrase = {-}, range-units = single]{2}{3}{\mesi} e TR~\SI{1}{\anno}; i tratti che presentano queste buone relazioni sono quelli montani e quelli vallivi, quasi mai quelli intermedi.

Osservando le equazioni delle relazioni provenienti dal set selezionato come migliore (\cref{tab:giov-iote-4tr-buono}, \cref{tab:interm-iote-4tr-buono}, \cref{tab:mat-iote-4tr-buono}), è possibile notare che:
%
\begin{itemize}
	\item il coefficiente angolare aumenta sempre all'incrementare del tempo di ritorno considerato;
	questo significa che, a parità di integrale dei livelli, il tasso di erosione è maggiore considerando eventi di piena via via più intensi;
	il fattore dell'incremento della percentuale di erosione può arrivare a superare l'ordine di grandezza ($10\times$ confrontando per la fascia intermedia il TR minore di~\SIrange[range-phrase = {-}, range-units = single]{2}{3}{\mesi} con coefficiente~\num{0.0293} e il TR maggiore di~\SI{1}{\anno} con coefficiente~\num{0.2819});
	\item a parità di tempo di ritorno, il coefficiente angolare diminuisce considerando classi d'età più vecchie; ad esempio nei i tratti 17$\div$20 per TR~\SIrange[range-phrase = {-}, range-units = single]{2}{3}{\mesi} il coefficiente della fascia giovane è~\num{0.0564}, della fascia intermedia è~\num{0.0293}, della fascia matura è~\num{0.0220};
	si ricorda che considerare gruppi diversi può portare a confronti non corretti poiché ogni gruppo ha caratteristiche peculiari;
	il fattore del decremento del tasso di vegetazione erosa è circa $2 \times$.
\end{itemize}
%
Queste osservazioni sono in linea con ciò che si poteva pensare dalla statistica dei tempi di ritorno dei picchi superiori ad \SI{1}{\m} (grafico in \cref{graph:tr-picchi}) e dall'analisi dei tassi di erosione nelle tre classi d'età (grafici in \cref{graph:erosione-classi-eta-matrix}), sebbene ora sia possibile fornire dati quantitativi:
le piene più intense hanno effetti maggiori sulle isole rispetto a piene meno intense di circa un ordine di grandezza per tempi di ritorno inferiori o pari ad~\SI{1}{\anno};
la vegetazione matura è molto più resistente rispetto a quella giovane, dato che i tassi di erosione sono circa dimezzati a parità di piena, mentre rispetto alla fascia intermedia la classe matura mostra una resistenza lievemente maggiore.

La normalizzazione dell'integrale dei livelli rispetto ad una quantità temporale (durata delle piene sopra soglia in un confronto o durata di un confronto) ha mostrato di non fornire risultati particolarmente buoni per il basso numero di tratti in cui ci sono relazioni valide. Questo può trovare una spiegazione nell'aver utilizzato dati idrometrici di livello, che sono una misura rispetto ad un riferimento locale che si presta meno a subire trasformazioni rispetto alla portata, la quale, d'altra parte, è più difficile da quantificare.
\\
L'utilizzo di tassi di erosione rispetto alle isole della stessa fascia d'età presenti all'inizio di ogni confronto, anziché considerare gli areali, rende i risultati di gruppi diversi di tratti e di diverse classi d'età più confrontabili tra di loro; mostra inoltre di restituire risultati migliori.
\\
Ricercare relazioni esponenziali non porta a risultati migliori rispetto alle regressioni lineari; nei grafici mostrati nella sezione~\ref{sec:migliori-tratti-piene-erosione} i punti non sembrano avere tendenze diverse da quella lineare (logaritmica, iperbolica, etc.).
\\
Per tempi di ritorno superiori ad~\SI{1}{\anno} non sono presenti trasformazioni dei parametri che diano risultati positivi; il motivo può risiedere nel fatto di avere pochi eventi che superano i~\SI{2.8}{\m} di livello e questo porta a non avere sufficienti punti nelle regressioni.
