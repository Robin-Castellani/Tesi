\section{Risultati: potenza della corrente e isole}
Il grafico in \cref{graph:omega-perc-50} mostra la mediana temporale della potenza della corrente per ogni tratto, calcolata secondo la equazione~\eqref{eq:omega-finta}; si rappresenta la mediana poiché la variazione di $\Omega$ nel tempo è trascurabile (pochi punti percentuali).
Questo grafico riflette sia le caratteristiche morfologiche (pendenza e larghezza) che idrologiche (percentuale di area drenante).
%
\begin{figure}
	\centering
	\tikzsetnextfilename{omega_perc_50}
\begin{tikzpicture}
	\begin{axis}[
		width = 0.95\textwidth,
		height = .5\textwidth,
		enlarge x limits = 0.01,
		%enlarge y limits = 0.01,
		xlabel = {Tratto},
		ylabel = {Potenza della corrente \si{[\newton\per\metre\tothe{4}]}},
		xtick = data,
		grid = major,
		yticklabel style = {
			/pgf/number format/fixed
		},
		]
		\addplot[only marks, mark = *]
        	table [x = tratto, y = omega,] {graphics/data/omega_perc_50.txt};
	\end{axis}
\end{tikzpicture}
	\caption[potenza della corrente in ogni tratto]{mediana temporale su tutte le immagini della potenza della corrente pesata con la relativa area drenante mostrata per ogni tratto.}
	\label{graph:omega-perc-50}
\end{figure}
%
\\
È possibile osservare come i tratti più a monte abbiano una \emph{stream power} elevata grazie alla forte pendenza e alla ridotta larghezza; a monte del tratto~3 confluisce il Fella, il maggiore affluente del Tagliamento, e il suo contributo in termini di area drenante è evidente.
Più a valle, dove l'alveo si allarga a la pendenza diminuisce, $\Omega$ cala; tuttavia a monte e a valle della stretta di Pinzano (tratti~12 e~13, rispettivamente) il brusco restringimento incrementa la potenza della corrente.
Nei tratti planiziali $\Omega$ non varia particolarmente, mentre nell'ultimo tratto, dove la morfologia diventa di tipo transizionale e l'alveo si restringe sensibilmente, il valore di $\Omega$ quasi triplica.

Rappresentando la potenza della corrente rispetto alla proporzione di isole sull'alveo attivo, si vede un andamento iperbolico (grafico in \cref{graph:omega-area-percentuale-linear}); utilizzando per entrambi gli assi una scala logaritmica (\cref{graph:omega-area-percentuale}), è possibile osservare l'effetto di controllo sulla quantità massima di vegetazione esercitato da $\Omega$.
\\
Da queste analisi è stato escluso il tratto~9, dove è presente l'isola di Cornino, poiché quest'isola si fonda su roccia e non è soggetta alle stesse dinamiche delle altre isole.
%
\begin{figure}
	\centering
	\tikzsetnextfilename{omega_area_percentuale_linear}
\begin{tikzpicture}
	\begin{axis}[
		width = \textwidth,
		height = .7\textwidth,
		enlarge x limits = 0.01,
		%enlarge y limits = 0.01,
		ylabel = {$\Omega$ \si{[\newton\per\metre\tothe{4}]}},
		xlabel = {Isole rispetto all'alveo attivo},
		xtick distance = 0.04,
		grid = major,
		legend columns = -1,
		legend style = {
			anchor = south,
			at = {(0.5, 1.01)},
		},
		ticklabel style = {
			/pgf/number format/.cd,fixed,
		},
		]
		\foreach \tratto in {1,2,...,23}
			{
			\addplot[
				only marks,
				forget plot,
			]
				table [y = om_tr_\tratto, x = area_tr_\tratto]
				{graphics/data/omega_area_percentuale.txt};
			}
		
%		\addplot [color = green, % T1
%			 line width = 2 pt,
%			 domain = 1e-3:4e-1,
%			 samples = 10,
%			 ] 
%			 {10^(-1.5184 - 0.1838 * log10(x))};
%		\addlegendentry{Fit 1}
%%		\node [fill = white, draw = green, anchor = east] % T1 
%%        	at (axis description cs: 1,0.6) {$y = 10^{-1.5184} \, x^{- 0.1838}$};
%        	
%		\addplot [color = orange, % T2
%			 line width = 2 pt,
%			 domain = 1e-3:4e-1,
%			 samples = 10,
%			 ]
%			 {10^(-1.5056 - 0.2516 * log10(x))};
%		\addlegendentry{Fit 2}
%%		\node [fill = white, draw = orange, anchor = east] % T2 
%%        	at (axis description cs: 1,0.75) {$y = 10^{-1.5056} \, x^{- 0.2516}$};
%        	
%		\addplot [color = cyan, % T3
%			 line width = 2 pt,
%			 domain = 1e-3:4e-1,
%			 samples = 10,
%			 ]
%			 {10^(-1.4085 - 0.2371 * log10(x))};
%		\addlegendentry{Fit 3}
%		\node [fill = white, draw = cyan, anchor = east] % T3 
%        	at (axis description cs: 1,0.9) {$y = 10^{-1.4085} \, x^{- 0.2371}$};
%        
%		\node [fill = white, draw = black, anchor = south west] % T3 
%        	at (axis description cs: 0,0) {$R^2 \in [0.3, 0.6]$ $P_\mathrm{val} < 0.0001$};
	\end{axis}
\end{tikzpicture}
	\caption[potenza della corrente rispetto alla proporzione di isole sull'alveo attivo, grafico lineare]{potenza della corrente rispetto alla proporzione di isole sull'alveo attivo; la scala degli assi è lineare.}
	\label{graph:omega-area-percentuale-linear}
\end{figure}
%
\begin{figure}
	\centering
	\tikzsetnextfilename{omega_area_percentuale}
\begin{tikzpicture}
	\begin{axis}[
		width = \textwidth,
		height = .5\textwidth,
		%enlarge x limits = 0.01,
		%enlarge y limits = 0.01,
		ylabel = {$\Omega$ \si{[\newton\per\metre\tothe{4}]}},
		xlabel = {Isole rispetto all'alveo attivo},
		grid = major,
		legend columns = -1,
		legend style = {
			anchor = south,
			at = {(0.5, 1.01)},
		},
		colormap = {fitting point colormap}{
				color = (black)
				color = (white!80!black)
%				color = (cyan!75!black)
%				color = (orange!75!black)
%				color = (green!75!black)
            },
        log ticks with fixed point,
        xmode = log,
        ymode = log,
        log basis x = 10,
        log basis y = 10,
		]
		\foreach \tratto in {1,2,...,23}
			{
			\addplot[
				scatter,
				only marks,
				point meta = {ifthenelse(y < -1.5184-0.1838*x, 1, ifthenelse(y < -1.5056 - 0.2516 * x, 0.7, 0))},
				point meta max = 1,
				point meta min = 0,
				forget plot,
			]
				table [y = om_tr_\tratto, x = area_tr_\tratto]
				{graphics/data/omega_area_percentuale.txt};
			}
		
		\addplot [color = green, % T1
			 line width = 2 pt,
			 domain = 1e-3:4e-1,
			 samples = 10,
			 ] 
			 {10^(-1.5184 - 0.1838 * log10(x))};
		\addlegendentry{Fit 1}
%		\node [fill = white, draw = green, anchor = east] % T1 
%        	at (axis description cs: 1,0.6) {$y = 10^{-1.5184} \, x^{- 0.1838}$};
        	
		\addplot [color = orange, % T2
			 line width = 2 pt,
			 domain = 1e-3:4e-1,
			 samples = 10,
			 ]
			 {10^(-1.5056 - 0.2516 * log10(x))};
		\addlegendentry{Fit 2}
%		\node [fill = white, draw = orange, anchor = east] % T2 
%        	at (axis description cs: 1,0.75) {$y = 10^{-1.5056} \, x^{- 0.2516}$};
        	
		\addplot [color = cyan, % T3
			 line width = 2 pt,
			 domain = 1e-3:4e-1,
			 samples = 10,
			 ]
			 {10^(-1.4085 - 0.2371 * log10(x))};
		\addlegendentry{Fit 3}
		\node [fill = white, draw = cyan, anchor = east] % T3 
        	at (axis description cs: 1,0.9) {$y = 10^{-1.4085} \, x^{- 0.2371}$};
        
		\node [fill = white, draw = black, anchor = south west] % T3 
        	at (axis description cs: 0,0) {$R^2 \in [0.3, 0.6]$ $P_\mathrm{val} < 0.0001$};
	\end{axis}
\end{tikzpicture}
	\caption[potenza della corrente rispetto alla proporzione di isole sull'alveo attivo, grafici bilogaritmici]{potenza della corrente rispetto alla proporzione di isole sull'alveo attivo con rette di regressione; nel grafico in alto ogni fit successivo considera solo i punti posti al di sopra del fit precedente; in quello in basso sono state utilizzate solo le massime percentuali di isole di ogni tratto nelle regressioni lineari; la scala di entrambi gli assi è logaritmica in base~\num{10}.}
	\label{graph:omega-area-percentuale}
\end{figure}
%
%\begin{figure}
%	\centering
%	\tikzsetnextfilename{omega_area_pura}
\begin{tikzpicture}
	\begin{axis}[
		width = \textwidth,
		height = .5\textwidth,
		%enlarge x limits = 0.01,
		%enlarge y limits = 0.01,
		ylabel = {$\Omega$ \si{[\newton\per\metre\tothe{4}]}},
		xlabel = {Isole rispetto all'alveo attivo \si{[\percent]}},
		xmode = log,
		ymode = log,
		grid = major,
		legend entries = {1,2,3,4,5,6,7,8,10,11,12,13,14,15,16,17,18,19,20,21,22,23},
		legend columns = 15,
		legend style = {
			anchor = south,
			at = {(0.5, 1.01)},
		},
		]
		\foreach \tratto in {1,2,...,23}
			{
			\addplot+[only marks]
				table [y = om_tr_\tratto, x = area_tr_\tratto]
				{graphics/data/omega_area_pura.txt};
			}
		
		\addplot [color = green, % T1
			 line width = 2 pt,
			 domain = 1e3:1e6,
			 samples = 10,
			 ] 
			 {10^(-0.3741 - 0.1820 * log10(x))};
		\node [fill = white, draw = cyan, anchor = east] % T1 
        	at (axis description cs: 1,0.9) {$10^{-0.3741} \, x^{- 0.1820}$};
        	
		\addplot [color = orange, % T2
			 line width = 2 pt,
			 domain = 1e3:1e6,
			 samples = 10,
			 ]
			 {10^(-0.0157 - 0.2366 * log10(x))};
		\node [fill = white, draw = orange, anchor = east] % T2 
        	at (axis description cs: 1,0.75) {$10^{-0.0157} \, x^{- 0.2366}$};
        	
		\addplot [color = cyan, % T3
			 line width = 2 pt,
			 domain = 1e3:1e6,
			 samples = 10,
			 ]
			 {10^(0.1130 - 0.2507 * log10(x))};
		\node [fill = white, draw = green, anchor = east] % T3 
        	at (axis description cs: 1,0.6) {$10^{0.1130} \, x^{- 0.2507}$};
        	
        \node [fill = white, draw = black, anchor = south west] % T3 
        	at (axis description cs: 0,0) {$R^2 \in [0.4, 0.8]$ $P_\mathrm{val} < 0.0001$};
	\end{axis}
\end{tikzpicture}
%	\caption[potenza della corrente rispetto all'areale delle isole]{potenza della corrente rispetto all'areale delle isole.}
%	\label{graph:omega-area-pura}
%\end{figure}
%
\\
Per ottenere relazioni che indichino quale sia il limite massimo di isole, sono state fatte due distinte regressioni lineari.
\\
Nella prima si è proceduto tramite regressioni lineari in successione: da una prima regressione su tutti i punti si sono selezionati solo i punti al di sopra della retta; si è eseguita una nuova regressione; con i punti posti superiormente alla seconda retta, si è ottenuta la retta finale di regressione.
Questa terza retta mostra un $R^2 \simeq 0.6$ ed un $P_\mathrm{value}$, ottenuto tramite il test statistico di Pearson, minore di \num{0.0001} (valori di $R^2$ quanto più vicini all'unità e maggiori di~\num{0} indicano una buona regressione lineare; valori di $P_\mathrm{value}$ inferiori a~\num{0.05} mostrano che la regressione è affidabile nel risultato, sia che $R^2$ sia prossimo ad~\num{1} o sia quasi nullo).
\\
Nella seconda regressione sono state approssimate le mediane su tutte le immagini di $\Omega$ di ogni tratto in funzione delle massime percentuali di isole in ogni tratto. Tenendo conto che si è escluso il tratto~9, si effettua la regressione su~22 punti: in ascissa è rappresentato il valore di percentuale massima di isole per ognuno dei~22 tratti rimanenti e in ordinata il valore di $\Omega$.
Questa regressione è caratterizzata da $R^2 = 0.74$ e da $P_\mathrm{val} < 0.0001$.
\\
Data la discreta bontà di queste regressioni, le si accetta come valide.

\section{Discussione: formalizzare un modello concettuale}
In letteratura sono stati proposti modelli concettuali che relazionano la potenza della corrente con la quantità di vegetazione e il tipo di forma vegetata presente:
a bassi livelli di $\Omega$, il disturbo indotto dalla corrente è modesto e si possono formare numerose isole sulle barre nude in alveo;
se la potenza della corrente, cioè il disturbo, è maggiore, si potranno formare meno isole e le forme fluviali rimarranno prevalentemente nude \squarecite{Gurnell:2014-plants-eng}.
\\
Difatti l'espansione maggiore della vegetazione e l'erosione minore hanno luogo dove il tasso di crescita della vegetazione è elevato e dove l'energia della corrente è ridotta \squarecite{Gurnell:2006-omega}.
\\
Inoltre le piante che crescono più rapidamente sembrano essere maggiormente flessibili \squarecite{Bertoldi:2011-ASTER}: la ricrescita vegetativa da tronchi vivi, la quale permette un rapido sviluppo, è proprio la caratteristica fondamentale delle piante che abitano questo ambiente, come già esposto nella sezione~\ref{sec:descr-area-studio}.
Dunque, è lecito supporre che proprio nei tratti dove si osserva una ridotta potenza della corrente si possa trovare un'elevata quantità di isole nella maggior parte del periodo di studio.

Le due regressioni ottenute ricalcano bene il profilo iperbolico che mostravano i punti quando posti in un grafico in scala lineare (\cref{graph:omega-area-percentuale-linear-modello}); il terzo fit si pone al di sopra dei punti per percentuali di isole superiori al~\SI{5}{\percent}, mentre si adatta poco ai punti nel caso di percentuali inferiori; la regressione sulle massime percentuali sembra avere un comportamento generalmente migliore.
%
\begin{figure}
	\centering
	\tikzsetnextfilename{omega_area_percentuale_linear_regressioni}
\begin{tikzpicture}
	\begin{axis}[
		width = 0.97\textwidth,
		height = 0.97\textwidth,
		enlarge x limits = 0.01,
		%enlarge y limits = 0.01,
		ylabel = {$\Omega$ \si{[\newton\per\metre\tothe{4}]}},
		xlabel = {Isole rispetto all'alveo attivo},
		xtick distance = 0.04,
		enlargelimits = 0.02,
		grid = major,
		legend columns = -1,
		legend style = {
			anchor = south,
			at = {(0.5, 1.01)},
		},
		ticklabel style = {
			/pgf/number format/.cd,fixed,
		},
		]
		\foreach \tratto in {1,2,...,23}
			{
			\addplot[
				only marks,
				forget plot,
			]
				table [y = om_tr_\tratto, x = area_tr_\tratto]
				{graphics/data/omega_area_percentuale.txt};
			}
		
		\addplot [color = cyan, % modello
			 line width = 2 pt,
			 samples at = {0.01,0.012,...,0.32},
			 %domain y = 0.5:0.125,
			 %samples = 10,
			 ] 
			 {10 ^ (-1.4085) * x ^ (-0.2371)};
		%\addlegendentry{Fit 3}
%		\node [fill = white, draw = green, anchor = east] % T1 
%        	at (axis description cs: 1,0.6) {$y = 10^{-1.5184} \, x^{- 0.1838}$};
		
		\addplot [color = red, % modello
			 line width = 2 pt,
			 samples at = {0.01,0.012,...,0.32},
			 %domain y = 0.5:0.125,
			 %samples = 10,
			 ] 
			 {10 ^ (-1.6806) * x ^ (-0.4476)};
		%\addlegendentry{Max}
	\end{axis}
\end{tikzpicture}
	\caption[relazioni iperboliche tra potenza della corrente e percentuale di isole]{relazioni iperboliche tra potenza della corrente e percentuale di isole; in rosso la regressione sulle massime percentuali di isole rispetto all'alveo attivo, in azzurro la regressione sul terzo gruppo di punti di cui al grafico in \cref{graph:omega-area-percentuale}; la scala degli assi è lineare.}
	\label{graph:omega-area-percentuale-linear-regressioni}
\end{figure}
%
\\
Si sceglie quindi quest'ultima come rappresentativa del limite superiore di isole che l'alveo può supportare data una potenza della corrente (cioè una pendenza, una larghezza e un'area drenante del bacino, secondo la sua definizione).

In un precedente lavoro è stato definito un unico valore limite di \emph{stream power} oltre il quale le isole non riescono più ad insediarsi a causa dell'intenso disturbo \squarecite{Gurnell:2006-omega}.
Tuttavia, questo valore è stato calcolato utilizzando la relazione empirica~\eqref{eq:area-portata-mosetti}; per quanto già esposto, si è preferito non utilizzare tale relazione.
\\
I grafici mostravano un andamento iperbolico, il quale regola la percentuale massima di isole che riesce a stabilirsi con una data potenza;
questo andamento sembra fermarsi superiormente, attorno ad un valore di $\Omega$ di~\SIrange[range-phrase={-}, range-units = single]{0.12}{0.13}{\newton\per\metre\tothe{4}}, oltre il quale non è più presente vegetazione.
Si vede inoltre che l'alveo non ospita percentuali di vegetazione superiori al \SIrange[range-phrase={-}, range-units = single]{30}{35}{\percent}.

La relazione ottenuta presenta un limite implicito: $\Omega$ non tiene conto della connessione con la falda, dell'\emph{upwelling} e del \emph{downwelling}, che influenzano notevolmente la crescita delle piante.
Anzi, è stato mostrato che le isole complesse sono confinate nei tratti dove la crescita delle piante da seme può essere sufficientemente rapida (\SIrange[range-phrase={-}, range-units = single]{1}{3}{\m} in \SI{10}{\anni}; si ricorda che tramite la riproduzione vegetativa la crescita è più rapida di un ordine di grandezza); questi tratti sono quelli relativamente più stretti, dove c'è disponibilità d'acqua durante i periodi di magra grazie alla falda non troppo profonda, come i tratti pochi chilometri a monte della stretta di Pinzano o quelli a monte della zona di cambiamento di morfologia fluviale \squarecite{Gurnell:2006-omega}.
Dall'altra parte, come gli autori osservano e come è verificato dai risultati appena mostrati, dove i tratti si restringono maggiormente la potenza della corrente è tanto grande che nemmeno l'incrementato tasso di crescita delle piante è tale da permettere alle isole di insediarsi prima di essere portate via.
Gli autori suggeriscono quindi l'esistenza di un equilibrio tra processi idrologici, piante riparie e sviluppo delle isole, che si concretizza nei seguenti aspetti (grafico in \cref{graph:omega-area-percentuale-linear-modello}):
%
\begin{itemize}
	\item un valore massimo di $\Omega$ pari a \SIrange[range-phrase={-}]{0.12}{0.13}{\newton\per\metre\tothe{4}}, oltre il quale non sono presenti isole;
	\item un range di $\Omega$ in cui è possibile l'insediamento di isole, ma caratterizzato da un limite in cui il disturbo delle piene è predominante anche nelle zone poste a quote relative più elevate dell'alveo;
	\item un limite massimo di isole, \numrange[range-phrase={-}]{0.30}{0.35}, che possono essere presenti anche con $\Omega$ molto bassi, in quanto oltre questo limite la crescita delle isole avrebbe luogo sulle zone dell'alveo poste a quote relativamente minori, che sono le più disturbate dalle piene, mentre la zone situate a quote relativamente maggiori (creste delle barre, altre isole) sono tutte già vegetate.
\end{itemize}
%
%
\begin{figure}
	\centering
	\tikzsetnextfilename{omega_area_percentuale_linear_modello}
\begin{tikzpicture}
	\begin{axis}[
		width = \textwidth,
		height = .7\textwidth,
		enlarge x limits = 0.05,
		%enlarge y limits = 0.01,
		ylabel = {$\Omega$ \si{[\newton\per\metre\tothe{4}]}},
		xlabel = {Isole rispetto all'alveo attivo},
		xtick distance = 0.04,
		xmax = 0.44,
		grid = major,
		legend columns = -1,
		legend style = {
			anchor = south,
			at = {(0.5, 1.01)},
		},
		ticklabel style = {
			/pgf/number format/.cd,fixed,
		},
		]
		\foreach \tratto in {1,2,...,23}
			{
			\addplot[
				only marks,
				forget plot,
				gray,
			]
				table [y = om_tr_\tratto, x = area_tr_\tratto]
				{graphics/data/omega_area_percentuale.txt};
			}
		
		\addplot [color = purple, % limite superiore
			 line width = 2 pt,
			 ] 
			 coordinates {(-0.01,0.13) (0.02,0.13)};
		\node [fill = white, draw = purple, anchor = west] 
        	at (0.04,0.13) {$\Omega_\mathrm{max} \simeq\SI{0.13}{\newton\per\metre\tothe{4}}$};
		
		\addplot [color = red, % iperbole
			 line width = 2 pt,
			 samples at = {0.018,0.019,...,0.325},
			 ] 
			 {10 ^ (-1.6806) * x ^ (-0.4476)};
		\node [fill = white, draw = red, anchor = south west] 
        	at (0.08,0.08) {$y = 0.021 \, x ^ {-0.4476}$};
		
		\addplot [color = teal, % limite superiore
			 line width = 2 pt,
			 dashed,
			 ] 
			 coordinates {(0.35,0.02) (0.35,0.035)};
		\node [fill = white, draw = teal, anchor = east] 
        	at (0.42,0.06) {Proporzione max di isole $\simeq \num{0.35}$};
	\end{axis}
\end{tikzpicture}
	\caption[modello che lega la potenza della corrente con la percentuale massima di isole]{modello che lega la potenza della corrente $\Omega$ con la percentuale massima di isole; sono individuati in rosso scuro il limite di $\Omega$ oltre il quale non sono presenti isole, in rosso la relazione iperbolica tra $\Omega$ e la massima proporzione di isole osservata in ogni tratto, in verde-blu il massimo rapporto di isole rispetto all'alveo attivo osservato.}
	\label{graph:omega-area-percentuale-linear-modello}
\end{figure}
%
