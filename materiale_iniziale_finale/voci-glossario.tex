% voci del glossario

\newglossaryentry{ortofoto}{name={Ortofoto},description={immagine aerea corretta affinché non vi siano effetti di distorsione dovuti all'inclinazione del sensore o dei rilievi topografici; osservare un punto su una ortofoto è come osservarlo dal suo zenit; è possibile misurarvi distanze vere},see={imm-sat}}

\newglossaryentry{imm-sat}{name={Immagine satellitare},description={indicata semplicemente come “immagine”, viene distinta dalla ortofoto poiché il sensore è montato su satellite e, nel presente lavoro, include solitamente ulteriori bande oltre al visibile},see={ortofoto}}

\newglossaryentry{rgb}{name={R-G-B},description={\emph{Red}, \emph{Green}, \emph{Blue}, le tre bande in cui solitamente si suddividono tutte le immagini; ciascuna banda indica la quantità di colore presente in ogni punto}}

\newglossaryentry{floodplain}{name={\emph{Floodplain}},sort={Floodplain},description={piana alluvionale, porzione dell'alveo posta lateralmente rispetto all'alveo attivo, solitamente vegetata; viene inondata solamente in caso di eventi estremi (tempo di ritorno superiore a \SI{10}{\anni}); in base ai fenomeni di erosione delle sponde può essere modificata nel tempo}}

\newglossaryentry{alveo-att}{name={Alveo attivo},description={porzione dell'alveo che ospita i canali attivi (dove scorre l'acqua) e le forme morfologiche non vegetate (ad esempio barre in ghiaia); può cambiare nel corso degli anni}}

\newglossaryentry{sponda}{name={Sponda},description={parte della \emph{floodplain} più prossima all'alveo attivo e soggetta a dinamiche di erosione durante gli eventi di piena e dinamiche di accrescimento quando le isole si espandono fino a diventare parte della sponda stessa}}

\newglossaryentry{bankfull}{name={Evento \emph{bankfull}},description={evento di piena “a piene rive”, cioè che riempie pienamente l'alveo attivo fino alle sponde; il tempo di ritorno di tali eventi è di \SIrange[range-phrase={-}, range-units=single]{1}{2}{\anni}}}

\newglossaryentry{ndvi}{name={NDVI},description={\emph{Normalized Difference Vegetation Index}, indice compreso tra \numrange[range-phrase={ e }]{-1}{1} dato dal rapporto $\frac{IR-R}{IR+R}$; valori particolarmente alti sono associati con vegetazione viva}}

\newglossaryentry{r}{name={R},description={banda del Rosso, con lunghezza d'onda compresa circa tra \SIrange[range-phrase={ e }]{0.6}{0.7}{\micro\m}}}

\newglossaryentry{ir}{name={IR},description={banda dell'InfraRosso vicino, con lunghezza d'onda compresa circa tra \SIrange[range-phrase={ e }]{0.75}{0.95}{\micro\m}}}

\newglossaryentry{banda}{name={Banda},description={radiazione elettromagnetica compresa in un determinato intervallo di lunghezze d'onda}}

\newglossaryentry{isola}{name={Isola},description={zona più o meno densamente vegetata completamente circondata dall'alveo attivo, priva di alcuna connessione con la \emph{floodplain}; è soggetta a specifiche dinamiche di erosione laterale, durante piene sufficientemente intense, e di accrescimento a partire da tronchi depositati nell'alveo attivo così come da piante nate da seme}}

\newglossaryentry{large-wood}{name={\emph{Large wood}},sort={Large wood},description={legno di grandi dimensioni, indica i tronchi di almeno \SI{20}{\centi\m} di diametro e \SI{1}{\m} di lunghezza che sono sradicati durante gli eventi di piena sufficientemente intensi e che sono in seguito depositati poco più a valle nell'alveo attivo; questi tronchi possono successivamente rigettare rami e radici e dare vita a nuove piante; si distingue dalla vegetazione viva poiché è stato eroso da isole o sponde}}

\newglossaryentry{wood-jam}{name={\emph{Wood jam}},sort={Wood jam},description={accumulo di legno, insieme di numerosi \emph{large wood} e di frammenti più piccoli che possono formare accumuli a ridosso di piante vive nell'alveo attivo, su forme morfologiche o lateralmente a sponde e isole}}

\newglossaryentry{risoluzione}{name={Risoluzione},description={immagini, ortofoto e mappe digitali sono formate da un insieme di celle quadrate, ognuna con un valore; il numero di celle costituisce la risoluzione. Tuttavia, se si riduce il lato delle celle aumentandone il numero, la risoluzione non aumenta poiché l'informazione contenuta nella cella originale ha significato e validità solamente se la lunghezza del suo lato non cambia. È invece  possibile e corretto ridurre la risoluzione tramite un aumento del lato delle celle, una contemporanea diminuzione del numero e una distribuzione dell'informazione originale nelle nuove celle}}

\newglossaryentry{percentile}{name={Percentile},description={dato un insieme ordinato di valori, l'$n$-esimo percentile $p_n$ (con $n \in [0,1]$) della popolazione è il valore $p_n$ che divide la popolazione in due parti proporzionali rispettivamente a $n$ e a $1-n$; ognuna di queste due parti ha rispettivamente valori minori di $p_n$ e maggiori di $p_n$}}

\newglossaryentry{csm-dem-lidar}{name={CHM, DEM, LiDAR},description={un rilievo aereo LiDAR (\emph{Airborne Light Detection and Ranging}) produce dati dai quali solitamente si ottiene un DSM (\emph{Digital Surface Model}), che è una mappa con le quote del terreno e degli oggetti che vi sono sopra (vegetazione, edifici, artefatti, \ldots); è possibile estrarre anche un DEM (\emph{Digital Elevation Model}, chiamato anche DTM, \emph{Digital Terrain Model}), che è una mappa “ripulita” dagli elementi soprastanti, che rappresenta solamente le quote del terreno; infine, un CHM (\emph{Canopy Height Model}) non è altro che l'altezza della vegetazione sopra il suolo ottenuta come differenza tra un DSM e un DEM}}

\newglossaryentry{flow-pulse}{name={\emph{Flow pulse}},sort={Flow pulse},description={piena di media entità, cioè meno intensa degli eventi \emph{bankfull} e quindi con livelli d'acqua che non arrivano ad inondare le sponde e le isole poste a quote relative più alte}}

\newglossaryentry{stream-power}{name={\emph{Stream power}},sort={Stream power},description={la potenza della corrente $\Omega$ per unità di larghezza e di profondità si misura in \si{[\watt\per\metre\tothe{2}]}; è direttamente proporzionale alla portata transitante $Q$ \si{[\metre\tothe{3}\per\second]}, alla pendenza del tratto $i_f$ \si{[\m\per\m]} e al peso specifico dell'acqua $\gamma$ \si{[\newton\per\metre\tothe{3}]}, mentre è inversamente proporzionale alla larghezza $B$ \si{[\m]}. In tratti stretti, a parità di portata la corrente ha una velocità maggiore e la \emph{stream power} è di conseguenza maggiore; alvei a bassa pendenza hanno una potenza della corrente minore rispetto ad alvei con la stessa larghezza e in cui transita la stessa portata}}

\newglossaryentry{boxplot}{name={\emph{Boxplot}},sort={Boxplot},description={il diagramma a scatola e baffi, qui chiamato per brevità diagramma a scatola o \emph{boxplot}, è un metodo grafico di rappresentazione della distribuzione di dati numerici attraverso i percentili; il limite inferiore e il limite superiore della scatola rappresentano rispettivamente il $25_\mathrm{mo}$ e il $75_\mathrm{mo}$ percentile; la linea nella scatola rappresenta il $50_\mathrm{mo}$ percentile; il baffo inferiore e il baffo superiore rappresentano il $10_\mathrm{mo}$ e il $90_\mathrm{mo}$ percentile, rispettivamente}}

\newglossaryentry{disturbo}{name={Disturbo},description={consiste in qualunque evento, ben distinguibile nel tempo, che modifica le risorse e l'ambiente fisico dove vivono degli organismi e che altera la struttura di popolazioni, comunità o ecosistemi}}

\newglossaryentry{gis}{name={GIS},description={\emph{Geographic Information System}, o Sistema Informativo Territoriale, è uno strumento informatico per gestire (acquisire, elaborare, trasmettere, archiviare,\ldots) dati che posseggono una componente spaziale, cioè che sono collegati a coordinate}}

\newglossaryentry{r2}{name={$\mathbf{R^2}$},sort={R2},description={il coefficiente di determinazione è un numero compreso tra~0 ed~1; indica la proporzione di varianza nella variabile dipendente (ed esempio~$y$) che è possibile prevedere a partire dalla variabile indipendente (ad esempio~$x$); in questa tesi viene utilizzato per caratterizzare la bontà di una regressione lineare: se $R^2 = 1$ allora le variabili~$x$ e~$y$ hanno una perfetta correlazione lineare; quanto più $R^2$ si avvicina al valore nullo, tanto meno la correlazione lineare è forte}}

\newglossaryentry{pearson}{name={Coefficiente di correlazione di Pearson e $\mathbf{P_\mathrm{value}}$},description={un'alternativa all'uso del coefficiente $R^2$ per valutare la correlazione lineare tra due variabili, ad esempio~$x$ e~$y$, è utilizzare il coefficiente di correlazione di Pearson, che si interpreta nello stesso modo dell'$R
^2$. Per ottenere il $P_\mathrm{value}$ si permutano i valori di una variabile, ad esempio~$y$, mantenendo invariati i valori dell'altra variabile; si ricalcola il coefficiente di correlazione di Pearson per ogni permutazione; il $P_\mathrm{value}$ rappresenta la proporzione dei casi in cui, permutando, si è ottenuto un valore del coefficiente di correlazione di Pearson maggiore o uguale a quello ottenuto senza permutazioni; solitamente una regressione si considera affidabile se $P_\mathrm{value} < 0.05$}}

\newglossaryentry{tempo_ritorno}{name={Tempo di ritorno},description={di un evento è il periodo di tempo che mediamente intercorre tra l'accadimento dell'evento e di un altro evento di intensità pari o superiore; il tempo di ritorno è pari all'inverso della probabilità di superamento, cioè della probabilità di osservare altri eventi con intensità pari o superiore a quella dell'evento}}
