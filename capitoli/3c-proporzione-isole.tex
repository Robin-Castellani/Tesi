\section{Risultati e discussione: percentuale di isole in alveo}
\label{sec:perc-isole-in-alveo}
Grazie alle mappe di classificazione dell'alveo è possibile ottenere informazioni riguardo la quantità di isole presenti nei tratti rispetto all'alveo attivo.
Questa proporzione è utile per delineare zone in cui la vegetazione riesce a prosperare, o per individuare momenti in cui le isole si sono espanse, così come per verificare come la connessione con la falda nel materasso alluvionale influenzi la crescita delle piante.
\\
Il grafico in \cref{graph:rapp-isl-tutti-tratti} mostra il rapporto tra l'area delle isole presenti in ogni tratto valido in ogni immagine e il relativo areale di alveo attivo.
La sua lettura è analoga al grafico in \cref{graph:larghezze-tutti-tratti}: le “strisce” verticali sono delle fotografie di un preciso momento in cui l'alveo presentava quella specifica proporzione di isole; le “strisce” orizzontali sono le traiettorie evolutive dei singoli tratti.
%
\begin{figure}
	\centering
	\tikzsetnextfilename{rapp_isl_tutti_tratti}
\begin{tikzpicture}
	\begin{axis}[
		width = \textwidth,
		height = \textwidth,
		symbolic x coords = {2000-09-17, 2001-06-07, 2002-05-18, 2002-06-12, 2003-06-22, 2004-10-14, 2005-08-30, 2006-07-16, 2007-09-21, 2008-07-05, 2009-07-08, 2010-09-29, 2011-10-02, 2012-08-01, 2013-09-05, 2014-09-08, 2014-10-31, 2015-08-13, 2015-09-12, 2015-10-22, 2016-09-13, 2017-04-21, 2017-06-13, 2018-06-15, 2018-09-16},
		xticklabel style = {
			rotate = 90,
			%anchor = near yticklabel,
		},
		xtick distance = 1,
		ymin = 1,
		ymax = 23,
		%yticklabel style = {font=\footnotesize},
		ytick = data,
		enlarge x limits = 0,
		enlarge y limits = 0,
		ylabel = {Tratto},
		y dir = reverse,
		colorbar horizontal,
		colorbar style = {
			xlabel = {Proporzione di isole sull'alveo attivo},
			xticklabel style = {/pgf/number format/.cd,fixed},
			xtick distance = 0.04,
		},
		]
		\addplot[
			matrix plot*,
			mesh/cols = 23,	% per fargli leggere colonne formate da 23 righe dal file di testo
			shader = flat corner,	% per interpolare i colori
		]
        	table [y = tratto, x = data, point meta = \thisrow{rapp_isl}] {graphics/data/rapp_isl_tutti_tratti.txt};
	\end{axis}
\end{tikzpicture}

	\caption[rapporto delle isole sull'alveo attivo in ogni tratto per ogni immagine]{rapporto delle isole sull'alveo attivo in ogni tratto per ogni immagine; i quadrati bianchi indicano assenza di dati (a causa della presenza di nuvole o limitata estensione dell'immagine); è stato escluso il tratto~9, che contiene l'isola di Cornino che si fonda su roccia e non è soggetta alle stesse dinamiche delle altre isole.}
	\label{graph:rapp-isl-tutti-tratti}
\end{figure}
%
\\
Da monte verso valle, si vede come alcuni tratti siano più adatti di altri ad ospitare delle isole, come i tratti dal~7 all'11 o quelli planiziali;
i tratti montani e quelli più vallivi sono i meno vegetati, sicuramente poiché sono stretti e il disturbo indotto dalle piene è frequente ed intenso.
È da ricordare che nel tratto~9 è presente l'isola di Cornino, che a differenza delle altre isole si fonda su roccia anziché su ghiaia.
\\
Si osserva inoltre l'effetto della risalita e dello sprofondamento della falda:
i tratti con \emph{upwelling}, come quelli tra il~7 e l'11 e quelli tra il~19 e il~22, sono quelli che possono essere colonizzati da molte isole se le altre condizioni ambientali sono favorevoli; i tratti con \emph{downwelling} sono invece generalmente meno vegetati, a meno di grandi isole che si sono in seguito fuse con le sponde. Queste isole si sono formate negli ultimi due decenni del 1900, quando le condizioni erano molto probabilmente favorevoli per la crescita, l'espansione e la coalescenza di più isole.
Questi trend spaziali sono mostrati nel grafico in \cref{graph:rapp-isl-2008-2016}.
%
\begin{figure}
	\centering
	\tikzsetnextfilename{rapp_isl_2008_2016}
\begin{tikzpicture}
	\begin{axis}[
		width = 0.98\textwidth,
		height = 0.5\textwidth,
		xtick = data,
		enlarge x limits = 0.01,
		xlabel = {Tratto},
		ylabel = {Rapporto isole su alveo attivo},
		y tick label style = {
			/pgf/number format/.cd,
			fixed,
			fixed zerofill,
			precision = 1,
			/tikz/.cd,
		},
		grid = major,
		legend style = {
			at = {(0,1)},
			anchor = north west,		
		},
		]
		\addplot[
			violet,
			mark = triangle,
			unbounded coords = jump,
			]
        	table [x = tratto, y = rapp_isl_2008,] {graphics/data/rapp_isl_2008_2016.txt};
        \addlegendentry{2008-07-05};
        	
		\addplot[
			cyan,
			mark = triangle,
			unbounded coords = jump,
			]
        	table [x = tratto, y = rapp_isl_2016,] {graphics/data/rapp_isl_2008_2016.txt};
        \addlegendentry{2016-09-13};
	\end{axis}
\end{tikzpicture}

	\caption[proporzione di isole sull'alveo attivo nel 2008-07-05 e nel 2016-09-13]{proporzione di isole sull'alveo attivo nell'immagine \AST{} 2008-07-05 e nell'immagine \Se{} 2016-09-13; si notano le zone più vegetate proprio nei tratti dove c'è \emph{upwelling} e quelle meno vegetate dove c'è \emph{downwelling}; non è rappresentato il tratto~9 poiché contiene l'isola di Cornino, che si fonda su roccia.}
	\label{graph:rapp-isl-2008-2016}
\end{figure}
%

Si è proceduto confrontando parte dei dati ottenuti con i risultati riportati in due lavori \squarecites{Zanoni:2008}{Surian:2015} (\cref{graph:rapp-isl-vs-letteratura}).
Non sono presenti dati antecedenti al 1940, come invece erano stati riportati per la larghezza, poiché gli autori non hanno considerato affidabili le mappe del catasto austriaco nel definire i limiti delle isole.
Questi risultati sono limitati ai tratti compresi tra il ponte autostradale presso Braulins (tratto~6) e la stretta di Pinzano (tratto~12).
%
\begin{figure}
	\centering
	\tikzsetnextfilename{rapp_isl_vs_letteratura}
\begin{tikzpicture}
	\begin{axis}[
		width = 0.95\textwidth,
		height = 0.5\textwidth,
		date coordinates in = x,
		date ZERO = 1940-01-01,
		xticklabel = {$\year$},
		xticklabel style = {
			rotate = 80,
			anchor = near xticklabel
		},
		xtick distance = 3660,
		enlarge x limits = 0.02,
		ylabel = {Rapporto isole su alveo attivo},
		yticklabel style = {
			/pgf/number format/.cd,
			fixed,
			fixed zerofill,
			precision = 2,
			/tikz/.cd,
		},
		grid = major,
		legend style = {
			anchor = north west,
			at = {(0,1)},
		},
		scaled ticks = false,
		%legend columns = -1,
		]
		\addplot[green!70!black, mark = square*] % zanoni 2008
	        table [x = data, y = rapp_isl] {graphics/data/rapp_isl_zanoni.txt};
	        \addlegendentry{Zanoni et al. 2008}
	    \addplot[orange, mark = triangle*] % surian 2015
	        table [x = data, y = rapp_isl] {graphics/data/rapp_isl_surian.txt};
	     	\addlegendentry{Surian et al. 2015}
		\addplot[blue, mark = *] % mie
	       	table [x = data, y = rapp_isl] {graphics/data/rapp_isl_mie.txt};
	       	\addlegendentry{Presente lavoro}
	\end{axis}
\end{tikzpicture}

	\caption[rapporto tra isole e alveo attivo nell'area compresa tra il tratto~6 e il tratto~12]{rapporto tra isole e alveo attivo nell'area compresa tra il tratto~6 e il tratto~12; sono presenti i dati provenienti da \squarecite{Zanoni:2008} e da \squarecite{Surian:2015}.}
	\label{graph:rapp-isl-vs-letteratura}
\end{figure}
%
\\
Da una parte \squarecite{Zanoni:2008} suggerisce che la percentuale di vegetazione in alveo oscilla attorno al valore di \SI{8}{\percent}, dall'altra \squarecite{Surian:2015} mostra che la traiettoria evolutiva è assai simile a quella della larghezza.
I risultati di questa tesi sembrano seguire i risultati di entrambi: ad un primo periodo di crescita del rapporto isole su alveo seguono oscillazioni continue attorno ad un valore compreso tra \numrange[range-phrase={ e }]{0.12}{0.13}.
\\
Si vede chiaramente come nello stesso anno il rapporto tra isole e alveo attivo mostrato dai tre lavori considerati presenti valori molto diversi, come nel~1954, nel~2000, nel~2005, nel~2009 e nel~2011.
I dati della tesi si discostano abbastanza rapidamente dai dati da letteratura, in particolare dagli anni 2005-2006 quando in un periodo privo di piene intense la vegetazione ha potuto prosperare.
\\
Queste discordanze sono con tutta probabilità dovute ai differenti metodi che ogni autore ha utilizzato per delimitare le isole e l'alveo attivo, così come le mappe utilizzate:
nelle ortofoto ad alta risoluzione (celle di dimensione di poche decine di centimetri) è spesso possibile riconoscere la chioma di ogni pianta, mentre nelle immagini satellitari si può solo distinguere le forme vegetate dall'alveo;
\squarecite{Surian:2015} ha considerato come isole solamente le piante che vi crescono, senza tenere conto delle zone parzialmente vegetate o poco vegetate che possono esservi all'interno, ed ha escluso le piante “intermedie” e “alte” dall'areale dell'alveo attivo;
la presente tesi ha invece incluso le isole nel definire l'alveo attivo e la risoluzione minore della maggior parte delle immagini (cioè la maggior dimensione delle celle) ha permesso di delineare le isole come zone compatte di vegetazione, che raramente presentavano nel mezzo celle classificate come alveo o come canale.
In quanto le immagini satellitari ad alta risoluzione \Pl{} e \WV{} mostrano le stesse differenze di percentuale di isole su alveo attivo rispetto alle digitalizzazioni manuali, si ritiene che siano le scelte durante la fase di mappatura la fonte principale di scostamento tra i dati presentati.
\\
Le diversità non permettono un confronto dei risultati al fine di evincere un trend temporale.
Dai risultati della tesi sembra comunque certo che nei tratti in questione (dal~6 al~12) l'alveo abbia subito un incremento nella percentuale di isole, ma generalmente non oltre il \SI{15}{\percent}.

