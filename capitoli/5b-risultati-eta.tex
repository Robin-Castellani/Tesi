\section{Risultati: osservare l'erosione secondo l'età}
Nei grafici in \cref{graph:rapp-isl-classi-eta-matrix} sono riportate le proporzioni tra isole e alveo attivo suddivise in classi d'età per ogni tratto, tranne il~9 dove c'è l'isola rocciosa di Cornino, e per ogni immagine a partire dal 2007-09-21.
%
\begin{figure}
	\centering
	\tikzsetnextfilename{rapp_isl_classe_eta_matrix}
\begin{tikzpicture}
	\begin{axis}[
		width = 0.49\textwidth,
		height = 0.85\textwidth,
		name = giovane,
		title = {Giovane},
		symbolic x coords = {2007-09-21, 2008-07-05, 2009-07-08, 2010-09-29, 2011-10-02, 2012-08-01, 2013-09-05, 2014-09-08, 2014-10-31, 2015-08-13, 2015-09-12, 2015-10-22, 2016-09-13, 2017-04-21, 2017-06-13, 2018-06-15, 2018-09-16},
		xticklabel style = {
			rotate = 90,
			font = \footnotesize,
		},
		xtick distance = 1,
		ymin = 1,
		ymax = 23,
		ytick = data,
		enlargelimits = 0.02,
		ylabel = {Tratto},
		y dir = reverse,
		]
		\addplot[
			matrix plot*,
			mesh/cols = 23,	% per fargli leggere colonne formate da 23 righe dal file di testo
			shader = flat corner,	% per interpolare i colori
			point meta max = 0.3,
		]
        	table [y = tratto, x = data, point meta = \thisrow{cambiamento}] {graphics/data/rapp_isl_giov_matrix.txt}; 	
    \end{axis}
    
    \begin{axis}[
		width = 0.49\textwidth,
		height = 0.85\textwidth,
		name = intermedia,
		title = {Intermedia},
		at = {($(giovane.east) + (0.2cm, 0cm)$)},
		anchor = west,
		symbolic x coords = {2007-09-21, 2008-07-05, 2009-07-08, 2010-09-29, 2011-10-02, 2012-08-01, 2013-09-05, 2014-09-08, 2014-10-31, 2015-08-13, 2015-09-12, 2015-10-22, 2016-09-13, 2017-04-21, 2017-06-13, 2018-06-15, 2018-09-16},
		xticklabel style = {
			rotate = 90,
			font = \footnotesize,
		},
		xtick distance = 1,
		ymin = 1,
		ymax = 23,
		ytick = data,
		yticklabels = {,,},
		enlargelimits = 0.02,
		y dir = reverse,
		]
		\addplot[
			matrix plot*,
			mesh/cols = 23,	% per fargli leggere colonne formate da 23 righe dal file di testo
			shader = flat corner,	% per interpolare i colori
			point meta max = 0.3,
		]
        	table [y = tratto, x = data, point meta = \thisrow{cambiamento}] {graphics/data/rapp_isl_int_matrix.txt};
    \end{axis}
    
    \begin{axis}[
		width = 0.49\textwidth,
		height = 0.85\textwidth,
		name = matura,
		title = {Matura},
		at = {($(intermedia.east) + (0.2cm, 0cm)$)},
		anchor = west,
		symbolic x coords = {2007-09-21, 2008-07-05, 2009-07-08, 2010-09-29, 2011-10-02, 2012-08-01, 2013-09-05, 2014-09-08, 2014-10-31, 2015-08-13, 2015-09-12, 2015-10-22, 2016-09-13, 2017-04-21, 2017-06-13, 2018-06-15, 2018-09-16},
		xticklabel style = {
			rotate = 90,
			font = \footnotesize,
		},
		xtick distance = 1,
		ymin = 1,
		ymax = 23,
		ytick = data,
		yticklabels = {,,},
		enlargelimits = 0.02,
		y dir = reverse,
		colorbar right,
		colorbar style = {
			ylabel = {Isole rispetto all'alveo attivo},
			ylabel style = {
				rotate = 180,
			},
			yticklabel style = {
				/pgf/number format/.cd,fixed,
			},
		},
		]
		\addplot[
			matrix plot*,
			mesh/cols = 23,	% per fargli leggere colonne formate da 23 righe dal file di testo
			shader = flat corner,	% per interpolare i colori
			point meta max = 0.3,
		]
        	table [y = tratto, x = data, point meta = \thisrow{cambiamento}] {graphics/data/rapp_isl_mat_matrix.txt};
    \end{axis}
\end{tikzpicture}
	%\vspace*{-1cm}
	\caption[rapporto tra isole e alveo attivo suddiviso in classi d'età]{rapporto tra isole e alveo attivo suddiviso in classi d'età.
	Sono mostrate solamente le immagini a partire dal~2007; il tratto~9 non è stato considerato poiché contiene l'isola di Cornino, che si fonda su roccia e non può essere erosa; le celle bianche indicano mancanza di dati a causa della limitata estensione delle immagini o della presenza di nuvole.}
	\label{graph:rapp-isl-classi-eta-matrix}
\end{figure}
%
\\
Nella classe giovane si vede la grande quantità di isole che sono cresciute in gran parte del tratto di studio nel periodo 2005/2008 privo di piene; oltre a questi anni, nel 2011/2012 si sono rilevate molte isole intermedie negli stessi tratti, probabilmente sopravvissute alle piene che hanno avuto luogo dalla fine del~2008, ma che in seguito sono state erose poiché nella classe matura non si vede un periodo con vegetazione diffusa nei tratti intermedi e vallivi.
\\
Nei tratti~10 e~11 sono presenti quasi sempre alte percentuali di isole giovani ed intermedie che nel corso del tempo sono diventate mature: agli eventi di piena che hanno asportato vegetazione si sono continuamente succeduti periodi di ricrescita che hanno ripristinato la vegetazione giovane; le isole si sono espanse e consolidate in alveo.
\\
Nel tratto~21 un nucleo di isole si è stabilito a partire dall'immagine~2008-07-05 e si è mantenuto fino all'ultima immagine.
\\
La percentuale della vegetazione matura è generalmente minore rispetto a quella giovane, mentre la variazione nel tempo di quest'ultima è maggiore.

I grafici in \cref{graph:erosione-classi-eta-matrix} mostrano il rapporto tra l'areale di erosione e l'areale delle isole presenti nell'immagine iniziale dei confronti, suddiviso per le classi d'età individuate.
Non è rappresentato il tratto~9 poiché in esso è presente l'isola di Cornino; il maggior numero di dati mancanti è dovuto all'assenza di vegetazione in alcuni tratti e in alcune immagini.
%
\begin{figure}
	\centering
	\tikzsetnextfilename{erosione_classe_eta_matrix}
\begin{tikzpicture}[rotate=270]
	\begin{axis}[
		width = 0.6\textwidth,
		height = 0.89\textwidth,
		name = giovane,
		title = {Giovane},
		symbolic x coords = {2007-09-21, 2008-07-05, 2009-07-08, 2010-09-29, 2011-10-02, 2012-08-01, 2013-09-05, 2014-09-08, 2014-10-31, 2015-08-13, 2015-09-12, 2015-10-22, 2016-09-13, 2017-04-21, 2017-06-13, 2018-06-15, 2018-09-16},
		xticklabel style = {
			rotate = 90,
			font = \footnotesize,
		},
		xtick distance = 1,
		ymin = 1,
		ymax = 23,
		ytick = data,
		yticklabel style = {
			rotate = 90,
			font = \footnotesize,
		},
		enlargelimits = 0.02,
		ylabel = {Tratto},
		y dir = reverse,
		colormap/bluered,
		]
		\addplot[
			matrix plot*,
			mesh/cols = 23,	% per fargli leggere colonne formate da 23 righe dal file di testo
			shader = flat corner,	% per interpolare i colori
			point meta max = 1,
		]
        	table [y = tratto, x = data, point meta = \thisrow{cambiamento}] {graphics/data/eros_giov_matrix.txt}; 	
    \end{axis}
    
    \begin{axis}[
		width = 0.6\textwidth,
		height = 0.89\textwidth,
		name = intermedia,
		title = {Intermedia},
		at = {($(giovane.east) + (0.2cm, 0cm)$)},
		anchor = west,
		symbolic x coords = {2007-09-21, 2008-07-05, 2009-07-08, 2010-09-29, 2011-10-02, 2012-08-01, 2013-09-05, 2014-09-08, 2014-10-31, 2015-08-13, 2015-09-12, 2015-10-22, 2016-09-13, 2017-04-21, 2017-06-13, 2018-06-15, 2018-09-16},
		xticklabel style = {
			rotate = 90,
			font = \footnotesize,
		},
		xtick distance = 1,
		ymin = 1,
		ymax = 23,
		ytick = data,
		yticklabels = {,,},
		enlargelimits = 0.02,
		y dir = reverse,
		colormap/bluered,
		]
		\addplot[
			matrix plot*,
			mesh/cols = 23,	% per fargli leggere colonne formate da 23 righe dal file di testo
			shader = flat corner,	% per interpolare i colori
			point meta max = 1,
		]
        	table [y = tratto, x = data, point meta = \thisrow{cambiamento}] {graphics/data/eros_int_matrix.txt};
    \end{axis}
    
    \begin{axis}[
		width = 0.6\textwidth,
		height = 0.89\textwidth,
		name = matura,
		title = {Matura},
		at = {($(intermedia.east) + (0.2cm, 0cm)$)},
		anchor = west,
		symbolic x coords = {2007-09-21, 2008-07-05, 2009-07-08, 2010-09-29, 2011-10-02, 2012-08-01, 2013-09-05, 2014-09-08, 2014-10-31, 2015-08-13, 2015-09-12, 2015-10-22, 2016-09-13, 2017-04-21, 2017-06-13, 2018-06-15, 2018-09-16},
		xticklabel style = {
			rotate = 90,
			font = \footnotesize,
		},
		xtick distance = 1,
		ymin = 1,
		ymax = 23,
		ytick = data,
		yticklabels = {,,},
		enlargelimits = 0.02,
		y dir = reverse,
		colorbar right,
		colorbar style = {
			ylabel = {Vegetazione erosa rapportata alla vegetazione presente},
			yticklabel style = {
				rotate = 90,
				/pgf/number format/.cd,
				fixed,
				fixed zerofill,
				precision = 1,
				/tikz/.cd,
			},
		},
		colormap/bluered,
		]
		\addplot[
			matrix plot*,
			mesh/cols = 23,	% per fargli leggere colonne formate da 23 righe dal file di testo
			shader = flat corner,	% per interpolare i colori
			point meta max = 1,
		]
        	table [y = tratto, x = data, point meta = \thisrow{cambiamento}] {graphics/data/eros_mat_matrix.txt};
    \end{axis}
\end{tikzpicture}
	%\vspace*{-1cm}
	\caption[erosione delle vegetazione divisa in classi d'età rispetto alle isole presenti]{erosione delle vegetazione divisa in classi d'età rispetto alle isole presenti.
	Sono mostrate solamente le immagini a partire dal~2007; il tratto~9 non è stato considerato poiché contiene l'isola di Cornino, che si fonda su roccia e non può essere erosa; le celle bianche indicano mancanza di dati a causa della limitata estensione delle immagini, della presenza di nuvole o dell'assenza di vegetazione.}
	\label{graph:erosione-classi-eta-matrix}
\end{figure}
%
\\
La classe giovane mostra tassi di erosione maggiori di quella intermedia; questa ne presenta a sua volta di più alti rispetto alla vegetazione matura.
I confronti nei quali si era rilevata molta erosione, come il 2008-07-05/2009-07-08, il 2012-08-01/2013-09-05 e il 2014-09-08/2014-10-31, hanno subito erosione in tutte le classi d'età, anche se in maniera differenziata secondo il trend appena descritto.
\\
Non sembrano esserci tratti in cui si osserva l'erosione preferenziale di una classe d'età. Uno dopo l'altro, tutti i tratti sperimentano erosione in ogni classe, tranne per i tratti a valle di Cornino (il~10 e l'11), per i quali non si è assistito ad erosioni intense di vegetazione matura.
\\
I tratti~12 e~23 possono mostrare alte percentuali di erosione, dovute alla presenza di piccoli areali di isole, come è evidente dal grafico dei rapporti tra isole e alveo attivo in \cref{graph:rapp-isl-classi-eta-matrix}.

I grafici in \cref{graph:erosione-classi-eta-4tr-matrix} presentano i medesimi dati con i tratti accorpati in gruppi di~4.
I trend temporali sono ancora più evidenti: in alcuni momenti c'è stata una forte erosione in ogni classe, anche se di entità decrescente secondo l'età; in generale, nessun gruppo mostra tassi d'erosione superiori agli altri.
%
\begin{figure}
	\centering
	\tikzsetnextfilename{eros_classi_eta_4tr_matrix}
\begin{tikzpicture}
	\begin{axis}[
		width = 0.95\textwidth,
		height = 0.4\textwidth,
		name = eros_giov,
		symbolic x coords = {2007-09-21, 2008-07-05, 2009-07-08, 2010-09-29, 2011-10-02, 2012-08-01, 2013-09-05, 2014-09-08, 2014-10-31, 2015-08-13, 2015-09-12, 2015-10-22, 2016-09-13, 2017-04-21, 2017-06-13, 2018-06-15, 2018-09-16},
		xticklabel style = {
			rotate = 90,
		},
		xtick distance = 1,
		symbolic y coords = {1-4, 5-8, 10-12, 13-16, 17-20, 21-23},
		ymin = {1-4},
		ymax = {21-23},
		ytick distance = 1,
		ytick = data,
		ylabel = {Giovane},
		y dir = reverse,
		enlarge x limits = 0.02,
		enlarge y limits = 0.04,
		colormap/bluered,
		colorbar horizontal,
		colorbar style = {
			at = {(0.5, 1.03)},
			anchor = south,
			xticklabel pos = upper,
			xlabel = {Tasso d'erosione rispetto alle isole presenti},
		},
		]
		\addplot[
			matrix plot*,
			mesh/cols = 6,	% per fargli leggere colonne formate da 23 righe dal file di testo
			shader = flat corner,	% per interpolare i colori
			point meta max = 1,
		]
        	table [y = tratto, x = data, point meta = \thisrow{cambiamento}] {graphics/data/eros_giov_4tr_matrix.txt};
    \end{axis}
    
    
	\begin{axis}[
		width = 0.95\textwidth,
		height = 0.4\textwidth,
		name = eros_int,
		at = {($(eros_giov.south) + (0cm, -2.2cm)$)},
		anchor = north,
		symbolic x coords = {2007-09-21, 2008-07-05, 2009-07-08, 2010-09-29, 2011-10-02, 2012-08-01, 2013-09-05, 2014-09-08, 2014-10-31, 2015-08-13, 2015-09-12, 2015-10-22, 2016-09-13, 2017-04-21, 2017-06-13, 2018-06-15, 2018-09-16},
		xticklabel style = {
			rotate = 90,
		},
		xtick distance = 1,
		symbolic y coords = {1-4, 5-8, 10-12, 13-16, 17-20, 21-23},
		ymin = {1-4},
		ymax = {21-23},
		ytick distance = 1,
		ytick = data,
		ylabel = {Intermedia},
		y dir = reverse,
		enlarge x limits = 0.02,
		enlarge y limits = 0.04,
		colormap/bluered,
		]
		\addplot[
			matrix plot*,
			mesh/cols = 6,	% per fargli leggere colonne formate da 23 righe dal file di testo
			shader = flat corner,	% per interpolare i colori
			point meta max = 1,
		]
        	table [y = tratto, x = data, point meta = \thisrow{cambiamento}] {graphics/data/eros_int_4tr_matrix.txt};
    \end{axis}
    
    \begin{axis}[
		width = 0.95\textwidth,
		height = 0.4\textwidth,
		at = {($(eros_int.south) + (0cm, -2.2cm)$)},
		anchor = north,
		symbolic x coords = {2007-09-21, 2008-07-05, 2009-07-08, 2010-09-29, 2011-10-02, 2012-08-01, 2013-09-05, 2014-09-08, 2014-10-31, 2015-08-13, 2015-09-12, 2015-10-22, 2016-09-13, 2017-04-21, 2017-06-13, 2018-06-15, 2018-09-16},
		xticklabels = {,,},
		xtick distance = 1,
		symbolic y coords = {1-4, 5-8, 10-12, 13-16, 17-20, 21-23},
		ymin = {1-4},
		ymax = {21-23},
		ytick distance = 1,
		ytick = data,
		ylabel = {Matura},
		y dir = reverse,
		enlarge x limits = 0.02,
		enlarge y limits = 0.04,
		colormap/bluered,
		]
		\addplot[
			matrix plot*,
			mesh/cols = 6,	% per fargli leggere colonne formate da 23 righe dal file di testo
			shader = flat corner,	% per interpolare i colori
			point meta max = 1,
		]
        	table [y = tratto, x = data, point meta = \thisrow{cambiamento}] {graphics/data/eros_mat_4tr_matrix.txt};
    \end{axis}
\end{tikzpicture}
	\vspace*{-1cm}
	\caption[erosione delle vegetazione divisa in classi d'età rispetto alle isole presenti unendo i tratti 4 a~4]{erosione delle vegetazione divisa in classi d'età rispetto alle isole presenti unendo i tratti 4 a~4.
	Sono mostrate solamente le immagini a partire dal~2007; il tratto~9 non è stato considerato poiché contiene l'isola di Cornino, che è inerodibile; le celle bianche indicano mancanza di dati a causa della limitata estensione delle immagini.}
	\label{graph:erosione-classi-eta-4tr-matrix}
\end{figure}
%
\\
La \cref{tab:varianza-eros-per-eta} mostra la varianza calcolata su tutti i gruppi di tratti per ogni immagine, divisa secondo le classi d'età.
I valori della vegetazione giovane sono maggiori di quelli della vegetazione intermedia, che a sua volta presenta varianza in genere più alta della vegetazione matura.
In ogni caso, le varianze non superano le poche unità percentuali (\num{\sim 0.01}).
%
\begin{table}
	\centering
	\begin{tabular}{c *{3}{S[table-format = 1.4, table-comparator = true]}}
		\toprule
		{\textbf{Ultima data confronto}}	&	{\textbf{Giovane}}	&	{\textbf{Intermedia}}	&	{\textbf{Matura}}	\\
		\midrule
		2007-09-21	&	0.003	&	0.008	&	<0.001	\\
		2008-07-05	&	0.001	&	<0.001	&	<0.001	\\
		2009-07-08	&	0.019	&	0.017	&	0.007	\\
		2010-09-29	&	0.004	&	0.001	&	0.001	\\
		2011-10-02	&	0.023	&	0.013	&	0.005	\\
		2012-08-01	&	<0.001	&	<0.001	&	<0.001	\\
		2013-09-05	&	0.017	&	0.010	&	0.007	\\
		2014-09-08	&	0.008	&	0.010	&	0.007	\\
		2014-10-31	&	0.038	&	0.014	&	<0.001	\\
		2015-08-13	&	0.003	&	0.004	&	0.002	\\
		2015-09-12	&	0.005	&	0.005	&	0.001	\\
		2015-10-22	&	0.006	&	0.005	&	0.014	\\
		2016-09-13	&	0.003	&	0.005	&	0.001	\\
		2017-04-21	&	0.005	&	<0.001	&	<0.001	\\
		2017-06-13	&	0.011	&	0.004	&	0.001	\\
		2018-06-15	&	<0.001	&	0.002	&	<0.001	\\
		2018-09-16	&	0.030	&	0.016	&	<0.001	\\
		\bottomrule
	\end{tabular}
	\caption[varianza spaziale dei tassi d'erosione per le tre classi d'età]{varianza spaziale dei tassi di erosione divisi secondo le classi d'età per ogni confronto tra immagini successive calcolata sui dati dei tratti uniti 4 a~4; i grafici dei tassi sono mostrati in \cref{graph:erosione-classi-eta-4tr-matrix}.
	La data indica l'immagine finale del confronto; sono riportati solo i confronti con data finale a partire dal~2007-09-21 poiché prima di questa immagine la suddivisione in classi d'età non è affidabile.}
	\label{tab:varianza-eros-per-eta}
\end{table}
%

\section{Discussione: un trend per la vegetazione erosa}
La varianza in genere più elevata della vegetazione giovane mostra che questa viene erosa non uniformemente nel tratto di studio: i tassi di erosione sono spazialmente differenziati probabilmente a causa delle diverse condizioni ambientali che possono favorire la crescita in maniera diversificata.
Bisogna comunque ricordare che elevati tassi di erosione si hanno anche quando è presente poca vegetazione di una classe d'età e questa viene portata via.
\\
La classe d'età intermedia mostra varianze elevate nei medesimi confronti in cui le varianze sono elevate anche nella classe giovane; ciò potrebbe essere dovuto alla quota relativa in cui le isole si stanno insediando, non ancora sufficientemente elevata da proteggere le isole da piene intermedie.
\\
La vegetazione matura mostra varianze in genere inferiori ad un punto percentuale: è possibile affermare che la sua erosione avviene in tutto il tratto di studio con tassi molto simili, sebbene contenuti.

Il periodo con assenza di piene intense tra il~2005 e il~2008 ha portato ad un forte incremento della vegetazione, che in seguito è stata erosa durante gli eventi successivi di piena.
La maggior parte della vegetazione erosa è quella giovane cresciuta negli anni precedenti, mentre la percentuale di erosione nelle altre classi d'età è molto minore, sebbene presente.
Dinamiche simili hanno avuto luogo tra il~2011 e il~2013: un periodo favorevole per la crescita ha sperimentato una forte erosione a causa delle intense piene che sono succedute.
Entrambi questi momenti, gli anni in cui c'è stata molta crescita e contemporaneamente sono stati rilevati bassi tassi d'erosione, mostrano di avere varianze spaziali di fatto trascurabili: nel~2008-07-05 e nel~2012-08-01 l'erosione è stata praticamente nulla in ogni tratto e per ogni classe.
\\
Nel periodo 2011-10-02/2012-08-01 si sono rilevate una non trascurabile espansione delle isole e poche piene rilevanti; l'erosione è stata molto debole, quasi assente in ogni classe. Forse le piene avvenute prima di questo periodo hanno asportato gran parte della vegetazione e i pochi eventi oltre i~\SI{2}{\m} tra il~2011 e il~2012 non hanno trovato piante in zone facilmente inondabili.
\\
A seguito dell'evento \emph{bankfull} della fine del~2012 e dei \emph{flood pulses} tra il~2013 e il~2014 molte isole mature sono state erose nei tratti vallivi; il loro tasso di erosione è paragonabile a quello rilevato nelle isole giovani ed intermedie. Queste piene sono riuscite ad avere un effetto di cui non si osservano pari; i periodi di crescita che hanno preceduto questo evento potrebbero aver favorito l'espansione delle isole in zone relativamente più protette dalle piene, quali le zone a quota relativa più alta, mentre molte altre piante giovani ed intermedie sono state travolte; la piena più importante e rara di fine~2012 è riuscita ad erodere molta della vegetazione sopravvissuta ormai matura perché ne era presente una grande quantità e in aree non ancora particolarmente aggradate.
C'è comunque da ricordare che anche piene poco importanti possono erodere isole insediatesi in alveo da tempo, come mostrano le celle con tassi considerevoli di erosione della componente matura sparse nel grafico, poiché, se queste si trovano all'estradosso di canali in curva, possono crollare per scalzamento al piede dell'isola.
\\
Riassumendo, sembra che le possibilità di erosione in una classe dipendano sia dalla disponibilità di isole da erodere, sia dal regime delle piene che ha preceduto il momento che si sta osservando.

Si considerino le isole presenti in alcuni tratti e in alcune immagini, divise per classi di età; si considerino anche le isole erose in seguito alle immagini di cui sopra, divise per classi di età e nei medesimi tratti: ciò che si vede nel grafico in \cref{graph:distr-eta} è la quantità di isole erose per ogni classe, rispetto alle isole presenti. 
%
\begin{figure}
	\centering
	\tikzsetnextfilename{eta_tratti_8_11_17}
\begin{tikzpicture}
	\begin{groupplot}[
		group style = {
			group size = 3 by 1,
			x descriptions at = edge bottom,
			y descriptions at = edge left,
			xticklabels at = all,
			horizontal sep = 0.2cm,
			vertical sep = 0.25cm,
		},
		width = 0.37\textwidth,
		height = 0.8\textwidth,
	    xbar stacked,
		enlarge x limits = 0.02,
		enlarge y limits = 0.10,
		symbolic y coords = {
			2008-07-05, 2009-07-08, 
			2012-08-01, 2013-09-05,  
			2016-09-13, 2017-06-13
		},
		ytick distance = 1,
		%scaled x ticks = false,
		xlabel = {Areale \si{[\m\tothe{2}]}},
		xmajorgrids = true,
		]
		\nextgroupplot % tratto_8
			\addplot[bar shift = .4cm, pattern = north east lines]
		       	table [y=data, x=giovane-e] {graphics/data/tr_8_eta.txt};
			\addplot[bar shift = .4cm, fill = green]
		       	table [
		       		y=data, 
		       		x expr=\thisrow{giovane} - \thisrow{giovane-e}
		       		] {graphics/data/tr_8_eta.txt};
		       		
			\resetstackedplots
			\addplot[bar shift = 0cm, pattern = north east lines, forget plot]
		       	table [y=data, x=interm-e] {graphics/data/tr_8_eta.txt};
			\addplot[bar shift = 0cm, fill = green!75!black]
		       	table [
		       		y=data, 
		       		x expr=\thisrow{interm} - \thisrow{interm-e}
		       		] {graphics/data/tr_8_eta.txt};
		       		
			\resetstackedplots
			\addplot[bar shift = -.4cm, pattern = north east lines, forget plot]
		       	table [y=data, x=matura-e] {graphics/data/tr_8_eta.txt};
			\addplot[bar shift = -.4cm, fill = green!40!black]
		       	table [
		       		y=data, 
		       		x expr=\thisrow{matura} - \thisrow{matura-e}
		       		] {graphics/data/tr_8_eta.txt};
		    
        	\node [fill = white, draw = black, anchor = north east] 
        		at (axis description cs: 1,1) {Tr. 8};
        %
		\nextgroupplot [% tratto_11
			legend style = {
				at = {(0.5,1.02)},
				legend columns = 4,
				anchor = south
			}, 
			]
			\addplot[bar shift = .4cm, pattern = north east lines]
		       	table [y=data, x=giovane-e] {graphics/data/tr_11_eta.txt};
		    \addlegendentry{Erosione}
			\addplot[bar shift = .4cm, fill = green]
		       	table [
		       		y=data, 
		       		x expr=\thisrow{giovane} - \thisrow{giovane-e}
		       		] {graphics/data/tr_11_eta.txt};
		    \addlegendentry{Giovane}
		       		
			\resetstackedplots
			\addplot[bar shift = 0cm, pattern = north east lines, forget plot]
		       	table [y=data, x=interm-e] {graphics/data/tr_11_eta.txt};
			\addplot[bar shift = 0cm, fill = green!75!black]
		       	table [
		       		y=data, 
		       		x expr=\thisrow{interm} - \thisrow{interm-e}
		       		] {graphics/data/tr_11_eta.txt};
		    \addlegendentry{Intermedia}
		       		
			\resetstackedplots
			\addplot[bar shift = -.4cm, pattern = north east lines, forget plot]
		       	table [y=data, x=matura-e] {graphics/data/tr_11_eta.txt};
			\addplot[bar shift = -.4cm, fill = green!40!black]
		       	table [
		       		y=data, 
		       		x expr=\thisrow{matura} - \thisrow{matura-e}
		       		] {graphics/data/tr_11_eta.txt};
		    \addlegendentry{Matura}
		    
        	\node [fill = white, draw = black, anchor = north east] 
        		at (axis description cs: 1,1) {Tr. 11};
        	%
        	
		\nextgroupplot % tratto_17
			\addplot[bar shift = .4cm, pattern = north east lines]
		       	table [y=data, x=giovane-e] {graphics/data/tr_17_eta.txt};
			\addplot[bar shift = .4cm, fill = green]
		       	table [
		       		y=data, 
		       		x expr=\thisrow{giovane} - \thisrow{giovane-e}
		       		] {graphics/data/tr_17_eta.txt};
		       		
			\resetstackedplots
			\addplot[bar shift = 0cm, pattern = north east lines, forget plot]
		       	table [y=data, x=interm-e] {graphics/data/tr_17_eta.txt};
			\addplot[bar shift = 0cm, fill = green!75!black]
		       	table [
		       		y=data, 
		       		x expr=\thisrow{interm} - \thisrow{interm-e}
		       		] {graphics/data/tr_17_eta.txt};
		       		
			\resetstackedplots
			\addplot[bar shift = -.4cm, pattern = north east lines, forget plot]
		       	table [y=data, x=matura-e] {graphics/data/tr_17_eta.txt};
			\addplot[bar shift = -.4cm, fill = green!40!black]
		       	table [
		       		y=data, 
		       		x expr=\thisrow{matura} - \thisrow{matura-e}
		       		] {graphics/data/tr_17_eta.txt};
		    
        	\node [fill = white, draw = black, anchor = north east] 
        		at (axis description cs: 1,1) {Tr. 17};
	\end{groupplot}
\end{tikzpicture}




	\caption[areale delle isole e dell'erosione subita divise in classi d'età per i tratti~8, 11 e~17]{areale delle isole e dell'erosione subita dalle isole stesse, divise in classi d'età, per alcune immagini per i tratti~8, 11 e~17.}
	\label{graph:distr-eta}
\end{figure}
%
Focalizzandosi sugli eventi di piena della fine del~2008 e del~2012, si vede chiaramente come l'areale della vegetazione giovane sia stato fortemente eroso (tratti \numrange[range-phrase={ e }]{11}{17});
sia in termini relativi che in termini assoluti, la vegetazione giovane è quella prevalentemente portata via dalle piene.
\\
Questo è molto probabilmente dovuto alla colonizzazione delle piante delle zone a quota medio-elevata, come le creste delle barre, durante il periodo tra due piene; 
se le condizioni ambientali sono adatte, se non hanno luogo piene particolarmente intense ma piene di piccola-media entità (\emph{flow pulses}) allora le piante possono crescere ottimamente e le isole si accrescono, come riportato nella sezione introduttiva~\ref{sec:descr-area-studio}.
\\
Il periodo compreso tra il~2005 e il~2008 molto probabilmente è stato caratterizzato da condizioni idrologiche ottimali per la crescita di nuova vegetazione, come già osservato nella sezione~\ref{sec:camb-ris}.
Ora è possibile affermare che la forte erosione osservata dall'immagine del cambiamento 2008-07-05/2009-07-08 si è concentrata sulla vegetazione giovane.
\\
Data la ricolonizzazione per via vegetativa di \emph{Populus nigra} e \emph{Salix spp.} che ha luogo dopo ogni piena da parte di tronchi depositati su barre e attorno a isole già esistenti, la presenza di piante giovani è quasi costante.

Momenti in cui non ci sono state piene con livello oltre i~\SI{2}{\m}, come nei periodi 2013-2014 e 2015-2017, mostrano che l'erosione delle isole è stata modesta, di entità ben minore rispetto a quanto appena esposto. Tuttavia, nella classe giovane l'erosione presenta spesso tassi elevati anche in seguito a periodi in cui il tasso di crescita è basso, ad esempio nel~2011-10-02 (grafico in \cref{graph:accrescimento-matrix}).
A meno di eventi molto intensi o di isole poste all'estradosso di canali, la componente giovane delle isole è quella che viene asportata per prima.
%Si tenga comunque in conto che parte della vegetazione non erosa invecchia e può passare da una classe a quella successiva; questo fenomeno non è tuttavia quello predominante durante le piene del~2008 e del~2012 per evidenza.

