% voci del glossario

\newglossaryentry{ortofoto}{name={Ortofoto},description={immagine aerea corretta affinché non vi siano effetti di distorsione dovuti all'inclinazione del sensore o dei rilievi topografici; osservare un punto su una ortofoto è come osservarlo dal suo zenit; è possibile misurarvi distanze vere},see={imm-sat}}

\newglossaryentry{imm-sat}{name={Immagine satellitare},description={indicata semplicemente come “immagine”, viene distinta dalla ortofoto poiché il sensore è montato su satellite e, nel presente lavoro, include solitamente ulteriori bande oltre al visibile},see={ortofoto}}

\newglossaryentry{rgb}{name={R-G-B},description={\emph{Red}, \emph{Green}, \emph{Blue}, le tre bande in cui solitamente si suddividono tutte le immagini; ciascuna banda indica la quantità di colore presente in ogni punto}}

\newglossaryentry{floodplain}{name={\emph{Floodplain}},sort={Floodplain},description={piana alluvionale, porzione dell'alveo posta lateralmente rispetto all'alveo attivo, solitamente vegetata; viene inondata solamente in caso di eventi estremi (tempo di ritorno superiore a \SI{10}{\anni}); in base ai fenomeni di erosione delle sponde può essere modificata nel tempo}}

\newglossaryentry{alveo-att}{name={Alveo attivo},description={porzione dell'alveo che ospita i canali attivi (dove scorre l'acqua) e le forme morfologiche non vegetate (ad esempio barre in ghiaia); può cambiare nel corso degli anni}}

\newglossaryentry{sponda}{name={Sponda},description={parte della \emph{floodplain} più prossima all'alveo attivo e soggetta a dinamiche di erosione durante gli eventi di piena e dinamiche di accrescimento quando le isole si espandono fino a diventare parte della sponda stessa}}

\newglossaryentry{bankfull}{name={Evento \emph{bankfull}},description={evento di piena “a piene rive”, cioè che riempie pienamente l'alveo attivo fino alle sponde; il tempo di ritorno di tali eventi è di \SIrange[range-phrase={-}, range-units=single]{1}{2}{\anni}}}

\newglossaryentry{ndvi}{name={NDVI},description={\emph{Normalized Difference Vegetation Index}, indice compreso tra \SIrange[range-phrase={ e }]{-1}{1}{} dato dal rapporto $\frac{IR-R}{IR+R}$; valori particolarmente alti sono associati con vegetazione viva}}

\newglossaryentry{r}{name={R},description={banda del Rosso, con lunghezza d'onda compresa circa tra \SIrange[range-phrase={ e },range-units=single]{0.6}{0.7}{\micro\m}}}

\newglossaryentry{ir}{name={IR},description={banda dell'InfraRosso vicino, con lunghezza d'onda compresa circa tra \SIrange[range-phrase={ e },range-units=single]{0.75}{0.95}{\micro\m}}}

\newglossaryentry{banda}{name={Banda},description={radiazione elettromagnetica compresa in un determinato intervallo di lunghezze d'onda}}

\newglossaryentry{isola}{name={Isola},description={zona più o meno densamente vegetata completamente circondata dall'alveo attivo, priva di alcuna connessione con la \emph{floodplain}; è soggetta a specifiche dinamiche di erosione laterale durante piene sufficientemente intense e di accrescimento a partire da tronchi depositati nell'alveo attivo così come da piante nate da seme}}

\newglossaryentry{large-wood}{name={\emph{Large wood}},sort={Large wood},description={legno di grandi dimensioni, indica i tronchi di almeno \SI{20}{\centi\m} di diametro e \SI{1}{\m} di lunghezza che sono sradicati durante gli eventi di piena sufficientemente intensi e che sono in seguito depositati poco più a valle nell'alveo attivo; questi tronchi possono successivamente rigettare rami e radici e dare vita a nuove piante; si distingue dalla vegetazione viva poiché è stato eroso da isole o sponde}}

\newglossaryentry{wood-jam}{name={\emph{Wood jam}},sort={Wood jam},description={accumulo di legno, insieme di numerosi \emph{large wood} e di frammenti più piccoli che possono formare accumuli a ridosso di piante vive nell'alveo attivo, su forme morfologiche o lateralmente a sponde e isole}}

\newglossaryentry{risoluzione}{name={Risoluzione},description={immagini, ortofoto e mappe digitali sono formate da un insieme di celle quadrate, ognuna con un valore; il numero di celle costituisce la risoluzione. Tuttavia, se si riduce il lato delle celle aumentandone il numero, la risoluzione non aumenta poiché l'informazione contenuta nella cella originale ha significato e validità solamente se la lunghezza del suo lato non cambia. È invece  possibile e corretto ridurre la risoluzione tramite un aumento del lato delle celle, una contemporanea diminuzione del numero e una distribuzione dell'informazione originale nelle nuove celle}}

\newglossaryentry{percentile}{name={Percentile},description={dato un insieme ordinato di valori, l'$n$-esimo percentile $p_n$ (con $n \in [0,1]$) della popolazione è il valore $p_n$ che divide la popolazione in due parti proporzionali rispettivamente a $n$ e a $1-n$; ognuna di queste due parti ha rispettivamente valori minori di $p_n$ e maggiori di $p_n$}}
