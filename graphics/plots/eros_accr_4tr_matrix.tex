\tikzsetnextfilename{eros_accr_4tr_matrix}
\begin{tikzpicture}
	\begin{axis}[
		width = 0.94\textwidth,
		height = 0.45\textwidth,
		name = erosione,
		symbolic x coords = {2000-09-17, 2001-06-07, 2002-05-18, 2002-06-12, 2003-06-22, 2004-10-14, 2005-08-30, 2006-07-16, 2007-09-21, 2008-07-05, 2009-07-08, 2010-09-29, 2011-10-02, 2012-08-01, 2013-09-05, 2014-09-08, 2014-10-31, 2015-08-13, 2015-09-12, 2015-10-22, 2016-09-13, 2017-04-21, 2017-06-13, 2018-06-15, 2018-09-16},
		xticklabel style = {
			rotate = 90,
		},
		xtick distance = 1,
		symbolic y coords = {1-4, 5-8, 10-12, 13-16, 17-20, 21-23},
		ymin = {1-4},
		ymax = {21-23},
		ytick distance = 1,
		ytick = data,
		ylabel = {Tratto},
		y dir = reverse,
		enlarge x limits = 0.02,
		enlarge y limits = 0.04,
		colorbar horizontal,
		colorbar style = {
			at = {(0.5, 1.03)},
			anchor = south,
			xticklabel pos = upper,
			xlabel = {Tasso d'erosione rispetto alle isole presenti},
			x tick label style = {
				/pgf/number format/.cd,
				fixed,
				fixed zerofill,
				precision = 1,
				/tikz/.cd,
			},
%			xlabel style = {
%				rotate = 180,
%			},
		},
		]
		\addplot[
			matrix plot*,
			mesh/cols = 6,	% per fargli leggere colonne formate da 23 righe dal file di testo
			shader = flat corner,	% per interpolare i colori
			point meta min = 0,
			point meta max = 1,
		]
        	table [y = tratto, x = data, point meta = \thisrow{cambiamento}] {graphics/data/erosione_4tr_matrix.txt};
    \end{axis}
    
    
	\begin{axis}[
		width = 0.94\textwidth,
		height = 0.45\textwidth,
		at = {($(erosione.south) + (0cm, -2.3cm)$)},
		anchor = north,
		symbolic x coords = {2000-09-17, 2001-06-07, 2002-05-18, 2002-06-12, 2003-06-22, 2004-10-14, 2005-08-30, 2006-07-16, 2007-09-21, 2008-07-05, 2009-07-08, 2010-09-29, 2011-10-02, 2012-08-01, 2013-09-05, 2014-09-08, 2014-10-31, 2015-08-13, 2015-09-12, 2015-10-22, 2016-09-13, 2017-04-21, 2017-06-13, 2018-06-15, 2018-09-16},
		xticklabels = {,,},
		xtick distance = 1,
		symbolic y coords = {1-4, 5-8, 10-12, 13-16, 17-20, 21-23},
		ymin = {1-4},
		ymax = {21-23},
		ytick distance = 1,
		ytick = data,
		ylabel = {Tratto},
		y dir = reverse,
		enlarge x limits = 0.02,
		enlarge y limits = 0.04,
		colormap/bluered,
		colorbar horizontal,
		colorbar style = {
			at = {(0.5, -0.03)},
			anchor = north,
			xlabel = {Tasso di accrescimento rispetto alle isole presenti},
			x tick label style = {
				/pgf/number format/.cd,
				fixed,
				fixed zerofill,
				precision = 1,
				/tikz/.cd,
			},
%			xlabel style = {
%				rotate = 180,
%			},
		},
		]
		\addplot[
			matrix plot*,
			mesh/cols = 6,	% per fargli leggere colonne formate da 23 righe dal file di testo
			shader = flat corner,	% per interpolare i colori
			point meta min = 0,
			point meta max = 3,
		]
        	table [y = tratto, x = data, point meta = \thisrow{cambiamento}] {graphics/data/accrescimento_4tr_matrix.txt};
    \end{axis}
\end{tikzpicture}