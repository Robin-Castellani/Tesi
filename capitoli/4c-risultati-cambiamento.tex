\section{Risultati: erosione e accrescimento delle isole}
\label{sec:camb-ris}
I grafici in \cref{graph:erosione-matrix} e in \cref{graph:accrescimento-matrix} mostrano i rapporti tra l'erosione e l'accrescimento delle isole in ogni confronto rispetto all'area delle isole presenti inizialmente.
La data finale della mappa che si confronta corrisponde a quella delle celle colorate; la data iniziale corrisponde a quella della prima mappa valida immediatamente precedente alla data finale (cioè alla data della prima cella colorata a sinistra di ogni cella).
Con riferimento alla \cref{tab:confronti}, la mappa valida più vecchia è il~2000-09-17 per i tratti 1-21 e~2001-06-07 per i tratti~22 e~23; questa non è colorata poiché non costituisce la data finale di alcun confronto.
Il tratto~9 non viene mostrato poiché l'isola di Cornino altera i risultati, in quanto non può essere erosa.
%
\begin{figure}
	\centering
	\tikzsetnextfilename{erosione_matrix}
\begin{tikzpicture}
	\begin{axis}[
		width = 0.98\textwidth,
		height = \textwidth,
		symbolic x coords = {2000-09-17, 2001-06-07, 2002-05-18, 2002-06-12, 2003-06-22, 2004-10-14, 2005-08-30, 2006-07-16, 2007-09-21, 2008-07-05, 2009-07-08, 2010-09-29, 2011-10-02, 2012-08-01, 2013-09-05, 2014-09-08, 2014-10-31, 2015-08-13, 2015-09-12, 2015-10-22, 2016-09-13, 2017-04-21, 2017-06-13, 2018-06-15, 2018-09-16},
		xticklabel style = {
			rotate = 90,
		},
		xtick distance = 1,
		ymin = 1,
		ymax = 23,
		ytick = data,
		ylabel = {Tratto},
		y dir = reverse,
		enlargelimits = 0.02,
		colorbar horizontal,
		colorbar style = {
			xlabel = {Erosione rispetto alle isole inizialmente presenti},
			x tick label style = {
				/pgf/number format/.cd,
				fixed,
				fixed zerofill,
				precision = 1,
				/tikz/.cd,
			},
			},
		]
		\addplot[
			matrix plot*,
			mesh/cols = 23,	% per fargli leggere colonne formate da 23 righe dal file di testo
			shader = flat corner,	% per interpolare i colori
			point meta max = 1,
		]
        	table [y = tratto, x = data, point meta = \thisrow{cambiamento}] {graphics/data/erosione_matrix.txt};
    \end{axis}
\end{tikzpicture}
	\caption[erosione in tutti i confronti rispetto all'area delle isole presenti inizialmente]{erosione in tutti i confronti rispetto all'area delle isole presenti nella data iniziale del confronto; i valori sono compresi tra \numrange[range-phrase = { e }]{0}{1}, che corrispondono rispettivamente a nessuna erosione e alla completa erosione di tutte le isole.
	Ogni confronto termina in una cella colorata e comincia nella cella colorata immediatamente precedente; le mappe inizialmente valide sono il~2000-09-17 per i tratti 1-21 e~2001-06-07 per i tratti~22 e~23; nel tratto~9 è presente l'isola di Cornino e pertanto non viene considerato.}
	\label{graph:erosione-matrix}
\end{figure}
%

Osservando le colonne nell'erosione, si nota il periodo fine~2004/metà~2008 che presenta generalmente bassi tassi di erosione, così come il confronto 2012-08-01/2013-09-05;
al contrario, i confronti 2008-07-05/2009-07-08 e 2012-08-11/2013-09-05 mostrano una forte erosione in quasi tutti i tratti.
Nei primi confronti non è possibile evidenziare trend spaziali data la parziale estensione delle immagini sul tratto di studio e la presenza di nuvole in pochi tratti.
\\
Considerando le righe, i tratti~12 e~23 hanno tassi elevati di erosione in quasi ogni confronti; anche i tratti più a monte subiscono molta erosione nelle isole, anche se ciò non avviene sempre.
I tratti con la larghezza maggiore, cioè i tratti 6-11 e 14-21, mostrano periodi con un'erosione relativamente più intensa alternati da periodi con erosione minore; tuttavia, aldilà dei periodi con bassi tassi di erosione evidenziati prima, non hanno un trend comune.
%
\begin{figure}
	\centering
	\tikzsetnextfilename{accrescimento_matrix}
\begin{tikzpicture}
	\begin{axis}[
		width = 0.98\textwidth,
		height = \textwidth,
		symbolic x coords = {2000-09-17, 2001-06-07, 2002-05-18, 2002-06-12, 2003-06-22, 2004-10-14, 2005-08-30, 2006-07-16, 2007-09-21, 2008-07-05, 2009-07-08, 2010-09-29, 2011-10-02, 2012-08-01, 2013-09-05, 2014-09-08, 2014-10-31, 2015-08-13, 2015-09-12, 2015-10-22, 2016-09-13, 2017-04-21, 2017-06-13, 2018-06-15, 2018-09-16},
		xticklabel style = {
			rotate = 90,
		},
		xtick distance = 1,
		ymin = 1,
		ymax = 23,
		ytick = data,
		ylabel = {Tratto},
		y dir = reverse,
		enlargelimits = 0.02,
		colorbar horizontal,
		colorbar style = {
			xlabel = {Accrescimento rispetto alle isole inizialmente presenti},
			xtick distance = 0.25,
			x tick label style = {
				/pgf/number format/.cd,
				fixed,
				fixed zerofill,
				precision = 1,
				/tikz/.cd,
			},
		},
		colormap/bluered,
		]
		\addplot[
			matrix plot*,
			mesh/cols = 23,	% per fargli leggere colonne formate da 23 righe dal file di testo
			shader = flat corner,	% per interpolare i colori
			point meta max = 3,
		]
        	table [y = tratto, x = data, point meta = \thisrow{cambiamento}] {graphics/data/accrescimento_matrix.txt};
    \end{axis}
\end{tikzpicture}
	\caption[accrescimento in tutti i confronti rispetto all'area delle isole presenti inizialmente]{accrescimento in tutti i confronti rispetto all'area delle isole presenti nella data iniziale del confronto; i valori possono essere maggiori dell'unità se le isole si sono espanse oltre il lor areale iniziale.
	Ogni confronto termina in una cella colorata e comincia nella cella colorata immediatamente precedente; le mappe inizialmente valide sono il~2000-09-17 per i tratti 1-21 e~2001-06-07 per i tratti~22 e~23; nel tratto~9 è presente l'isola di Cornino e pertanto non viene considerato..}
	\label{graph:accrescimento-matrix}
\end{figure}
%

Il valore massimo nel grafico dell'accrescimento è stato limitato a~\num{3} per evitare che i pochi valori molto elevati alterassero completamente la scala di colori; questi valori sono quelli del confronto 2000-09-17/2002-05-18 e quelli nel tratto~23.
\\
Ci sono stati due periodi, tra il~2004 e il~2008 e nel confronto 2011-10-02/2012-08-01, di forte crescita nella maggior parte dei tratti; anche l'ultimo confronto mostra alti tassi di crescita.
Osservando le altre righe si vede che si alternano confronti con poca crescita ad altri con una discreta espansione delle isole.
\\
Alcuni tratti, ad esempio quelli montani, il~6, il~10, il~12 o il~20, presentano generalmente accrescimenti maggiori di altri, come il~7 e il~15, dove l'espansione rispetto alle isole presenti è bassa.


\section{Discussione: eventi eccezionali}
I dati relativi ai tratti più stretti, come quelli montani, il~12 (stretta di Pinzano) e il~23 (cambio morfologico), mostrano spesso alti tassi di erosione e accrescimento poiché l'areale delle isole presenti è generalmente basso (poche celle nelle immagini satellitari) e anche piccoli cambiamenti possono portare ad alte percentuali.
\\
Dalla comparazioni dei grafici di erosione e accrescimento delle isole si può vedere che quando si assiste a molta erosione, la crescita è bassa, e viceversa.
Inoltre, i periodi di forte crescita sono quasi sempre seguiti da confronti che mostrano alti tassi di erosione: molto probabilmente gli anni in cui le piene sono state deboli o in cui si sono susseguite più morbide (\emph{flow pulses}) hanno incentivato la crescita delle piante e la colonizzazione di nuove aree; successivamente, gli eventi intensi di piena hanno trovato molta vegetazione in zone facilmente inondabili e l'hanno asportata.
Infine, si vede come i tratti con \emph{upwelling} più o meno pronunciato siano quelli con i maggiori tassi di crescita (come i tratti 7-11 e i tratti 17-22), mentre quelli dove la falda è in gran parte sprofondata nel materasso alluvionale, come i tratti~15 e~16, mostrano accrescimenti minori rispetto agli altri;
i tratti~4, 5 e~6, in \emph{downwelling}, è probabile che si comportino come i tratti~12 e~23: possono supportare poche isole (grafico in \cref{graph:rapp-isl-tutti-tratti}) e quando queste si espandono si moltiplicano considerevolmente, sebbene il loro areale sia particolarmente inferiore a quello di altri tratti e la loro proporzione rispetto all'alveo attivo sia bassa.
\\
Sembra che il sistema riesca efficacemente a riprendersi dopo ogni periodo di erosione intensa: nell'arco di una stagione vegetativa è possibile osservare una espansione delle isole non trascurabile.
Non ci sono tratti che si riprendono in maniera spiccatamente maggiore di altri; la ricrescita ha luogo in maniera generalizzata in tutto il fiume, come l'erosione dovuta alle piene importanti, anche se, tratto a tratto, altri fattori ambientali inducono crescite spazialmente differenziate.
\\
Probabilmente esiste un limite dinamico di proporzione di isole che possono colonizzare l'alveo dipendente dal regime delle piene: si suppone che non esiste un'unica percentuale massima ammissibile di isole, ma che questa possa variare spazialmente e temporalmente in base agli eventi di piena e alle altre condizioni ambientali che regolano l'erosione e, più ampiamente, la crescita.

Da queste considerazione, osservando l'idrogramma in \cref{graph:livelli-matrix} e in \cref{graph:livelli-orto-sat} (riportato a \cpageref{graph:tr-17-camb}) si può ipotizzare che le piene maggiori, come quelle avvenute verso la fine degli anni~2012 e~2014, siano quelle che abbiano asportato il maggior quantitativo di isole.
\\
I grafici in \cref{graph:tr-17-camb} mostrano alcuni risultati ottenuti dalle mappe di cambiamento per il tratto~17, posto nel tratto vallivo immediatamente a valle della confluenza con il torrente Cosa.
I dati sono rappresentati con due simboli: il pallino rappresenta la data finale di ogni confronto, mentre la croce indica la data iniziale; il pallino di un confronto ha la stessa data della croce del confronto successivo. 
La croce definisce quando inizia ogni confronto; il pallino è il dato vero e proprio.
%
\begin{figure}
	\centering
	\begin{tikzpicture}
	%\begin{groupplot}
	\begin{axis}[
		%name = orto-sat,
		axis y line* = right,
		axis x line* = top,
		%height = .3\textwidth,
		width = \textwidth,
		date coordinates in = x,
		%symbolic y coords = {ASTER,PLEIADES,SENTINEL2,G-EARTH},
		xticklabel = {\year-\month-\day},
		xtick = data,
		ytick = data,
		xticklabel style = {
			rotate = 90,
			anchor = near xticklabel
		},
		enlarge x limits = 0.05,
		enlarge y limits = 0.01,
		ylabel = {Fonte},
		ymax = 3.6,
		ymin = -0.1,
		grid = none,
		only marks,
		]
		\addplot table [x=data, y=numero] {graphics/data/data-orto-sat.txt};
	\end{axis}
	%
	\begin{axis}[
		%name = stages,
		%at = {($(orto-sat.south)-(0,2cm)$)},
		%anchor = north,
		axis y line* = left,
		width = \textwidth,
		date coordinates in = x,
		xticklabel = {\year-\month-\day},
		xticklabel style = {
			rotate = 45,
			anchor = near xticklabel
		},
		enlarge x limits = 0.05,
		enlarge y limits = 0.01,
		ymax = 3.6,
		ymin = -0.1,
		ylabel = {Livello idrometrico},
		grid = major,
		no markers,
		]
		\addplot table [x=data, y=media-gg] {graphics/data/Dati_Villuzza.csv};
	\end{axis}
\end{tikzpicture}
	\\
	\begin{tikzpicture}
	\begin{groupplot}[
		group style = {
			group size = 2 by 1,
			ylabels at = edge left,
			x descriptions at = edge bottom,
			horizontal sep = 1.1cm,
			vertical sep = 0.1cm,
		},
		width = 0.5\textwidth,
		height = 0.5\textwidth,
		date coordinates in = x,
		xticklabel = {$\year$},
		xticklabel style = {
			rotate = 80,
			anchor = near xticklabel
		},
		xtick distance = 731,
		ymax = 0.75,
		ylabel = {Cambiamento/Isole iniziali \si{[\percent]}},%\si{[\m\tothe{2}]}},
		grid = major,
		]
	\nextgroupplot % tr_17_accrescimento
		\addplot+
        	[only marks, blue]
        	table [
        		x=data_fine, 
        		y expr=\thisrow{alv->is}/808650.0
        		] {graphics/data/tr_17_camb_eros_accr.txt};
		\addplot+
        	[only marks, mark=x, black]
        	table [
        		x=data_ini, 
        		y expr=\thisrow{alv->is}/808650.0
        		] {graphics/data/tr_17_camb_eros_accr.txt};
        \node [fill = white, draw = black, anchor = south west] 
        	at (axis description cs: 0.05,0.8) {Accr.};
	\nextgroupplot % tr_17_erosione
		\addplot+
        	[only marks, blue]
        	table [
        		x=data_fine, 
        		y expr=\thisrow{is->alv}/808650.0,
        		] {graphics/data/tr_17_camb_eros_accr.txt};
		\addplot+
        	[only marks, mark=x, black]
        	table [
        		x=data_ini, 
        		y expr=\thisrow{is->alv}/808650.0,
        		] {graphics/data/tr_17_camb_eros_accr.txt};
        \node [fill = white, draw = black, anchor = south west] 
        	at (axis description cs: 0.05,0.8) {Eros.};
	\end{groupplot}
\end{tikzpicture}

	\caption[cambiamenti esperiti dalle isole nel tratto~17]{cambiamenti rilevati nelle isole nel tratto~17 rappresentati come percentuale rispetto all'areale delle isole nella prima data del confronto. 
	Si notano i dati relativi al periodo 2005/2008, al 2011/2012 e al 2017/2018, che mostrano valori particolarmente elevati di crescita; tranne l'ultimo confronto di cui non si hanno dati successivi, questi sono seguiti da momenti in cui la nuova vegetazione è stata erosa.}
	\label{graph:tr-17-camb}
\end{figure}
%
\\
C'è una spiccata crescita a cavallo degli anni~2005 e~2008, seguita da un'importante erosione nel~2009 (\SI{60}{\percent} delle isole presenti nel 2008).
Al contrario, nonostante la maggior intensità e durata, le piene del 2012 e del~2014 hanno eroso una percentuale inferiore di isole, forse perché ne era presente una minore quantità dato l'intervallo di una sola stagione di crescita tra piene intense.
\\
Si può supporre che con la piena avvenuta alla fine del~2018 abbia rimosso gran parte delle isole che si sono formate dal~2017.
Questa osservazione può trovare la seguente giustificazione: il periodo privo di piene con livello al di sopra dei \SI{2}{\m} tra fine del~2004 e la fine del~2008 è stato favorevole per l'insediamento di nuova vegetazione, anche grazie agli eventi di morbida (\emph{flow pulses}) che hanno favorito la crescita delle piante;
le macchie vegetate sono diventate visibili da satellite solamente quando le piante hanno sviluppato una chioma sufficientemente ampia, cioè dopo qualche anno, come si nota nell'immagine del 2008-07-05;
questa più recente vegetazione si è espansa sulle barre e sulle forme morfologiche a quota minore rispetto alle isole più vecchie;
verosimilmente, la prima piena che è giunta ha facilmente portato via tutte queste isole giovani e basse.
\\
Occorre quindi ragionare non solo in termini di singoli eventi di piena, ma estendere le proprie considerazioni all'intero idrogramma, al periodo di tempo tra piene superiori ad un certo livello, alla loro frequenza, poiché sono questi i fattori che possono determinare le dinamiche delle isole.
Il periodo 2005-2008 privo di grandi piene può essere considerato un evento tanto importante quanto la piena lunga ed intensa del mese di novembre 2014.
\\
Sono presenti altri fattori, oltre al periodo privo di piene intense, che regolano la crescita: il 2001/2002 è stato un periodo con piene di piccola entità, ma non si è osservata una forte crescita; invece, il periodo 2017/2018, a cavallo di due eventi \emph{bankfull}, ha supportato quasi un raddoppio delle isole presenti

