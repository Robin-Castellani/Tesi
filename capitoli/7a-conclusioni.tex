In questa tesi sono stati estratti numerosi dati da immagini aeree e satellitari del fiume Tagliamento e delle isole che lo vegetano; si sono osservati numerosi trend spaziali e temporali: dopo aver subito un forte restringimento durante il secolo scorso, negli ultimi due decenni l'alveo si è leggermente riallargato, anche se più lentamente negli ultimi anni; molti tratti hanno sperimentato un'espansione delle isole non trascurabile, mentre altri hanno visto fondersi le loro grandi isole nelle sponde.
\\
Sono state analizzate più fasi della successione biogeomorfologica:
gli elementi legnosi in condizione di poter rigettare rami e radici si trovano nelle zone dell'alveo più elevate e lontane dalla corrente;
l'espansione delle isole è un fenomeno complesso governato da molteplici fattori;
le piante nelle isole hanno diverse strutture d'età e ogni classe d'età si espande e viene erosa differentemente dalle altre;
le piene hanno effetti maggiori sulla vegetazione giovane, mentre quella matura può resistere in alveo più a lungo; sia le piene intense che quelle lievi esercitano effetti ben rilevabili sulla vegetazione di ogni fascia d'età, sebbene in modo non identico.

Sono stati incontrati alcuni limiti durante lo studio di questi fenomeni naturali, di seguito elencati:
%
\begin{itemize}
	\item il legno si individua abbastanza facilmente, anche se non è chiara la distinzione del tipo di elemento legnoso, se questo sta rivegetando o meno, se tra un'immagine e la successiva il legno è il medesimo o è stato rimpiazzato da elementi simili;
	inoltre le immagini satellitari ad alta risoluzione e le ortofoto sono risorse preziose, difficili da ottenere e costose, sebbene possano fornire una grande quantità di informazioni;
	occorre quindi sviluppare una metodologia per individuare e riconoscere in maniera automatica gli elementi legnosi; così facendo, sarebbe possibile osservare grandi porzioni di alveo in tempi brevi;
	%
	\item formalizzare modelli concettuali di crescita vegetativa a partire da elementi legnosi richiede lo studio dell'interazione tra questi e diverse componenti, quali la profondità della falda, le sue oscillazioni, \emph{upwelling} e \emph{downwelling}, la disponibilità e il tipo di superfici da colonizzare, le dinamiche di espansione e coalescenza delle isole.
	Tutti questi fattori rendono il fenomeno della crescita molto più complesso da studiare di quello dell'erosione;
	%
	\item si è visto che anche le parti più vecchie delle isole, stabilitesi in alveo da diversi anni, possono essere erose da piene di piccola entità;
	questo avviene quando le isole si trovano all'estradosso di canali in curva.
	Questo fattore non è stato considerato poiché, data la frequenza delle immagini, prima e durante ogni piena non è possibile conoscere la posizione di ogni canale attivo, se questo è in curva e se un'isola si trova vicino ad esso;
	inoltre, è stato suggerito che variazioni longitudinali della \emph{stream power} e della percentuale di vegetazione possano influenzare il movimento dei canali \squarecite{Arscott:2002-habitat-dynamics}.
\end{itemize}
%

Il confronto dei risultati ottenuti con dati in letteratura ha chiaramente mostrato come sia necessario definire e condividere procedure standardizzate e ripetibili per classificare le isole e analizzarne i cambiamenti (ad esempio erosione, età), come evidenziato da \squarecite{Zanoni:2008}.
Infatti, si è visto come il confronto delle medesime grandezze in diversi lavori possa portare a risultati leggermente diversi, se non a conclusioni contrastanti.
Nonostante ciò, è stato possibile osservare interessanti trend temporali per la larghezza e la quantità di isole in alveo.

È di sicuro interessante verificare se i risultati ottenuti siano trasferibili a fiumi con caratteristiche simili al Tagliamento, come il fiume Piave, i torrenti Meduna e Cellina; contemporaneamente e successivamente, si possono applicare le relazioni tra erosione della vegetazione e regime delle piene in contesti fortemente gestiti, come il fiume Brenta, dove l'eccessiva crescita della vegetazione sta alterando le dinamiche fluviali e sta diminuendo i benefici che l'uomo può ottenere dagli ecosistemi ripari.

Con questa tesi è stato possibile analizzare in modo esteso ed efficace il Tagliamento, soprattutto grazie ai molteplici strumenti liberi e \emph{open source} con cui è stato possibile ottenere, gestire, elaborare e visualizzare i dati;
un importantissimo contributo è stato dato dagli enti governativi, quali la NASA e l'ESA, che hanno messo gratuitamente a disposizione di ogni cittadino un grande quantità di immagini satellitari.
Negli anni futuri, nella speranza che queste fonti di dati inestimabili non vengano meno, è possibile proseguire e raffinare l'analisi svolta, al fine di estendere e validare i risultati ottenuti e le conclusioni che si sono tratte da essi.
