\section{Potenza della corrente}
La potenza della corrente per unità di larghezza e di profondità $\Omega$ in \si{[\watt\per\meter\tothe{2}]} è definita come:
%
\begin{equation}
	\label{eq:omega}
	\Omega = \frac{\rho \, g \, Q \, i_f}{B}
\end{equation}
%
dove:
%
\begin{itemize}
	\item $\rho$ è la densità dell'acqua, pari a \SI{1000}{\kilo\g\per\meter\tothe{3}};
	\item $g$ è l'accelerazione gravitazionale, pari a \SI{9.81}{\m\per\s\tothe{2}};
	\item $Q$ è la portata fluente nel canale misurata in \si{[\m\tothe{3}\per\s]};
	\item $i_f$ è la pendenza in \si{[\m\per\m]};
	\item $B$ è la larghezza del canale in \si{[\m]}.
\end{itemize}
%
In letteratura sono stati proposti modelli concettuali che relazionano la potenza della corrente con la quantità di vegetazione e il tipo di forma vegetata presente:
a bassi livelli di $\Omega$, il disturbo indotto dalla corrente è modesto e si possono formare numerose isole sulle barre nude in alveo;
se la potenza della corrente, cioè il disturbo, è maggiore, si potranno formare meno isole e le forme fluviali rimarranno prevalentemente nude\squarecite{Gurnell:2014-plants-eng}.
\\
Difatti l'espansione maggiore della vegetazione e l'erosione minore hanno luogo dove il tasso di crescita della vegetazione è elevato e dove l'energia della corrente è ridotta \squarecite{Gurnell:2006-omega}.
\\
Inoltre le piante che crescono più rapidamente sembrano essere maggiormente flessibili \squarecite{Bertoldi:2011-ASTER}: la ricrescita vegetativa da tronchi vivi, la quale permette un rapido sviluppo, è proprio la caratteristica fondamentale delle piante che abitano questo ambiente, come già esposto nella sezione~\ref{sec:descr-area-studio}.
È dunque lecito supporre che proprio nei tratti dove si osserva una ridotta potenza della corrente si possa trovare un'elevata quantità di isole nella maggior parte del periodo di studio.


Per ottenere la \emph{stream power} si è prima ottenuta la pendenza media di ogni tratto grazie al DEM del 2009 nella seguente maniera:
%
\begin{aenumerate}
	\item si è definita una coordinata curvilinea che segue il corso principale del fiume;
	\item sono state considerate le quote in tutti i punti sopra i quali scorre la coordinata curvilinea;
	\item è stata effettuata una regressione lineare tra la coordinata curvilinea e le quote raggruppando i tratti quattro a quattro (tranne per l'ultimo gruppo, formato dai tratti 21, 22 e 23);
	\item il coefficiente angolare della retta rappresenta la pendenza del gruppo di 4 tratti.
\end{aenumerate}
%
Il risultato è riportato nella \cref{tab:pendenza}.
%
\begin{table}
	\centering
	\begin{tabular}{
		S[list-separator={, }, list-final-separator={ e }]
		S[table-format=1.2]
	}
		\toprule
		\multicolumn{1}{c}{Tratti}	&	\multicolumn{1}{c}{Pendenza \si{[\percent]}}	\\
		\midrule
		\numlist{1;2;3;4}	&	0.52	\\
		\numlist{5;6;7;8}	&	0.48	\\
		\numlist{9;10;11;12}	&	0.23	\\
		\numlist{13;14;15;16}	&	0.40	\\
		\numlist{17;18;19;20}	&	0.31	\\
		\numlist{21;22;23}	&	0.25	\\
		\bottomrule
	\end{tabular}
	\caption[pendenze dei tratti]{pendenze dei tratti.}
	\label{tab:pendenza}
\end{table}
%
\\
In quanto modificazioni della pendenza avvengono in tempi scala elevati, si considera la pendenza ottenuta come rappresentativa e costante nel periodo di studio.

Il grafico in \cref{graph:omega-perc-50} mostra la mediana temporale della potenza della corrente per ogni tratto; si rappresenta la mediana poiché la variazione di $\Omega$ nel tempo è trascurabile.
%TODO scrivi che non hai considerato la portata per quanto esposto nell'introduzione → Omega fittizia con una portata unica
%TODO scrivi che hai pesato con l'area drenante relativa a Villuzza in ogni tratto utilizzando i dati degli affluenti
%
\begin{figure}
	\centering
	\tikzsetnextfilename{omega_perc_50}
\begin{tikzpicture}
	\begin{axis}[
		width = 0.95\textwidth,
		height = .5\textwidth,
		enlarge x limits = 0.01,
		%enlarge y limits = 0.01,
		xlabel = {Tratto},
		ylabel = {Potenza della corrente \si{[\newton\per\metre\tothe{4}]}},
		xtick = data,
		grid = major,
		yticklabel style = {
			/pgf/number format/fixed
		},
		]
		\addplot[only marks, mark = *]
        	table [x = tratto, y = omega,] {graphics/data/omega_perc_50.txt};
	\end{axis}
\end{tikzpicture}
	\caption[potenza della corrente in ogni tratto]{mediana temporale della potenza della corrente per ogni tratto pesata con la relativa area drenante.}
	\label{graph:omega-perc-50}
\end{figure}

%TODO scrivi che si vede un trend spaziale di allargamento-restringimento

Rappresentando questa potenza della corrente rispetto alle isole presenti in un grafico bilogaritmico (), è possibile osservare l'effetto di controllo sulla quantità massima di vegetazione che esercita $\Omega$.
%TODO è stata escluso il tratto con l'isola di Cornino
%
\begin{figure}
	\centering
	\tikzsetnextfilename{omega_area_percentuale}
\begin{tikzpicture}
	\begin{axis}[
		width = \textwidth,
		height = .5\textwidth,
		%enlarge x limits = 0.01,
		%enlarge y limits = 0.01,
		ylabel = {$\Omega$ \si{[\newton\per\metre\tothe{4}]}},
		xlabel = {Isole rispetto all'alveo attivo},
		grid = major,
		legend columns = -1,
		legend style = {
			anchor = south,
			at = {(0.5, 1.01)},
		},
		colormap = {fitting point colormap}{
				color = (black)
				color = (white!80!black)
%				color = (cyan!75!black)
%				color = (orange!75!black)
%				color = (green!75!black)
            },
        log ticks with fixed point,
        xmode = log,
        ymode = log,
        log basis x = 10,
        log basis y = 10,
		]
		\foreach \tratto in {1,2,...,23}
			{
			\addplot[
				scatter,
				only marks,
				point meta = {ifthenelse(y < -1.5184-0.1838*x, 1, ifthenelse(y < -1.5056 - 0.2516 * x, 0.7, 0))},
				point meta max = 1,
				point meta min = 0,
				forget plot,
			]
				table [y = om_tr_\tratto, x = area_tr_\tratto]
				{graphics/data/omega_area_percentuale.txt};
			}
		
		\addplot [color = green, % T1
			 line width = 2 pt,
			 domain = 1e-3:4e-1,
			 samples = 10,
			 ] 
			 {10^(-1.5184 - 0.1838 * log10(x))};
		\addlegendentry{Fit 1}
%		\node [fill = white, draw = green, anchor = east] % T1 
%        	at (axis description cs: 1,0.6) {$y = 10^{-1.5184} \, x^{- 0.1838}$};
        	
		\addplot [color = orange, % T2
			 line width = 2 pt,
			 domain = 1e-3:4e-1,
			 samples = 10,
			 ]
			 {10^(-1.5056 - 0.2516 * log10(x))};
		\addlegendentry{Fit 2}
%		\node [fill = white, draw = orange, anchor = east] % T2 
%        	at (axis description cs: 1,0.75) {$y = 10^{-1.5056} \, x^{- 0.2516}$};
        	
		\addplot [color = cyan, % T3
			 line width = 2 pt,
			 domain = 1e-3:4e-1,
			 samples = 10,
			 ]
			 {10^(-1.4085 - 0.2371 * log10(x))};
		\addlegendentry{Fit 3}
		\node [fill = white, draw = cyan, anchor = east] % T3 
        	at (axis description cs: 1,0.9) {$y = 10^{-1.4085} \, x^{- 0.2371}$};
        
		\node [fill = white, draw = black, anchor = south west] % T3 
        	at (axis description cs: 0,0) {$R^2 \in [0.3, 0.6]$ $P_\mathrm{val} < 0.0001$};
	\end{axis}
\end{tikzpicture}
	\caption[potenza della corrente rispetto alla proporzione di isole sull'alveo attivo]{potenza della corrente rispetto alla proporzione di isole sull'alveo attivo.}
	\label{graph:omega-area-percentuale}
\end{figure}
%
\begin{figure}
	\centering
	\tikzsetnextfilename{omega_area_pura}
\begin{tikzpicture}
	\begin{axis}[
		width = \textwidth,
		height = .5\textwidth,
		%enlarge x limits = 0.01,
		%enlarge y limits = 0.01,
		ylabel = {$\Omega$ \si{[\newton\per\metre\tothe{4}]}},
		xlabel = {Isole rispetto all'alveo attivo \si{[\percent]}},
		xmode = log,
		ymode = log,
		grid = major,
		legend entries = {1,2,3,4,5,6,7,8,10,11,12,13,14,15,16,17,18,19,20,21,22,23},
		legend columns = 15,
		legend style = {
			anchor = south,
			at = {(0.5, 1.01)},
		},
		]
		\foreach \tratto in {1,2,...,23}
			{
			\addplot+[only marks]
				table [y = om_tr_\tratto, x = area_tr_\tratto]
				{graphics/data/omega_area_pura.txt};
			}
		
		\addplot [color = green, % T1
			 line width = 2 pt,
			 domain = 1e3:1e6,
			 samples = 10,
			 ] 
			 {10^(-0.3741 - 0.1820 * log10(x))};
		\node [fill = white, draw = cyan, anchor = east] % T1 
        	at (axis description cs: 1,0.9) {$10^{-0.3741} \, x^{- 0.1820}$};
        	
		\addplot [color = orange, % T2
			 line width = 2 pt,
			 domain = 1e3:1e6,
			 samples = 10,
			 ]
			 {10^(-0.0157 - 0.2366 * log10(x))};
		\node [fill = white, draw = orange, anchor = east] % T2 
        	at (axis description cs: 1,0.75) {$10^{-0.0157} \, x^{- 0.2366}$};
        	
		\addplot [color = cyan, % T3
			 line width = 2 pt,
			 domain = 1e3:1e6,
			 samples = 10,
			 ]
			 {10^(0.1130 - 0.2507 * log10(x))};
		\node [fill = white, draw = green, anchor = east] % T3 
        	at (axis description cs: 1,0.6) {$10^{0.1130} \, x^{- 0.2507}$};
        	
        \node [fill = white, draw = black, anchor = south west] % T3 
        	at (axis description cs: 0,0) {$R^2 \in [0.4, 0.8]$ $P_\mathrm{val} < 0.0001$};
	\end{axis}
\end{tikzpicture}
	\caption[potenza della corrente rispetto all'areale delle isole]{potenza della corrente rispetto all'areale delle isole.}
	\label{graph:omega-area-pura}
\end{figure}

