\section{Quantità di vegetazione in alveo}
Si è misurata l'area della vegetazione presente in alveo e nella parte di floodplain soggetta a a forti disturbi (erosione e distacco di isole); 
a tal fine si è seguito l'approccio di altri autori in analisi simili eseguite su immagini ASTER e LandSat~TM \parencites{Bertoldi:2011-ASTER}{Henshaw:2013-LandSat}.
%
\begin{enumerate}
	\item Dapprima è stata individuata manualmente una maschera di calcolo che comprendesse l'alveo attivo e la parte di piana alluvionale che è stata erosa quando coinvolta nelle piene; 
	tale maschera si estende da Tolmezzo al ponte di Madrisio
	(\vref{fig:esempio-maschera}). 
	Applicandola, il dominio computazionale è stato ridotto a comprendere l'inviluppo degli alvei attivi che si sono succeduti dall'immagine del~2000 all'ortofoto del~2017.
	%
	\begin{figure}[t]
		\centering
		\begin{subfigure}[b]{0.296\textwidth}
			\includegraphics[width=\textwidth]{files/esempio_mask_2000_09_17.jpeg}
			\caption{ASTER 2000-09-17.}
		\end{subfigure}
		\qquad
		\begin{subfigure}[b]{0.30\textwidth}
			\includegraphics[width=\textwidth]{files/esempio_mask_2015_09_11.jpeg}
			\caption{ASTER 2015-09-11.}
		\end{subfigure}
		\caption[definizione della maschera per limitare il dominio computazionale]
			{esempio in cui si vede come la maschera utilizzata per limitare il dominio computazionale (in giallo) sia il risultato dell'inviluppo degli alvei attivi che si sono modificati nel tempo; le immagini sono in falsi colori (IR-R-G).}
		\label{fig:esempio-maschera}
	\end{figure}
	%
	\item In questa area è stato calcolato il \emph{Normalized Difference Vegetation Index} (NDVI) grazie alle bande del \emph{Near Infrared} (NIR) e del \emph{Red} (R)
	%
	\begin{equation}
		%\notag
		NDVI = \frac{NIR - R}{NIR + R} \quad .
		\label{eq:ndvi}
	\end{equation}
	%
	\item \`{E} stata effettuata una digitalizzazione manuale di alcune aree campione per le immagini ASTER del~2005-08-30 ($\sim 70$) e del~2012-08-01 ($\sim 100$), l'immagine Plaiades del~2014-10-31 ($\sim 40$) e l'immagine Sentinel2 del~2017-04-21 ($\sim 45$) (\vref{fig:esempio-aree-campione}).
	Sono state selezionate immagini per ogni satellite poiché ciascuno è sensibile a bande leggermente diverse; si sono osservate due immagini ASTER dato il grande numero di immagini disponibili scegliendo quelle con minor nuvolosità.
	Queste aree campione sono state suddivise in tre classi: vegetazione, alveo attivo e canale.
	%
	\begin{figure}[ht]
		\centering
		\includegraphics[width=\textwidth]{files/esempio_aree_campione_2014_10_31.jpeg}
		\caption[esempio di aree campione per calcolare la distribuzione dell'NDVI]{esempio di digitalizzazione di alcune aree campione per l'immagine Pleiades del~2014-10-31; sullo sfondo la mappa dell'NDVI.}
		\label{fig:esempio-aree-campione}
	\end{figure}
	%
	\item Per ogni immagine si è osservata la distribuzione dell'NDVI in ogni classe (\vref{graph:percentili}).
	% 
	\begin{figure}[ht]
		\centering
		\begin{tikzpicture}
	\begin{groupplot}[
		group style = {
			group size = 2 by 3,
			ylabels at = edge left,
			x descriptions at = edge bottom,
			horizontal sep = 1.1cm,
			vertical sep = 0.1cm,
		},
		width = 0.45\textwidth,
		height = 0.45\textwidth,
		ylabel = NDVI,
		boxplot/draw direction = y,
		xtick = {1,2,3},
		xticklabels = {Veg., Alveo, Canale},
		ymax = 0.8,
		ymin = -0.52,
		grid = major,
	]
	\nextgroupplot % ASTER 2005-08-31
		\addplot+ [ % vegetazione
			teal, very thick,
			boxplot prepared = {
				lower whisker = 0.353656,
				lower quartile = 0.470411,
				median = 0.560063,
				upper quartile = 0.614701,
				upper whisker = 0.644957,
				},
        	]
        	coordinates {};
		\addplot+ [ % alveo attivo
			brown, very thick,
			boxplot prepared = {
				lower whisker = 0.077472,
				lower quartile = 0.091653,
				median = 0.122488,
				upper quartile = 0.149573,
				upper whisker = 0.171459,
				},
        	]
        	coordinates {};
		\addplot+ [ % canale
			cyan, very thick,
			boxplot prepared = {
				lower whisker = -0.477885,
				lower quartile = -0.362798,
				median = -0.269905,
				upper quartile = -0.058787,
				upper whisker = 0.072414,
				},
        	]
        	coordinates {};
        \node [fill = white, draw = black, anchor = center] 
        	at (2.5,0.6) {AST 2005-08-31};
	%------------------------------------------------------
	\nextgroupplot % ASTER 2012-08-01
		\addplot+ [ % vegetazione
			teal, very thick,
			boxplot prepared = {
				lower whisker = 0.341613,
				lower quartile = 0.444200,
				median = 0.586294,
				upper quartile = 0.672889,
				upper whisker = 0.709027,
				},
        	]
        	coordinates {};
		\addplot+ [ % alveo attivo
			brown, very thick,
			boxplot prepared = {
				lower whisker = 0.10506,
				lower quartile = 0.117969,
				median = 0.143631,
				upper quartile = 0.16549,
				upper whisker = 0.184871,
				},
        	]
        	coordinates {};
		\addplot+ [ % canale
			cyan, very thick,
			boxplot prepared = {
				lower whisker = -0.432201,
				lower quartile = -0.379825,
				median = -0.322239,
				upper quartile = -0.226459,
				upper whisker = -0.103914,
				},
        	]
        	coordinates {};
        \node [fill = white, draw = black, anchor = center] 
        	at (2.5,0.6) {AST 2012-08-01};
	%------------------------------------------------------
	\nextgroupplot % Pleiades 2014-10-31
		\addplot+ [ % vegetazione
			teal, very thick,
			boxplot prepared = {
				lower whisker = 0.286467,
				lower quartile = 0.350238,
				median = 0.415502,
				upper quartile = 0.483495,
				upper whisker = 0.549505,
				},
        	]
        	coordinates {};
		\addplot+ [ % alveo attivo
			brown, very thick,
			boxplot prepared = {
				lower whisker = 0.049796,
				lower quartile = 0.055794,
				median = 0.063049,
				upper quartile = 0.07173,
				upper whisker = 0.081427,
				},
        	]
        	coordinates {};
		\addplot+ [ % canale
			cyan, very thick,
			boxplot prepared = {
				lower whisker = -0.426415,
				lower quartile = -0.387978,
				median = -0.338308,
				upper quartile = -0.266515,
				upper whisker = -0.175373,
				},
        	]
        	coordinates {};
        \node [fill = white, draw = black, anchor = center] 
        	at (2.5,0.6) {PL 2014-10-31};
	%------------------------------------------------------
	\nextgroupplot % Pleiades 2015-08-13
		\addplot+ [ % vegetazione
			teal, very thick,
			boxplot prepared = {
				lower whisker = 0.415693,
				lower quartile = 0.5,
				median = 0.570359,
				upper quartile = 0.638507,
				upper whisker = 0.704044		
,
				},
        	]
        	coordinates {};
		\addplot+ [ % alveo attivo
			brown, very thick,
			boxplot prepared = {
				lower whisker = 0.075052,
				lower quartile = 0.080858,
				median = 0.087921,
				upper quartile = 0.096031,
				upper whisker = 0.106198,
				},
        	]
        	coordinates {};
		\addplot+ [ % canale
			cyan, very thick,
			boxplot prepared = {
				lower whisker = -0.262599,
				lower quartile = -0.244228,
				median = -0.214393,
				upper quartile = -0.176471,
				upper whisker = -0.132762,
				},
        	]
        	coordinates {};
        \node [fill = white, draw = black, anchor = center] 
        	at (2.5,0.6) {PL 2015-08-13};
	%------------------------------------------------------
	\nextgroupplot % Sentinel2 2017-04-21
		\addplot+ [ % vegetazione
			teal, very thick,
			boxplot prepared = {
				lower whisker = 0.163722,
				lower quartile = 0.241916,
				median = 0.374344,
				upper quartile = 0.548241,
				upper whisker = 0.672782,
				},
        	]
        	coordinates {};
		\addplot+ [ % alveo attivo
			brown, very thick,
			boxplot prepared = {
				lower whisker = 0.056176,
				lower quartile = 0.061278,
				median = 0.067681,
				upper quartile = 0.076396,
				upper whisker = 0.089304,
				},
        	]
        	coordinates {};
		\addplot+ [ % canale
			cyan, very thick,
			boxplot prepared = {
				lower whisker = -0.322237,
				lower quartile = -0.288822,
				median = -0.239533,
				upper quartile = -0.177094,
				upper whisker = -0.131119,
				},
        	]
        	coordinates {};
        \node [fill = white, draw = black, anchor = center] 
        	at (2.5,0.6) {S2 2017-04-21};
	%------------------------------------------------------
	\nextgroupplot % WorldView2 2018-06-15
		\addplot+ [ % vegetazione
			teal, very thick,
			boxplot prepared = {
				lower whisker = 0.569665,
				lower quartile = 0.657917,
				median = 0.719523,
				upper quartile = 0.759148,
				upper whisker = 0.791594,
				},
        	]
        	coordinates {};
		\addplot+ [ % alveo attivo
			brown, very thick,
			boxplot prepared = {
				lower whisker = 0.126214,
				lower quartile = 0.129661,
				median = 0.13373,
				upper quartile = 0.138542,
				upper whisker = 0.149326,
				},
        	]
        	coordinates {};
		\addplot+ [ % canale
			cyan, very thick,
			boxplot prepared = {
				lower whisker = -0.416974,
				lower quartile = -0.392405,
				median = -0.365385,
				upper quartile = -0.335135,
				upper whisker = -0.29979,
				},
        	]
        	coordinates {};
        \node [fill = white, draw = black, anchor = center] 
        	at (2.55,0.6) {WV2 2018-06-15};
	\end{groupplot}
\end{tikzpicture}

		\caption[boxplot dell'NDVI nelle aree campione in quattro immagini satellitari]{boxplot dell'NDVI nelle aree campione in quattro immagini satellitari; i baffi indicano il 10mo e il 90mo percentile, gli estremi della scatola rappresentano il 25mo e il 75mo percentile, la linea nella scatola è la mediana.}
		\label{graph:percentili}
	\end{figure}
	%
	\item Da tali grafici sono state ottenute delle soglie di NDVI per classificare le immagini satellitari (\vref{tab:ndvi-soglia}). Le soglie sono in accordo con quanto riportato in letteratura \parencite{Bertoldi:2011-ASTER}.
	%
	\begin{table}[ht]
		\centering
		\begin{tabular}{
			c 
			S[table-format=1.1]@{\,}
			c@{\,}
			c@{\,}
			c@{\,}
			S[table-format=1.1]
			}
			\toprule
			\multicolumn{6}{c}{\textbf{Soglie}}	\\
			\midrule
			Vegetazione		&	0.2	&	$\leq$	&	NDVI	&		& 	\\
			Alveo attivo	&	0.0	&	$\leq$	&	NDVI	&	$<$		&	0.2	\\
			Canale			&		&			&	NDVI	&	$<$		&	0.0	\\
			\bottomrule
		\end{tabular}
		\caption[soglie NDVI]{soglie di NDVI per la classificazione delle immagini satellitari.}
		\label{tab:ndvi-soglia}
	\end{table}
\end{enumerate}


Con la riclassificazione delle immagini dell'NDVI rispetto alle soglie proposte è stato possibile ottenere la percentuale di alveo coperta da vegetazione per ogni anno. 
Si ricorda, grazie alla maschera applicata, tale copertura include sia isole vegetate sia la parte di piana alluvionale che nel periodo di studio ha esperito fenomeni di erosione della vegetazione e quindi espansione dell'alveo attivo.
Infine, per le immagini l'alveo parzialmente coperto da nuvole, la maschera è stata estesa per escludere tali zone coperte poiché queste presentano valori di NDVI non corretti.
\\
I risultati sono mostrati nel grafico in \vref{graph:class-sat-veg}.


\begin{figure}[ht]
	\centering
	\begin{tikzpicture}
	\begin{axis}[
		width = \textwidth,
		height = 0.5\textwidth,
		date coordinates in = x,
		date ZERO = 2000-01-01,
		xticklabel = {\year},
		xticklabel style = {
			rotate = 80,
			anchor = near xticklabel
		},
		axis y line* = right,
		%ymax = 70,
		%ymin = 0,
		ylabel = {Percentuale di vegetazione},
		grid = none,
		]
		\addplot+
        	[red, mark=+, ultra thick]
        	table [x=data, y=veg] {graphics/data/Class_sat_veg-H2O-ghiaia.txt};
	\end{axis}
	
	\begin{axis}[
		width = \textwidth,
		height = 0.5\textwidth,
		date coordinates in = x,
		date ZERO = 2000-01-01,
		xticklabel = {\year},
		xticklabel style = {
			rotate = 80,
			anchor = near xticklabel
		},
		axis y line* = left,
		axis x line = none,
		enlarge x limits = 0.05,
		enlarge y limits = 0.01,
		ymax = 3.7,
		ymin = 2,
		ylabel = {Livello idrometrico \si{[\m]}},
		grid = none,
		]
		\addplot+
        	[blue, no markers, ultra thin]
        	table [x=data, y=media-gg] {graphics/data/Dati_Villuzza.csv};
	\end{axis}
\end{tikzpicture}

	\caption[andamento dell'areale della vegetazione nelle isole  e nella floodplain]{andamento dell'areale della vegetazione nelle isole e nella floodplain (in rosso). I dati provengono dalla classificazione delle immagini satellitari (ASTER, Pleiades e Sentinel2). In blu sono mostrati i livelli idrometrici superiori a~\SI{2}{\m} registrati alla stazione di Villuzza.}
	\label{graph:class-sat-veg}
\end{figure}
