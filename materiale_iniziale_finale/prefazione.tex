%\begin{abstract}
La presente tesi mira a studiare come cambia la vegetazione nel fiume braided Tagliamento in risposta all'idrologia. 
L'obiettivo principale è quello di ricercare una relazione tra i livelli del pelo libero registrati da alcuni idrometri e la vegetazione erosa, tra i livelli e la quantità di legname che si ritrova in alveo.

Si analizzano immagini satellitari e ortofoto al fine di distinguere la parte vegetata dell'alveo e il legname presente.
Con i dati di piovosità media mensile e di temperatura media mensile si cercano correlazioni con l'espansione della vegetazione che si osserva negli anni.
Dalla quantificazione dell'erosione della vegetazione dovuta alle piene, della quantità di legno in alveo e di un tasso di crescita della vegetazione si tenta di costruire un bilancio di materia vegetale a scala di evento di piena.
Inoltre si trovano valori soglia per l'erosione della vegetazione.
%\end{abstract}
