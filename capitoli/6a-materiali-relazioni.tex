\bigskip
Con i dati a disposizione è possibile ricercare le relazioni che leghino il regime delle piene con la dinamica delle isole.
Poiché l'erosione delle isole è legata solamente agli effetti delle piene, mentre, come è stato già osservato nei capitoli precedenti, l'accrescimento è influenzato da più fattori, si esclude quest'ultimo dall'analisi.

\section{Metodi: parametri considerati}

\subsection{Isole erose}
Dallo studio dei cambiamenti delle isole tra un'immagine e quella successiva si sono ottenuti i dati di areale delle isole erose (sezione~\ref{sec:cambiamento});
questi dati sono stati suddivisi secondo tre classi di età della vegetazione: giovane, intermedia e matura (sezione~\ref{sec:eta}).
Quindi è nota la quantità di isole erose per ogni confronto tra le immagini, in ogni tratto e per ogni classe di età.

\subsection{Integrale dei livelli}
Dai dati di livello dell'acqua presso la stazione idrometrica di Villuzza, alcuni autori hanno ottenuto una statistica dei tempi di ritorno dei picchi delle piene superiori ad \SI{1}{\m} \squarecite{Bertoldi:2009-2m}; questa è stata estesa utilizzando i dati di livello idrometrico dal~2000-01-01 al~2018-12-31 ed è riportata nel grafico in \cref{graph:tr-picchi}; in totale sono stati individuati poco più di~200 picchi.
%
\begin{figure}
	\centering
	\tikzsetnextfilename{tr_picchi}
\begin{tikzpicture}
	\begin{semilogxaxis}[
		width = \textwidth,
		height = 0.5\textwidth,
%		enlarge x limits = 0.05,
%		enlarge y limits = 0.01,
%		ytick distance = 0.5,
		ylabel = {Livello idrometrico \si{[\m]}},
		xlabel = {Tempo di ritorno \si{[\anni]}},
		y tick label style = {
			/pgf/number format/.cd,
			fixed,
			fixed zerofill,
			precision = 1,
			/tikz/.cd,
		},
		grid = major,
        log ticks with fixed point,		
		]
		\addplot
        	[blue, no markers]
        	table [x=tr_anni, y=picchi] {graphics/data/tr_picchi.txt};
        
        \draw[<->] (0.1,3) -- (0.5,3);
        \node [at = {(0.23,3)}, anchor = north] {\emph{Flow pulses}};
        
        \draw[<->] (0.5,1) -- (3,1);
        \node [at = {(1.3,1)}, anchor = south] {\emph{Flood pulses}};
        
        \draw[<->] (3,1) -- (20,1);
        \node [at = {(7,1)}, anchor = south] {\emph{Bankfull}};
	\end{semilogxaxis}
\end{tikzpicture}
	\caption[tempi di ritorno dei picchi superiori ad \SI{1}{\m}]{tempi di ritorno dei picchi superiori ad~\SI{1}{\m} ottenuti dall'individuazione di più di~200 picchi nei dati di livello registrati nel periodo di studio dall'idrometro presso Villuzza rispetto ad uno zero locale; rielaborato da \squarecite{Bertoldi:2009-2m}.}
	\label{graph:tr-picchi}
\end{figure}
%
\\
Non sono stati utilizzati i massimi annuali poiché piene con tempo di ritorno inferiore ad \SI{1}{\anno} hanno anch'esse importanti effetti sulla morfologia del fondo e sulle isole.
\\
La statistica effettuata da \squarecite{Bertoldi:2009-2m} utilizzava i dati dal~1981 al~2007.
Nel presente lavoro non sono stati utilizzati dati idrometrici anteriori al~1981 poiché precedentemente i livelli erano manualmente letti su un'asta graduata in un singolo momento della giornata; si capisce che questi dati sono molto meno affidabili di quelli rilevati con intervallo orario o semi orario, che sono in grado di descrivere dettagliatamente il passaggio di ogni piena.
Sono stati utilizzati solo i dati dal~2000 poiché questa è la data di installazione del nuovo sensore idrometrico presso Villuzza; la differenza tra la statistica elaborata in letteratura e quella della tesi è trascurabile.

Questa statistica permette di distinguere diversi tipi di eventi (in accordo con quanto riportato in letteratura \squarecites{Bertoldi:2009-2m}{Bertoldi:2010-d50}{Surian:2015}):
%
\begin{itemize}
	\item \emph{flow pulses}, cioè piene di modesta entità, sono quelle con un livello inferiore ai \SI{2}{\m}, e con un tempo di ritorno di circa \SI{6}{\mesi}; questi periodi di morbida possono rimodellare il fondo, anche se generalmente non riescono a sommergere le isole insediatesi in alveo da anni;
	\item \emph{flood pulses}, piene intense, superano il livello di \SI{2}{\m} e arrivano fino a \SI{3}{\m}, presentando quindi un tempo di ritorno dell'ordine di \SI{1}{\anno}; queste piene hanno effetti sulla vegetazione in quanto riescono ad erodere lateralmente le isole e a sommergere quelle a quote relative più basse;
	\item eventi \emph{bankfull}, cioè piene di grande magnitudine, oltrepassano i \SI{3}{\m} di livello idrometrico e hanno un tempo di ritorno superiore a \SIrange[range-phrase={-}]{3}{4}{\anni}; l'alveo viene completamente sommerso dall'acqua, così come la stragrande maggioranza delle isole.
\end{itemize}
%

Il \emph{natural flow regime} (regime naturale delle portate) è il pattern caratteristico che presentano le portate in fiume in base ad abbondanza, stagionalità e variabilità; di questo regime, sono state individuate cinque componenti principali: intensità, durata, stagionalità, frequenza e tasso di cambiamento delle portate \squarecite{Poff:1997}.
\\
Si ritiene che gli effetti di una piena sulle isole non dipendano soltanto dall'intensità (cioè dal livello raggiunto durante il picco) ma anche dalla durata: a parità di picco, un evento più lungo probabilmente eroderà un maggiore areale di isole rispetto ad un evento molto breve.
La stagionalità e il tasso di cambiamento delle portate sono fattori che esercitano un'influenza maggiore sulla crescita ed espansione della vegetazione, più che sull'erosione (tuttavia, la stagionalità diventa rilevante quando si considerano gli effetti secondari di resistenza che le piante esercitano alla corrente grazie alla presenza, o meno, delle foglie). 
\\
Per tenere contemporaneamente conto dell'intensità, della durata e del tempo di ritorno delle piene, si è calcolato per ogni intervallo di tempo tra due immagini successive l'integrale temporale dei livelli sopra un livello soglia definito dal grafico in \cref{graph:tr-picchi}, scegliendo un tempo di ritorno.
Si è scelto di misurare l'integrale in \si{[\m\giorno]}.
L'integrale è equivalente ad un evento di piena; tale evento ha durata pari a~\SI{1}{\giorno} e intensità, in~\si{[\m]}, pari alla somma tra i valori del livello soglia in \si{[\m]} e dell'integrale in \si{[\m\giorno\per\giorno]}.
Ad esempio, si consideri un livello soglia di~\SI{2}{\m} e si supponga di aver calcolato, in un periodo di tempo, un integrale dei livelli sopra soglia pari a~\SI{1}{\m\giorno}; questo integrale è equivalente ad una piena con livello costante di~$\SI{2}{\m} + \SI{1}{\m} = \SI{3}{\m}$.
\\
Poiché non è disponibile un'affidabile scala di deflusso per convertire i valori di livello in portata, l'integrale costituisce un sostituto rispetto alla somma delle portate sopra la soglia che fluiscono durante una piena.
\\
Un esempio visuale di integrale per due livelli soglia è mostrato nei grafici in \cref{graph:esempio-integrale-livelli}.
\\
Se tra due immagini successive l'integrale è nullo, allora non ha avuto luogo alcun evento con picco superiore al livello soglia corrispondente al tempo di ritorno scelto;
se l'integrale non è nullo, allora ci sono state piene sopra il livello soglia;
se un'integrale è maggiore di altri, nell'intervallo tra le immagini sono avvenute piene particolarmente intense e/o durature, anche con un'alta frequenza.
%
\begin{figure}
	\centering
	\tikzsetnextfilename{esempio_integrale_livelli}
\begin{tikzpicture}
	\begin{groupplot}[
		group style = {
			group size = 2 by 2,
			y descriptions at = edge left,
			xlabels at = edge top,
			horizontal sep = 0.5cm,
			vertical sep = 0.5cm,
		},
		width = 0.5\textwidth,
		height = 0.5\textwidth,
%		ymin = 0.5,
%		ymax = 3.5,
		enlarge y limits = 0.05,
		enlarge x limits = 0.05,
		ylabel = {Livello idrometrico \si{[\m]}},
		xlabel = {Tempo di ritorno \si{[\anni]}},
		]
	
	\nextgroupplot[
			xmode = log,
        	log ticks with fixed point,
			grid = major,
        ]
		\addplot[
			blue,
			no markers,
			]
        	table [x = tr_anni, y = picchi] {graphics/data/tr_picchi.txt};
        	
        \draw[->, orange, very thick] (0.2,1) -- (0.2,1.5) -- (0.1,1.5);
		
		\node at (axis description cs: 1,0) [draw = black, fill = white, anchor = south east, align = left] {TR \SIrange[range-phrase={-}, range-units = single]{2}{3}{\mesi} \\ Livello \SI{1.5}{\m}};
	
	\nextgroupplot[
			xmode = log,
        	log ticks with fixed point,
			grid = major,
        ]
		\addplot[
			blue,
			no markers,
			]
        	table [x = tr_anni, y = picchi] {graphics/data/tr_picchi.txt};
        	
        \draw[->, green!70!black, very thick] (0.45,1) -- (0.45,2) -- (0.1,2);
		
		\node at (axis description cs: 1,0) [draw = black, fill = white, anchor = south east, align = left] {TR \SIrange[range-phrase={-}, range-units = single]{4}{5}{\mesi} \\ Livello \SI{2}{\m}};
    
	\nextgroupplot[
			date coordinates in = x,
			xticklabel = {$\year-\month-\day$},
			xticklabel style = {
				rotate = 80,
				anchor = near xticklabel,
				font = \footnotesize,
			},
			xmin = 2002-05-18 00:00,
			xmax = 2002-06-12 23:30,
		]
		\addplot[
			blue,
			no markers,
			name path = livelli,
			]
        	table [x = data, y = livello, col sep = comma] {graphics/data/Idro_primo_intervallo.txt};
        	
        \addplot[
        	dashed,
        	very thick,
        	orange,
        	name path = soglia,
        	]
        	coordinates {(2002-05-18 00:00, 1.5) (2002-06-12 23:30, 1.5)};
        	
		\addplot fill between [
			of = soglia and livelli,
			split,
			every segment no 0/.style = {white},
			every segment no 1/.style = {orange},
			every segment no 2/.style = {white},
			every segment no 3/.style = {orange},
			every segment no 4/.style = {white},
			every segment no 5/.style = {white},
		];
		
		\node at (axis description cs: 1,0) [draw = black, fill = white, anchor = south east,] {$Int = \SI{0.90}{\m\giorno}$};
    
	\nextgroupplot[
			date coordinates in = x,
			xticklabel = {$\year-\month-\day$},
			xticklabel style = {
				rotate = 80,
				anchor = near xticklabel,
				font = \footnotesize,
			},
			xmin = 2002-05-18 00:00,
			xmax = 2002-06-12 23:30,
		]
		\addplot[
			blue,
			no markers,
			name path = livelli,
			]
        	table [x = data, y = livello, col sep = comma] {graphics/data/Idro_primo_intervallo.txt};
        	
        \addplot[
        	dashed,
        	very thick,
        	green!70!black,
        	name path = soglia,
        	]
        	coordinates {(2002-05-18 00:00, 2) (2002-06-12 23:30, 2)};
        	
		\addplot fill between [
			of = soglia and livelli,
			split,
			every segment no 0/.style = {white},
			every segment no 1/.style = {green!70!black},
			every segment no 2/.style = {green!70!black},
			every segment no 3/.style = {white},
			every segment no 4/.style = {white},
		];
		
		\node at (axis description cs: 1,0) [draw = black, fill = white, anchor = south east,] {$Int = \SI{0.12}{\m\giorno}$};
	\end{groupplot}
\end{tikzpicture}	
	\caption[esempio di integrale dei livelli]{esempio grafico di integrale dei livelli per due tempi di ritorno considerando l'intervallo tra le immagini \AST{} del 2002-05-18 e del 2002-06-12; in alto è riportata la statistica dei tempi di ritorno per i picchi con livello superiore a~\SI{1}{\m} (grafico in \cref{graph:tr-picchi}); in basso è mostrato l'andamento orario dei livelli per il confronto fra le immagini \AST{} del~2002-05-18 e~2002-06-12; l'integrale è pari all'area colorata e il suo valore in \si{[\m\giorno]} è riportato in basso nei grafici.}
	\label{graph:esempio-integrale-livelli}
\end{figure}
%

\subsection{Metodologia}
Al fine di ottenere relazioni significative si sono usate (non sempre contemporaneamente) queste metodologie:
%
\begin{itemize}
	\item tenere conto dei tempi di ritorno di \SIrange[range-phrase = {-}, range-units = single]{2}{3}{\mesi}, \SIrange[range-phrase = {-}, range-units = single]{4}{5}{\mesi}, \SI{1}{\anno} e \SI{2}{\anni}, cui corrispondono rispettivamente livelli soglia di \SI{1.5}{\m}, \SI{2}{\m}, \SI{2.4}{\m} e \SI{2.8}{\m};
	\item dividere la quantità rappresentata su un asse per un'altra quantità (ad esempio dividere le isole erose per le isole presenti prima dell'erosione);
	\item accorpare i dati di più tratti adiacenti (ad esempio dei primi 4 tratti);
	\item escludere dall'analisi alcuni tratti con caratteristiche peculiari, come il tratto~9, dove è presente l'isola su roccia di Cornino, e il tratto~23, in quanto presenta una morfologia nettamente diversa rispetto agli altri tratti;
	\item ignorare i punti che mostrano erosione anche per integrali nulli, in quanto gli areali di erosione sono molto modesti e sono con tutta probabilità dovuti agli errori di correzione della georeferenziazione attuata nella fase di elaborazione del cambiamento;
	\item applicare ad un asse una scala logaritmica in base~\num{10} per utilizzare leggi di potenza nelle regressioni;
	\item considerare un numero minimo di~5 punti per effettuare e ritenere valide le regressioni (per due punti passa una ed una sola retta, e in tal caso una regressione lineare darebbe ottimi risultati, ma poco validi; con un maggior numero di punti l'affidabilità dei risultati migliora);
	\item considerare come valide le regressioni che mostrano $R^2 \num{> 0.3}$ e $P_\mathrm{value} \num{< 0.05}$;
	\item limitare agli anni più recenti la suddivisione in classi d'età (dall'immagine del~2007-09-21), poiché tale divisione è debolmente valida nei primi anni di osservazione (si veda la sezione~\ref{sec:eta}).
\end{itemize}
